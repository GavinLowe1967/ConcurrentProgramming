

\begin{slide}
\heading{A partial queue}

Remember that we earlier decided to return |None| if |dequeue| is called when
the queue is empty.  An alternative is to block that thread until there is a
value to be dequeued.  This gives a partial queue, with the following
interface. 

\begin{scala}
trait PartialQueue[T]{
  /** Enqueue x. */
  def enqueue(x: T): Unit

  /** Dequeue a value.  Blocks until the queue is non-empty. */
  def dequeue: T

  /** Shut down the queue. */
  def shutdown
}
\end{scala}
Note that if the queue is empty and all threads are attempting a dequeue, the
system will deadlock. 
\end{slide}

%%%%%

\begin{slide}
\heading{A partial queue}

\begin{scala}
class ServerPartialQueue[T] extends PartialQueue[T]{
  // Channel for dequeueing
  private val dequeueChan = new SyncChan[T]

  def dequeue: T = dequeueChan?()

  private def server = thread{
    val queue = new Queue[T]
    serve(
      enqueueChan =?=> { x => queue.enqueue(x) }
      | queue.nonEmpty && dequeueChan =!=> queue.dequeue
    )
  }

  ... // other parts as earlier
}
\end{scala}
\end{slide}

%%%%%

\begin{slide}
\heading{Testing}

We can test this queue for linearizability, much as for the previous version.
However, we have to amend the sequential specification object to recognise
that a dequeue cannot happen when the queue is empty. 
\begin{scala}
  def seqDequeue(q: S) : (Int, S) = {
    require(q.nonEmpty); q.dequeue
  }
\end{scala}
An operation on the concurrent datatype cannot be linearized in a state where
the corresponding operation on the sequential datatype throws an
|IllegalArgumentException|.

Also we have to be careful that we never get into a deadlocked state where all
the threads are attempting a dequeue from the empty queue.  I arrange for half
the threads to perform just enqueues, and the others to perform just dequeues.  
\end{slide}

%%%%%

\begin{slide}
\heading{Termination}

Suppose the queue is empty, and all the threads are attempting a dequeue.
With the previous design, the system would be deadlocked.

A better approach would be for the server to detect such a state, and react
accordingly.  We choose to shut down the queue in this case, and arrange for
each dequeue attempt to throw a |Stopped| exception (an alternative would be
to use an |Option| value).

In order to do this, the server needs to know how many threads are attempting
a dequeue.  That means we can't just block the |dequeueChan| channel when the
queue is empty.  Instead, we arrange for the dequeueing thread to pass a reply
channel on the |dequeueChan|.  These reply channels are queued until the
request can be made.  If the number of queued reply channels equals the number
of workers, the queue shuts down.
\end{slide}

%%%%%

\begin{slide}
\heading{Termination}

\begin{scala}
/** A partial queue that terminates if all worker threads are attempting to
  * dequeue, and the queue is empty.
  * @param numWorkers the number of worker threads. */
class TerminatingPartialQueue[A](numWorkers: Int){
  private type ReplyChan = Chan[A]
  /** Channel for dequeueing. */
  private val dequeueChan = new SyncChan[ReplyChan]

  /** Attempt to dequeue a value.
    * @throws Stopped exception if the queue has been shutdown. */
  def dequeue: A = {
    val reply = new SyncChan[A] // or a BuffChan[A]
    dequeueChan!reply
    reply?()
  }
  ... 
}
\end{scala}
\end{slide}

%%%%%

\begin{slide}
\heading{Termination}

We might also want to externally shut down the server, (e.g.~in a search
algorithm, if we find what we're looking for).

It's necessary for the server to control termination in this case, so we
arrange for the |shutdown| method to send a message to the server. 
%
\begin{scala}
  private val shutdownChan = new SyncChan[Unit]

  /** Shut down this queue. */
  def shutdown = attempt{ shutdownChan!(()) }{ }
\end{scala}
Note: it's possible that the server has already terminated and |shutdownChan|
closed, in which case we catch the |Stopped| exception.

%% The construct
%% \begin{scala}
%%   attempt{ p }{ q }
%% \end{scala}
%% Executes |p|; but if |p| throws a |Stopped| exception, that exception is
%% caught, and |q| is executed. 
\end{slide}
  
%%%%%

\begin{slide}
\heading{The server}

\begin{scala}
  private def server = thread("server"){
    // Currently held values
    val queue = new Queue[A]()
    // Queue holding reply channels for current dequeue attempts.
    val waiters = new Queue[ReplyChan]()
    // Invariant: queue.isEmpty or waiters.isEmpty

    // Termination: signal to all waiting workers
    def close = {
      for(c <- waiters) c.close
      enqueueChan.close; dequeueChan.close; shutdownChan.close
    }
    ...
  }
\end{scala}
\end{slide}

%%%%%

\begin{slide}
\heading{The server main loop}

\begin{scala}
    serve(
      enqueueChan =?=> { x => 
        if(waiters.nonEmpty){ // pass x directly to a waiting dequeue
          assert(queue.isEmpty); waiters.dequeue!x
        }
        else queue.enqueue(x)
      }
      | dequeueChan =?=> { reply =>
          if(queue.nonEmpty) reply!(queue.dequeue) // service request immediately
          else{                                        // thread has to wait
            waiters.enqueue(reply)
            if(waiters.length == numWorkers) close
          }
        }
      | shutdownChan =?=> { _ => close }
    )
\end{scala}
\end{slide}
