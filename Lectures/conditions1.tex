
%% \begin{slide}
%% \heading{Multiple producers and consumers}

%% Recall the |Slot| example.
%% \begin{scala}
%% class Slot[T]{
%%   private var value = null.asInstanceOf[T]
%%   private var filled = false

%%   def put(v:T) = synchronized{
%%     while(filled) wait() // wait for the slot to be emptied
%%     value = v; filled = true; notify()
%%   }

%%   def get : T = synchronized{
%%     while(!filled) wait() // wait for the slot to be filled
%%     filled = false; notify(); value
%% } }
%% \end{scala}
%% This works correctly with a single producer and consumer, but not with
%% multiple ones. 
%% \end{slide}

%%%%%

%% \begin{slide}
%% \heading{Multiple producers and consumers}

%% Suppose we run the |Slot| with multiple producers and consumers.  It turns out
%% that we need to use |notifyAll| rather than |notify|.
%% %
%% Consider the following execution, if |notify| is used.
%% %
%% \begin{enumerate}
%% \item Two consumers run, and both have to wait.

%% \item Producer~1 runs, sets |filled = true|, wakes up consumer~1, and returns.

%% \item Producer~2 runs, but has to wait.

%% \item Consumer~1 runs, sets |filled = false|, calls |notify()| which wakes up
%% consumer~2, and returns.

%% \item Consumer~2 runs, finds |filled = false|, so waits again. 
%% \end{enumerate}
%% %
%% Now producer~2 and consumer~2 are both waiting, but they should have been able
%% to exchange data. 

%% If consumer~1 calls |notifyAll|, things will work correctly. 
%% \end{slide}

%%%%%

\begin{slide}
\heading{A slot for multiple producers and consumers}

Recall the |Slot| example, to be used by multiple producers and consumers,
implemented using a JVM monitor.
%
\begin{scala}
class Slot2[T]{
  private var value = null.asInstanceOf[T]
  private var filled = false

  def put(v:T) = synchronized{
    while(filled) wait()
    value = v; filled = true; notifyAll()
  }

  def get : T = synchronized{
    while(!filled) wait()
    filled = false; notifyAll(); value
  }
}
\end{scala}
\end{slide}

%%%%%

\begin{slide}
\heading{Targeted notifications}

The code on the previous slide isn't ideal: most threads that are awoken by a
|notifyAll| will be forced to wait again.

We would like to be able to target a |notify| by a |put| towards a thread
waiting in |get|; and target a |notify| by a |get| towards a thread
waiting in |put|.  This isn't possible using standard |notify|s.
\end{slide}

%%%%%

%% \begin{slide}
%% \heading{CSO monitors}

%% A CSO monitor is an extension of JVM monitors that allows more targeted
%% notifications.  

%% A new monitor is created with
%% \begin{scala}
%%   private val monitor = new Monitor
%% \end{scala}

%% The |withLock| method on monitors acts rather like a |synchronized| block:
%% \begin{scala}
%%   def put(v: T) = monitor.withLock{ ... }
%% \end{scala}  
%% At most one thread can be inside any |withLock| block on a given monitor.
%% \end{slide}

%%%%%

\begin{slide}
\heading{Conditions}

Signalling can be done via a |Condition|.  An SCL |Lock| can be
associated with any number of |Conditions|, created using the |newCondition|
operation.  For example
%
\begin{scala}
  private val lock = new Lock
  // Conditions for signalling the slot is empty or non-empty, respectively.
  private val empty = lock.newCondition
  private val nonEmpty = lock.newCondition
\end{scala}
%
(Choose good names for your conditions, and write comments to make clear their
roles.)
\end{slide}

%%%%%

\begin{slide}
\heading{Conditions}

Each condition has operations as follows:
%
\begin{itemize}
\item |await()|: wait for a signal on this condition;
\item |signal()|: signal to a thread waiting on this condition;
\item |signalAll()|: signal to all threads waiting on this condition.
\end{itemize}
%
Each operation should be performed while holding the related lock. 

An |await| releases the lock.  When it receives a signal, it has to re-obtain
the lock before proceeding.

We can implement monitors using |Lock|s and |Condition|s. 
\end{slide}

%%%%%

\begin{slide}
\heading{The slot, using conditions}

\begin{scala}
class MonitorSlot[T]{
  private var value = null.asInstanceOf[T]
  private var filled = false
  private val lock = new Lock
  // conditions for signalling the slot is empty or non-empty, respectively.
  private val empty, nonEmpty = lock.newCondition

  def put(v: T) = lock.mutex{
    while(filled) empty.await()           // wait for the slot to be emptied (1)
    value = v; filled = true; nonEmpty.signal() // signal to a get at (2)
  }

  def get: T = lock.mutex{
    while(!filled) nonEmpty.await()     // wait for the slot to be filled (2)
    filled = false; empty.signal()        // signal to a put at (1)
    value
} }
\end{scala}
\end{slide}

%%%%%
{\advance \slideheight by 2mm
\begin{slide}
\heading{Waiting for a condition}

It is often useful to wait on a |Condition| until some test becomes true.  The
following operation on a |Condition| provides this.
\begin{scala}
  def await(test: => Boolean): Unit = while(!test) await()
\end{scala}

For example
%
\begin{scala}
  def put(v: T) = lock.mutex{
    empty.await(!filled)                  // wait for the slot to be emptied (1)
    value = v; filled = true; nonEmpty.signal() // signal to a get at (2)
  }

  def get: T = lock.mutex{
    nonEmpty.await(filled)               // wait for the slot to be filled (2)
    filled = false; empty.signal()       // signal to a put at (1)
    value
  }
\end{scala}
\end{slide}}

%%%%%

\begin{slide}
\heading{Spurious wake-ups}

The implementation of SCL |Condition|s protects against spurious
wake-ups: a waiting thread will wake up \emph{only} when another
thread sends a signal.  (The same is not true of the implementation in
\texttt{java.util.concurrent.locks.Condition}!)

However, it might sometimes be necessary to re-check the relevant
condition after an |await|, in case another thread has invalidated it.
For example, on the previous slide, a waiting |get| may receive a
signal from a |put|; but another |get| may take the deposited value
before the first |get| obtains the lock.
\end{slide}

%%%%%

\begin{slide}
\heading{JVM monitors versus SCL monitors}

SCL monitors are convenient when one wants to target signals.
However, their implementation is more expensive than JVM monitors: use
SCL monitors only when there is a good reason to do so.
\end{slide}


%%%%%%%%%%%%%%%%%%%%%%%%%%%%%%%%%%%%%%%%%%%%%%%%%%%%%%%%%%%%

\begin{slide}
\heading{A bounded partial queue}

Concurrent datatypes can be implemented using monitors. 

We will implement a bounded partial queue.
%
\begin{itemize}
\item |dequeue| operations will block while the queue is empty;

\item |enqueue| operations will block while the queue is full.
\end{itemize}
\end{slide}

%%%%%

\begin{slide}
\heading{A bounded partial queue}

\begin{scala}
/** A bounded partial queue implemented as a monitor. */
class BoundedMonitorPartialQueue[T](bound: Int) extends PartialQueue[T]{
  /** The queue itself. */
  private val queue = new scala.collection.mutable.Queue[T]

  /** A lock, to control the synchronisations. */
  private val lock = new Lock

  /** Condition for signalling that the queue is not full. */
  private val notFull = lock.newCondition

  /** Condition for signalling that the queue is not empty. */
  private val notEmpty = lock.newCondition
  ...
}
\end{scala}
\end{slide}

%%%%%

\begin{slide}
\heading{A bounded partial queue}

\begin{scala}
  /** Enqueue x.  Blocks until the queue is not full. */
  def enqueue(x: T) = lock.mutex{
    notFull.await(queue.length < bound)           // wait for a signal (1)
    queue.enqueue(x); notEmpty.signal()           // signal to a dequeue at (2)
  }

  /** Dequeue a value.  Blocks until the queue is non-empty. */
  def dequeue: T = lock.mutex{
    notEmpty.await(queue.nonEmpty)               // wait for a signal (2)
    val result = queue.dequeue; notFull.signal()  // signal to an enqueue at (1)
    result
  }
\end{scala}
\end{slide}


  % /** Shut down the queue. */
  % def shutdown = {} // nothing needs doing
