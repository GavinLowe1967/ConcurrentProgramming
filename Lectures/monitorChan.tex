\begin{slide}
\heading{A synchronous channel}

We will adapt the |Slot| example to implement a synchronous channel, suitable
for use by a \emph{single} sender and a \emph{single} receiver.

We need to arrange for the sender to wait until it receives a signal from
the receiver. 
\end{slide}

%%%%%

\begin{slide}
\heading{A synchronous channel}

\begin{scala}
/** A one-one synchronous channel passing data of type A, implemented using a
  * monitor. */
class SyncChan[A]{
  /** The current or previous value. */
  private var value = null.asInstanceOf[A]

  /** Is the current value of £value£ valid, i.e. ready to be received? */
  private var full = false

  /** Send x on the channel, synchronously. */
  def send(x: A) = ...

  /** Receive a value on the channel. */
  def receive: A = ...
}
\end{scala}
\end{slide}


%%%%%

\begin{slide}
\heading{A synchronous channel}

\begin{scala}
  def send(x: A) = synchronized{
    assert(!full)           // previous round must be complete
    value = x; full = true // deposit my value 
    notify()                // signal to receiver at (2)
    while(full) wait()      // wait for receiver (1)
  }

  def receive: A = synchronized{
    while(!full) wait()     // wait for sender (2)
    full = false            // clear value 
    notify()                // notify sender at (1)
    value
  }
\end{scala}
\end{slide}

%%%%%

\begin{slide}
\heading{Testing}

We can test the channel using logging.  We arrange for a single sender and a
single receiver to use the channel, writing events to the log before and after
each operation.  We then traverse the log, checking that no call returns
before both the |send| and |receive| operations have been called, and that
each value received matches the value sent. 
\end{slide}

