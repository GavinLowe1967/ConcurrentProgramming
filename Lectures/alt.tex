\documentclass[notes,color]{sepslide0}
\usepackage{graphicx}
\usepackage[overheads]{mysepslides}
\usepackage{tech,graphicx,url,tikz,scalalistings}

\title{Alternation} 
\author{Gavin Lowe \\ with thanks to Bernard Sufrin}

% \everymath{\color{Plum}}

\def\scalacolour{\color{violet}}

\begin{document}

\begin{slide}
  
  \Title

\end{slide}

%%%%%


\begin{slide}
\heading{The need for alternation}

At present, we can write processes that try to send  or receive on a
\emph{single} channel.

However, it's often useful to be able to try to send or receive on either of
two or more channels: the \SCALA{alt} construct does that for us.
\end{slide}

%%%%%

\begin{slide}
\heading{\protect\SCALA{alt}: simple syntax}

The process
%
\begin{scala}
alt( 
  inport£$_1$£ =?=> {f£$_1$£}
  | ...
  | inport£$_n$£ =?=> {f£$_n$£}
)
\end{scala}
%
waits until one of the input ports \SCALA{inport}$_1$, \ldots,
\SCALA{inport}$_n$ is ready to communicate, reads a value~$v$ from the port, and
applies the relevant function |f|$_i$ to~$v$.

%% A construct of the form \SCALA{inport =?=> \{f\}} is referred to as an
%% \emph{event}.
\end{slide}

%%%%%

\begin{slide}
\heading{Example: a simple tagger}

The following process repeatedly inputs from one of two input ports, tags the
value input, and outputs it.
%
\begin{scala}
def tagger[T](l: ?[T], r: ?[T], out: ![(Int, T)]) = thread{
  repeat{
    alt ( l =?=> { x => out!(0, x) }
        | r =?=> { x => out!(1, x) }
    )
  }
  l.close; r.close; out.endOfStream
}
\end{scala}

Exercise: design a corresponding de-tagger.
\end{slide}


%%%%%

\begin{slide}
\heading{Guards}

It's sometimes useful to specify that a particular inport should be considered
only if some condition, or \emph{guard}, is true.

In the  process
%
\begin{scala}
alt( guard£$_1$£ && inport£$_1$£ =?=> {f£$_1$£}
   | ...
   | guard£$_n$£ && inport£$_n$£ =?=> {f£$_n$£}
   )
\end{scala}
%
a communication on each |inport|$_i$ is possible only if the corresponding
|guard|$_i$ is true.
\end{slide}

%%%%%

\begin{slide}
\heading{Guards}

Each guard is evaluated once, and should not have side effects.

If a guard evaluates to true and the inport is open, the corresponding
branch is \emph{feasible}.  If no branch is feasible, the |alt| throws
an |AltAbort| exception (a subclass of \SCALA{Stopped}).

If a branch is feasible and the inport is available for communication, then
the branch is \emph{ready}.

The |alt| waits until a branch is ready, and receives from the inport.  If
several are ready, it chooses between them.  If all the branches become
infeasible (because of channels being closed), the |alt| throws an |AltAbort|
exception.

\SCALA{inport =?=> ...} is equivalent to \SCALA{true && inport =?=> ...}.
\end{slide}

%%%%%

\begin{slide}
\heading{\scalashape serve}

Using an \SCALA{alt} inside a \SCALA{repeat} is very common.  The construct
 %
\begin{scala}
repeat{ 
  alt( g£$_1$£ && p£$_1$£ =?=> {f£$_1$£} | ... | g£$_n$£ && p£$_n$£ =?=> {f£$_n$£} ) 
}
\end{scala}
%
will terminate when, for every branch, either the guard is false or
the port is closed (or one of the |f|$_i$ throws a |Stopped| exception).
\end{slide}

%%%%%

\begin{slide}
\heading{\scalashape serve}

The construct 
%
\begin{scala}
serve( g£$_1$£ && p£$_1$£ =?=> {f£$_1$£} | ... | g£$_n$£ && p£$_n$£ =?=> {f£$_n$£} )
\end{scala}
%
is very similar to a corresponding |repeat{ alt(...) }| construct, except the
latter creates a new \SCALA{alt} object on each iteration, whereas the
|serve| creates a single \SCALA{alt} object which is used repeatedly.

Both constructs evaluate each guard expression~|g|$_i$ and each port
expression |p|$_i$  on each iteration.
\end{slide}

%%%%%

\begin{slide}
\heading{Fairness}

The implementation of an |alt| tests whether its branches are ready in the order
given.  In the |repeat{ alt(...) }| construct, a new |alt| object is created
for each iteration, so if the first branch is repeatedly ready, it will be
repeatedly selected, and the other branches will be starved.

By contrast, the |serve| aims to be fair.  It uses the same |alt| on each
iteration.  It remembers which branch was selected on the previous iteration,
and tests whether branches are ready starting from the following one (looping
round).

This means that if a particular branch is continuously ready, and the |serve|
performs enough iterations, then that branch will eventually be selected.  We
say that the |serve| is \emph{fair} to each branch.
\end{slide}


%%%%%

\begin{slide}
\heading{\protect\SCALA{serve}}

Here's the tagger again.
%
\begin{scala}
def tagger[T](l: ?[T], r: ?[T], out: ![(Int, T)]) = thread{
  serve( l =?=> { x => out!(0, x) }
       | r =?=> { x => out!(1, x) }
  )
  l.close; r.close; out.endOfStream
}
\end{scala}
%
Note that this is fair to its two input ports. 

The |serve| loop terminates when either both |l| and~|r| are closed, or |out|
is closed. 
\end{slide}


%%%%%

\begin{slide}
\heading{About fairness}

Fairness crops up in a number of scenarios in concurrent programming.

A typical fairness property is that if a particular option is continually
logically possible, then it eventually happens.  Fairness for an |alt|
construct means that if a particular branch is continually ready, and the
|alt| performs enough iterations, then eventually that branch is selected.

Note that ``fair'' doesn't necessarily mean equable: a construct that chooses
option~$A$ 99\% of the time, and chooses option~$B$ 1\% of the time is fair
but not equable.  

In some scenarios, achieving fairness has a performance overhead.  One should
consider whether or not fairness is desirable.  Often one can achieve higher
throughput without fairness.
\end{slide}

%%%%%

%%%%%%%%%%%%%%%%%%%%%%%%%%%%%%%%%%%%%%%%%%%%%%%%%%%%%%%%%%%%

\begin{slide}
\heading{The Dining Philosophers}

The story:
\begin{itemize}
\item Five philosophers spend their lives thinking and eating.

\item 
They share a common dining room, which has a circular table with five chairs
around it, a plate in front of each chair, and a big bowl of spaghetti in the
middle.

\item 
There are  five forks, placed between the five plates at the table.

\item 
After thinking for a while a philosopher gets hungry, picks up the fork to her
left as soon as it's available, then picks up the fork to her right as soon as
it's available.

\item 
Once she has two forks, she serves herself and spends some time eating.

\item 
Then she puts the forks down and does some more thinking.
\end{itemize}
\end{slide}

% %%%%%

\begin{slide}
\heading{The Dining Philosophers}

If all five philosophers get hungry at about the same time and pick up their
left fork, then they all starve!

How can we simulate this?
\end{slide}

%%%%%

\begin{slide}
\heading{Dining Philosophers: the basics}

\begin{scala}
import ox.scl._

/** Simulation of the Dining Philosophers example. */
object Phils{
  val N = 5 // Number of philosophers

  // Simulate basic actions
  def eat = Thread.sleep(500)
  def think = Thread.sleep(scala.util.Random.nextInt(900))
  def pause = Thread.sleep(500)
  ...
}
\end{scala}
\end{slide}

%%%%%

\begin{slide}
\heading{Picking and dropping philosopher}

Each philosopher will send ``pick'' and ``drop'' messages to her forks, which
we simulate using the following values.
%
\begin{scala}
  type Command = Boolean
  val Pick = true; val Drop = false
\end{scala}
\end{slide}

%%%%%

\begin{slide}
\heading{A philosopher}

We provide each philosopher with two channels, to her left- and right-hand
fork, respectively.  
\begin{scala}
  /** A single philosopher. */
  def phil(me: Int, left: ![Command], right: ![Command]) = thread("Phil"+me){
    repeat{
      think
      println(me+" sits"); pause
      left!Pick; println(me+" picks up left fork"); pause
      right!Pick; println(me+" picks up right fork"); pause
      println(me+" eats"); eat
      left!Drop; pause; right!Drop; pause
      println(me+" leaves")
    }
  }
\end{scala}
\end{slide}

%%%%%

\begin{selfnote}
The syntax 
\begin{scala}
thread(name){ <code> }
\end{scala}
Creates a process with name \SCALA{name}.
\end{selfnote}

%%%%%

\begin{slide}
\heading{A fork}

\begin{scala}
  /** A single fork. */
  def fork(me: Int, left: ?[Command], right: ?[Command]) = thread("Fork"+me){
    serve(
      left =?=> { 
        x => assert(x == Pick); val y = left?(); assert(y == Drop) 
      }
      | 
      right =?=> { 
        x => assert(x == Pick); val y = right?(); assert(y == Drop) 
      }
    )
  }
\end{scala}
\end{slide}

%%%%%


\begin{slide}
\heading{Putting it together}

\begin{scala}
  /** The complete system. */
  def system = {
    // Channels to pick up and drop the forks:
    val philToLeftFork, philToRightFork = Array.fill(N)(new SyncChan[Command])
    // philToLeftFork(i) is from Phil(i) to Fork(i);
    // philToRightFork(i) is from Phil(i) to Fork((i-1) mod N)
    val allPhils = || ( 
      for (i <- 0 until N) yield phil(i, philToLeftFork(i), philToRightFork(i))
    )
    val allForks = || ( 
      for (i <- 0 until N) yield
        fork(i, philToRightFork((i+1)%N), philToLeftFork(i))
    )
    allPhils || allForks
  }

  def main(args : Array[String]) = run(system) 
\end{scala}
\end{slide}

%%%%%

\begin{slide}
\heading{Running the system}

When we run this, sometimes it deadlocks almost immediately.
A typical deadlocking trace is:
%
\begin{scala}
1 sits,  0 sits,  4 sits,  2 sits,  1 picks up left fork
3 sits,  0 picks up left fork,  4 picks up left fork
2 picks up left fork,  3 picks up left fork
\end{scala}
%
When it deadlocks, typing control-backslash (on Linux/Unix machines) gives the
thread dump, which can help with debugging.  (If we have named threads, the
thread dump will use these names.)  

However, sometimes the system runs for a long time without deadlocking.
\end{slide}

%%%%%

\begin{slide}
\heading{Logging}

In the above, we used |println| statements so that we could understand
what happened.  However, this style is often not convenient.
Normally, a better way is to use logging.

The class |Log| provides log objects.
%
A new log storing events of type~|A|, suitable for |p| threads,
can be defined by
\begin{scala}
  val log = new Log[A](p)
\end{scala}
%
This provides operations
%
\begin{itemize}
\item |def add(me: Int, x: A)|, which adds |x| to the log,
performed by thread~|me|;

\item |def get: Array[A]|, which gets the contents of the log;

\item |def toFile(fname: String = "/tmp/logFile")|, which writes the
contents of the log to the file~|fname|.
\end{itemize}
\end{slide}

%%%%%

\begin{slide}
\heading{Using logging}

\begin{scala}
  val log = new Log[String](N)
 
  /** A single philosopher. */
  def phil(me: Int, left: ![Command], right: ![Command]) = thread("Phil"+me){
    repeat{
      think
      log.add(me, me+" sits"); pause
      left!Pick; log.add(me, me+" picks up left fork"); pause
      right!Pick; log.add(me, me+" picks up right fork"); pause
      log.add(me, me+" eats"); eat
      left!Drop; pause; right!Drop; pause
      log.add(me, me+" leaves")
      if(me == 0) print(".")
    }
  }
\end{scala}
\end{slide}

%%%%%

\begin{slide}
\heading{Using logging}

In many scenarios, when the system terminates, we can either write the
log to a file, or analyse the log using some code.

However, this won't work when the system deadlocks.  Instead, we need
to arrange for the log to write itself to a file when the user
interrupts the program.  This can be done with the
|writeToFileOnShutdown| method:
\begin{scala}
  def main(args : Array[String]) = {
    log.writeToFileOnShutdown("philsLog.txt")
    run(system)
  }
\end{scala}
\end{slide}


%%%%%

\begin{slide}
\heading{Logging}

Internally, each thread uses its own thread-local log, to avoid race
conditions, and for efficiency.  In the thread-local logs, each value is
stored together with a timestamp (more precisely, the time elapsed since the
|Log| object was created, to avoid problems with timestamp wrap-around).

The |get| operation merges the thread-local logs according to
timestamps.
\end{slide}

%%%%%


\begin{slide}
\heading{Logging}

Note that using timestamps in this way assumes that the clocks are loosely
synchronised across cores, and that the granularity of the clocks is
sufficiently fine, so that if one |add| event logically happens before
another, the former receives a strictly smaller timestamp.

The above assumption seems to be sound in Linux, but not in Windows.  If you
must use Windows, the  class |SharedLog| provides the same
interface.  However, it will give worse performance, and this might affect the
likelihood of detecting bugs. 
\end{slide}



%%%%%%%%%%%%%%%%%%%%%%%%%%%%%%%%%%%%%%%%%%%%%%%%%%%%%%%%%%%%

\begin{slide}
\heading{\protect\SCALA{alt}s versus shared channels}

Sometimes a shared channel can produce the same effect as an \SCALA{alt}.

For example, consider a variant of the dining philosophers where each
philosopher has channels \SCALA{pick} and \SCALA{drop} for each of its
forks, and where each fork has single \SCALA{pick} and \SCALA{drop}
channels, on which it can receive messages from either of the adjacent
philosophers:
%
\begin{scala}
  def Fork(me: Int, pick: ?[Unit], drop: ?[Unit]) = thread("Fork"+me){
    repeat{ pick?(); drop?() }
  }
\end{scala}
%
%% Note that the \SCALA{pick} communication could be from either neighbouring
%% philosopher, so these need to be |ManyOne| channels.
Note that we need separate |pick| and |drop| channels.
\end{slide}

%%%%%

\begin{slide}
\heading{Practical 2: Dining Philosophers}

The second practical asks you to investigate various ways to avoid such
deadlocks. 
%
\begin{itemize}
\item Make one philosopher right-handed, rather than all being left-handed; 

\item Using a butler process, that allows at most $N-1$ philosophers to sit at
a time;

\item Using timeouts: if a philosopher fails to get a fork within a reasonable
amount of time, she gives up, and tries again later.
\end{itemize}
\end{slide}


%%%%%

\begin{slide}
\heading{\protect\SCALA{alt}s with outports}

It is also possible to use an output port within an \SCALA{alt}:
%
\begin{scala}
bool && outport =!=> { expression }
\end{scala}
If |bool| is true, when the |outport| is ready for communication, |expression|
is evaluated and the result sent.

Sometimes it's necessary to do something \emph{after} the value has been sent,
using a continuation.
%
\begin{scala}
bool && outport =!=> { expression } ==> { command }
\end{scala}
\end{slide}

%%%%%

\begin{slide}
\heading{Example: \protect\SCALA{tee} again}

\begin{scala}
def tee[T](in: ?[T], out1: ![T], out2: ![T]) = thread{
  repeat{ 
    val v = in?()
    alt( out1 =!=> { v } ==> { out2!v }
       | out2 =!=> { v } ==> { out1!v }
    )
  }
}
\end{scala}
%
Note how this can output to \SCALA{out1} and \SCALA{out2} in either order.
\end{slide}

%%%%%%

\begin{slide}
\heading{A two-place buffer}

Inports and outports can be mixed within an \SCALA{alt}.

The following process copies data from |in| to |out|, and can hold up to two
pieces of data at a time: it is a two-place buffer.
%
\begin{scala}  
  /** Two place buffer. */
  def buff2[T](in: ?[T], out: ![T]): Unit = {
    val x = in?(); buff2full(in, out, x)
  }
  /** Two place buffer holding x. */
  def buff2full[T](in: ?[T], out: ![T], x: T): Unit = {
    alt(
      out =!=> { x } ==> { buff2(in, out) }
      | in =?=> { y => out!x; buff2full(in, out, y) }
    )
  }  
\end{scala}

%% /** Empty two-place buffer. */
%% def Buff2[T](in: ?[T], out: ![T]): ThreadGroup = thread{
%%   val x = in?(); run(Buff2a(in, out, x))
%% }
%% /** Two-place buffer holding x. */
%% def Buff2a[T](in: ?[T], out: ![T], x: T): ThreadGroup = thread{
%%   alt( out =!=> { x } ==> { run(Buff2(in, out)) }
%%      | in =?=> { y => out!x; run(Buff2a(in, out, y)) }
%%   )
%% }  
\end{slide}

%%%%%

\begin{slide}
\heading{A two place buffer}

Here's an alternative definition.  \SCALA{empty} records whether the buffer is
empty.  When $\sm{empty} = \sm{false}$,\, \SCALA{x} stores the next value to be
output. 
%
\begin{scala}
  def buff2Alt[T](in: ?[T], out: ![T]) = {
    var x: T = null.asInstanceOf[T]  // contents, possibly invalid
    var empty = true // is the buffer empty?
    serve(
      !empty && out =!=> { empty = true; x }
      | empty && in =?=> { v => x = v; empty = false }
      | !empty && in =?=> { v => out!x; x = v }
    )
  }
\end{scala}
%% \begin{scala}
%% def Buff2Alt[T](in: ?[T], out: ![T]) = thread{
%%   var x = in?(); var empty = false
%%   serve(
%%     (!empty && out) =!=> { empty = true; x }
%%     | (empty && in) =?=> { v => x = v; empty = false }
%%     | (!empty && in) =?=> { v => out!x; x = v }
%%   )
%% }
%
The last two branches could be merged, and an \SCALA{if} statement used. 
\begin{scala}
    | in =?=> { v => if(empty){ x = v; empty = false } else {out!x; x = v } }
\end{scala}
\end{slide}

%%%%%


\begin{slide}
\heading{Restrictions}

\begin{itemize}
%% \item
%% An |alt| may not have two simultaneously enabled branches using the
%% same port.

%% \item An |alt| may not use a port that is shared, either with another |alt| or
%% with a non-alt read or write.

\item A port may not be simultaneously feasible in two |alt|s or |serve|s.
  (But it may be feasible in an |alt|/|serve|, and also used by a thread with
  a simple send or receive.)

\item Both ports of a channel may not simultaneously be feasible in |alt|s or
  |serve|s.
\end{itemize}
%
\end{slide}

%%%%%

\begin{slide}
\heading{Generalised \protect\SCALA{alt}s}

\SCALA{alt}s (and \SCALA{serve}s) can be constructed from collections of
branches.

Here is a generalization of the tagger:
%
\begin{scala}
def tagger[T](ins: List[?[T]], out: ![(Int, T)]) = thread{
  serve ( 
    | (for (i <- 0 until ins.length) yield ins(i) =?=> { x => out!(i, x) })
  )
  for (in <- ins) in.close
  out.endOfStream
}
\end{scala}

This will terminate when all input channels are closed, or \SCALA{out} is
closed. 
\end{slide}

%%%%%

\begin{slide}
\heading{Caveat}

|alt|s and |serve|s are fairly heavyweight constructs.  Don't use them unless
there's a good reason to do so, i.e.~you really want to be able to communicate
on either of two channels.
Sometimes it's easier and more efficient to force communications to happen in
a particular order.
\end{slide}

%%%%%

\begin{slide}
\heading{Summary}

\begin{itemize}
\item
\SCALA{alt}s;

\item
Syntax: using inports and outports; guards; \SCALA{serve}; generalised |alt|s;
% timeouts; \SCALA{orelse}; \SCALA{prialt} (and \SCALA{priserve})

\item
Example: dining philosophers;

\item
Logging;

%% \item
%% \SCALA{alt}s versus \SCALA{ManyOne} channels;

\item
Restrictions on use of \SCALA{alt}s. 
\end{itemize}
\end{slide}

\end{document}
