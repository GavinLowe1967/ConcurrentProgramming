
\begin{slide}
\heading{Clients and servers}

A common pattern in concurrent systems is that of clients and servers.  A
\emph{server} is a thread or process that repeatedly handles requests from
\emph{client}s.
%
\begin{itemize}
\item Many modules, such as concurrent datatypes, can be easily implemented in
this way.

\item This is a pattern that is often used in the implementation of operating
systems.

\item It's the pattern that is prevalent in networked systems.
\end{itemize}
\end{slide}

%%%%%

\begin{slide}
\heading{Example: a resource server}

Consider a server that is responsible for managing and allocating multiple
resources of the same kind, such as memory or file blocks.  Clients acquire
resources for use, and later return them to the server.

What should happen if a client requests a resource and there is none
available?  We choose to use an |Option[Resource]| value:
%
\begin{itemize}
\item A value |Some(r)| indicates that resource |r| has been acquired;

\item The value |None| indicates that no resource was available.
\end{itemize}
%
In the latter case, it is up to the client to decide what to do (maybe try
again later, or maybe throw an exception).

An alternative would be for the server to queue the request until it can be
serviced. 
\end{slide}

%%%%%

\begin{slide}
\heading{Example: a resource server}

We will define a server with the following interface. 
%
\begin{scala}
/** A resource server. */
trait RAServer{
  /** Client identities. */
  type ClientId = Int

  /** Resource identities. */
  type Resource = Int

  /** Request a resource. */
  def requestResource(me: ClientId): Option[Resource]

  /** Return a resource. */
  def returnResource(me: ClientId, r: Resource) 
} 
\end{scala}
\end{slide}

%%%%%

\begin{slide}
\heading{First implementation}

In the first implementation, we assume clients have identities in the range
|[0..clients)|, and resources have identities in the range
  |[0..numResources)|, with |clients| and |numResources| known in advance.
%
\begin{scala}
/** A resource server. 
  * This version assumes the number of clients is known initially. 
  * @param clients the number of clients.
  * @param numResources the number of resources.  */
class RAServer1(clients: Int, numResources: Int) extends RAServer{
  ...
}
\end{scala}
\end{slide}

%%%%%

\begin{slide}
\heading{Channels}

Internally, clients will send requests to a server thread, and receive replies
back. 
%
We use one channel per client to allow the server to allocate resources.  We
use two buffered channels to allow clients to request or return resources.
%
\begin{scala}
class RAServer1(clients: Int, numResources: Int) extends RAServer{
  /* Channel for requesting a resource. */
  private val acquireRequestChan = new BuffChan[ClientId](clients)

  /* Channels for optionally allocating a resource, indexed by client IDs. */
  private val acquireReplyChan = 
    Array.fill(clients)(new BuffChan[Option[Resource]](1))

  /* Channel for returning a resource. */
  private val returnChan = new BuffChan[Resource](clients)
  ...
}
\end{scala}
\end{slide}

%%%%%

\begin{slide}
\heading{Client operations}

\begin{scala}
  /** Request a resource. */
  def requestResource(me: ClientId): Option[Resource] = {
    acquireRequestChan!me  // send request
    acquireReplyChan(me)?() // wait for response
  }

  /** Return a resource. */
  def returnResource(me: ClientId, r: Resource) = returnChan!r
\end{scala}
\end{slide}


%%%%%

\begin{slide}
\heading{The server}

The server needs to keep track of the resources that are currently free; we
use an array \SCALA{free} for this:
%
\begin{scala}
  private def server = thread{
    // Record whether resource i is available in free(i)
    val free = Array.fill(numResources)(true)

    serve( 
      ...
    )
  }

  // Fork off the server
  server.fork
\end{scala}
%
Note that only the server has access to |free|, which avoids race
conditions.  Declaring variables inside a server can clarify their scope. 
\end{slide}

%%%%%

\begin{slide}
\heading{The server main loop}

\begin{scala}
    serve(
      acquireRequestChan =?=> { c => 
	// Find free resource
	var r = 0
	while(r < numResources && !free(r)) r += 1
	if(r == numResources) acquireReplyChan(c)!None
        else{  // Pass resource r back to client c
	  free(r) = false; acquireReplyChan(c)!Some(r)
        }
      }
      | returnChan =?=> { r => free(r) = true }
    )
\end{scala}
\end{slide}




%%%%%

\begin{slide}
\heading{Shutting down the server}

It might be necessary to shut down the server, allowing the server thread to
terminate. 
%
\begin{scala}
  /* Channel for shutting down the server. */
  private val shutdownChan = new SyncChan[Unit]

  /** Shut down the server. */
  def shutdown = shutdownChan!()
  
  private def server = thread{
    ...
    serve(
      acquireRequestChan =?=> { c => ... }
      | returnChan =?=> { r => ... }
      | shutdownChan =?=> { _ =>
          acquireRequestChan.close; returnChan.close; shutdownChan.close }
    )
  }
\end{scala}  
\end{slide}
