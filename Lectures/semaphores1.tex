
\begin{slide}
\heading{Semaphores}

A semaphore is an important synchronisation tool.  The idea (and name) is
inspired by railway signalling, where a flag is raised to indicate that track
ahead can be entered, or lowered to indicate that the track ahead is occupied.
A train arriving at the section of track waits until the flag is up, enters
the section, and the flag is lowered again.  When the train leaves the section
of track, the flag is raised. 

A semaphore 
%% has a boolean variable, that indicates whether the flag is down.  It
has two operations:
%
\begin{description}
\item[{down}]
Wait until the flag is in the up position, and then put the flag down and
procced;

\item[{up}]
Put the flag up, to allow another process to proceed.
\end{description}

\SCALA{down} and \SCALA{up} are sometimes called \SCALA{P} and \SCALA{V},
{\codecolour\sf wait} and \SCALA{signal}, or \SCALA{acquire} and
\SCALA{release}. 
\end{slide}

%%%%%

\begin{slide}
\heading{Semaphores in Scala}

A semaphore can be implemented as a monitor, as below.
%
\begin{scala}
/** A binary semaphore.
  * @param isUp is the semaphore initially in the "up" state? */
class Semaphore(private var isUp: Boolean){
  def down = synchronized{
    while(!isUp) wait()
    isUp = false
  }

  def up = synchronized{
    require(!isUp); isUp = true; notify()
  }
}
\end{scala}
% Note that |up| has no effect if the semaphore is already up. 
\end{slide}

%%%%%

\begin{slide}
\heading{Implementing semaphores}

Note that the |up| operation requires that the semaphore is not already up.
My experience is that trying to put the semaphore up when it is already up is
nearly always an error: a signal gets lost.

However, other implementations of semaphores allow such an |up|, but are such
that the operation has no effect. 

In fact, many operating system kernels provide implementations of semaphores.  

Semaphores can then be used to implement monitors within a programming
language. 

%% \bigskip

%% \heading{Semaphores in SCO}

%% The factory method |BooleanSemaphore(available)| creates a new semaphore.
%% (The method has various optional arguments, not included on the previous
%% slide.) 


\end{slide}

%%%%%

\begin{slide}
\heading{Using a semaphore for mutual exclusion}

One use of semaphores is to provide mutual exclusion.  

Such a semaphore is created with
\begin{scala}
  val mutex = new Semaphore(true)
\end{scala}
or
\begin{scala}
  val mutex = new MutexSemaphore
\end{scala}
%
using the definition
%
\begin{scala}
class MutexSemaphore extends Semaphore(true)
\end{scala}
\end{slide}

%%%%%

\begin{slide}
\heading{Using a semaphore for mutual exclusion}

A thread entering the critical region executes
\begin{scala}
  mutex.down
\end{scala}
%
This blocks if the semaphore is already down, i.e.~if another thread is in the
critical region.  In the initial state, the operation will not block. 

A thread leaving the critical region executes
\begin{scala}
  mutex.up
\end{scala}
%
This unblocks a waiting thread (if there is one).
\end{slide}

%%%%%

\begin{slide}
\heading{Using a semaphore for mutual exclusion}

This version of the counter protects \SCALA{x} using a semaphore.
%
\begin{scala}
object Counter{
  private var x = 0
  
  private var mutex = new MutexSemaphore
    
  def inc = { mutex.down; x = x+1; mutex.up }

  def dec = { mutex.down; x = x-1; mutex.up }

  def getX = { mutex.down; val result = x; mutex.up; result }
}
\end{scala}
%
% Note that this method can be used to provide mutual exclusion on arbitrary
% objects (like a monitor).
\end{slide}

%%%%%

\begin{slide}
\heading{Misusing semaphores}

What is wrong with the following code, where semaphore \SCALA{sem1} is used to
protect resource \SCALA{res1}, and semaphore \SCALA{sem2} is used to protect
resource \SCALA{res2}?
%
\begin{scala}
val sem1 = MutexSemaphore(); val sem2 = MutexSemaphore()

def p = proc{
  sem1.down; ...; sem2.down; ... ; sem2.up; ...; sem1.up
}

def q = proc{
  sem2.down; ...; sem1.down; ... ; sem1.up; ...; sem2.up
}

run(p || q)
\end{scala}
\end{slide}
