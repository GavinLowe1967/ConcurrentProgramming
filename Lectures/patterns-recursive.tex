
\begin{slide}
\heading{Recursive parallelism}

Many sequential programs use a recursive procedure: a procedure that calls
itself.

The idea of recursive parallelism is that the recursive calls are replaced by
spawning off parallel processes with the same effect.  This is particularly
useful where the procedure would make two or more recursive calls to itself
that are independent (operate on disjoint data): the corresponding recursive
parallel processes can then run concurrently.

A typical pattern for recursive parallelism is:
%
\begin{itemize}
\item
In some base case(s), calculate the result directly;

\item
In other cases, spawn off a parallel process corresponding to each recursive
call, together with a controller to tie things together.
\end{itemize}

\end{slide}


%% %%%%% 

\begin{slide}
\heading{Example: Quicksort}

We saw an implementation of Quicksort using recursive parallelism and message
passing in Chapter~2.

We will now implement Quicksort using recursive parallelism and shared
variables, to sort an array~|a|.  Different threads will work on disjoint
segments of~|a|, to avoid race conditions.

We can define a sequential partition function in the normal way.
%
\begin{scala}
  /** Partition the segment a[l..r).
    * Permute a[l..r) and return k s.t. a[l..k) <= a(k) < a[k+1..r).  */
  private def partition(l: Int, r: Int): Int = ... // standard definition
\end{scala}
\end{slide}

%%%%%

\begin{slide}
\heading{Example: Quicksort}

The recursive parallel definition creates one process to sort each subsegment.
%
\begin{scala}
  /** Sort a[l..r) using recursive parallelism. */
  private def qsort(l: Int, r: Int): ThreadGroup = thread{
    if(l+1 < r){
      val m = partition(l, r)
      run(qsort(l, m) || qsort(m+1, r))
    }
  }

  /** Sort a using recursive parallelism. */
  def apply() = run(qsort(0, a.length))
\end{scala}
\end{slide}

%%%%%


\begin{slide}
\heading{Limits of recursive parallelism}

In many applications, full recursive parallelism will lead to many more
processes than there are processors.  The overheads involved in spawning off
new processes, and context-switching between processes will mean that this is
very inefficient.

A better way is to limit the number of processes.  The simplest way to do this
is to switch to the sequential algorithm for small sub-problems.
\end{slide}

%%%%%

\begin{slide}
\heading{Recursive parallel quicksort with a limit}

\begin{scala}
  /** Minimum size of segment to sort with recursive parallelism. */
  val parLimit = a.length / 20

  /** Sort a[l..r) using recursive parallelism for segments of size at least
    * parLimit. */
  private def qsortLimit(l: Int, r: Int): ThreadGroup = thread{
    if(l+1 < r){
      val m = partition(l, r)
      if(r-l >= parLimit) run(qsortLimit(l, m) || qsortLimit(m+1, r))
      else{ run(qsortLimit(l, m)); run(qsortLimit(m+1, r)) }
    }
  }
\end{scala}
\end{slide}

%%%%%

\begin{slide}
\heading{Quicksort using the bag-of-tasks pattern}

An alternative concurrent implementation of quicksort would use a fixed number
of threads and the bag-of-tasks pattern, with replacement.  

The bag of tasks could be implemented by a concurrent stack or queue
(supporting termination).  Each task is a pair |(l,r)| representing that the
segment |a[l..r)| has to be sorted.
\end{slide}

%%%%%

\begin{slide}
\heading{Quicksort using the bag-of-tasks pattern}


Each thread repeatedly:
%
\begin{itemize}
\item Removes a task from the bag;

\item If the task is sufficiently small, sorts the segment sequentially;

\item Otherwise, it partitions the segment, and returns two tasks to the bag,
  representing that the two sub-intervals need to be sorted.
\end{itemize}

The bag is likely to prove a bottleneck.  Possible enhancements are:
%
\begin{itemize}
\item Don't return empty or singleton tasks to the bag;

\item Rather than returning two bags to the task, just return one, and
  continue working on the other;

\item Using a more sophisticated concurrent data structure.
\end{itemize}
\end{slide}


%%%%%

\begin{slide}
\heading{Parallel collections}

Scala's parallel collections
API\footnote{%
\url{https://docs.scala-lang.org/overviews/parallel-collections/overview.html}.}
aims to support parallelism in a way that means the programmer doesn't have to
think too much about parallelism.  Example:
%
\begin{scala}
scala> (1 to 10000).toList.par.sum
res0: Int = 50005000
\end{scala}

Internally, recursive parallelism is used.  To sum a long list, it can be
split into two parts; each is summed, in parallel (possibly leading to further
splits); then the subresults are summed. 

Of course, careless use can create race conditions.
\begin{scala}
  var sum = 0
  (1 to 10000).toList.par.foreach(sum += _)
\end{scala}

\vfill
\end{slide}
