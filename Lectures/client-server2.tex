
\begin{slide}
\heading{Testing}

How can we test the resource server?  

The property we want to test is that two clients never hold the same resource:
if a resource~$r$ is allocated to client~$c_1$, it can't be allocated to
another client~$c_2$ until $c_1$ has returned it.

Idea: run a number of clients, who randomly request and return resources; log
these actions; test whether the resulting trace of log events is valid.
\end{slide}

%%%%%

\begin{slide}
\heading{Logging}

We use a log of type |Log[LogEvent]| where
\begin{scala}
  // Events put into the log
  trait LogEvent
  case class GotResource(c: ClientId, r: Resource) extends LogEvent
  case class ReturnedResource(c: ClientId, r: Resource) extends LogEvent
\end{scala}

Subsequently, we can use the |get| method on the log, to get the
results of logging as an |Array[LogEvent]|.  
\end{slide}

%%%%%

\begin{slide}
\heading{A client for testing}

\begin{scala}
  def client(me: ClientId, resourceServer: RAServer, log: Log[LogEvent]) = thread{
    var got = new scala.collection.mutable.Queue[Resource]() // resources held
    for(_ <- 0 until iters){
      if(Random.nextInt(2) == 0){        // Acquire new resource
	resourceServer.requestResource(me) match{
          case Some(r) =>  log.add(me, GotResource(me, r)); got.enqueue(r)
          case None => {}  // try again
        }
      }
      else if(got.nonEmpty){    	// Return resource
	val r = got.dequeue()
        log.add(me, ReturnedResource(me, r))
	resourceServer.returnResource(me, r)
      }
    }
  }
\end{scala}
\end{slide}  

%%%%%

\begin{slide}
\heading{Logging}

Note that we log acquiring resources \emph{after} the resource is acquired,
and log returning resources \emph{before} the resource is returned.  This
means that the period during which the log indicates that the thread holds a
resource is a \emph{subset} of the true time.

This is necessary to avoid false positives (the testing signalling an error,
when in fact there is none).  If we were to return a resource before logging
the return, it is possible that a client would be slow in logging the return,
and that another client might have obtained the same resource in the mean
time, giving a false positive.

However, this does risk false negatives (the testing failing to detect an
incorrect behaviour).  But if there is a bug, doing enough testing should also
produce true positives. 
\end{slide}

%%%%%

\begin{slide}
\heading{Running a test}

\begin{scala}
  /** Run a single test. */
  def runTest = {
    val log = new Log[LogEvent](p)
    val resourceServer = new RAServer(p, numResources)
    run(||(for (i <- 0 until p) yield client(i, resourceServer, log)))
    resourceServer.shutdown        // allow server to terminate
    if(!checkLog(log.get)) sys.exit()
  }
\end{scala}

The |main| function calls |runTest| many times. 
\end{slide}

%%%%%

\begin{slide}
\begin{scala}
  /** Check that events represents a valid log: if a GotResource event happens,
    * no thread is currently holding the resource. */
  def checkLog(events: Array[LogEvent]): Boolean = {
    val held = Array.fill(numResources)(false)  // Which resources are held
    var error = false; var i = 0
    while(i < events.size && !error){
      events(i) match{
        case GotResource(c, r) =>
          if(held(r)){ // error!
            println("Error found:"); println(events.take(i+1).mkString("\n"))
            error = true
          }
          else held(r) = true
        case ReturnedResource(c, r) => held(r) = false
      }
      i += 1
    }
    !error
  }
\end{scala}
\end{slide}

%%%%%%%%%%%%%%%%%%%%%%%%%%%%%%%%%%%%%%%%%%%%%%%%%%%%%%% 

\begin{slide}
\heading{Arbitrary many clients}

So far, we have assumed a fixed number of clients, with this number known when
the resource server is created.  This allowed us to create one reply channel
per client.

If we do not know the number of clients, an alternative is for each client to
create a new reply channel for each request, and to send the reply channel
within the request.  The server can then reply on this reply channel.  

We illustrate this technique in the following slides (omitting the parts that
are unchanged). 
\end{slide}

%%%%%

\begin{slide}
\heading{Requesting resources}

\begin{scala}
  private type ReplyChan = Chan[Option[Resource]]
  /* Channel for requesting a resource. */
  private val acquireRequestChan = new BuffChan[ReplyChan](clients)

  /** Request a resource. */
  def requestResource(me: ClientId): Option[Resource] = {
    val replyChan = new BuffChan[Option[Resource]](1)
    acquireRequestChan!replyChan  // send request
    replyChan?() // wait for response
  }
\end{scala}
\end{slide}

%%%%%

\begin{slide}
\heading{The server}

\begin{scala}
  private def server = thread{
    ...
    serve(
      acquireRequestChan =?=> { replyChan => 
	// Find free resource
	var r = 0
	while(r < numResources && !free(r)) r += 1
	if(r == numResources) replyChan!None
        else{  // Pass resource r back to client 
	  free(r) = false; replyChan!Some(r)
        }
      }
      | ...
    )
  }
\end{scala}
\end{slide}

%%%%%%%%%%%%%%%%%%%%%%%%%%%%%%%%%%%%%%%%%%%%%%%%%%%%%%%

\begin{slide}
\heading{Storing requests}

Previously, if the server couldn't meet a request, it immediately replied with
a |None| value.  

An alternative would be to queue such requests until they can be met.

Note that this can lead to a deadlock if all resources are allocated, and all
clients are requesting more.
\end{slide}

%%%%%

\begin{slide}
\heading{The server}

The server keeps track of the pending requests in a queue.
%
\begin{scala}
  private def server = thread{
    // Record whether resource i is available in free(i)
    val free = Array.fill(numResources)(true)
    // Reply channels for requests that cannot be served immediately.
    val pending = new scala.collection.mutable.Queue[ReplyChan]
    // Invariant: if pending is non-empty, then all entries in free are false.

    serve(
      ...
    )
  }
\end{scala}
\end{slide}

%%%%%

\begin{slide}
\heading{The server}

\begin{scala}
    serve(
      acquireRequestChan =?=> { replyChan => 
	// Find free resource
	var r = 0
	while(r < numResources && !free(r)) r += 1
	if(r == numResources) pending.enqueue(replyChan) // client has to wait
        else{  // Pass resource r back to client 
	  assert(pending.isEmpty); free(r) = false; replyChan!Some(r)
        }
      }
      | returnChan =?=> { r =>
          if(pending.nonEmpty)
            pending.dequeue()!Some(r) // allocate r to blocked client
          else free(r) = true
      }
      | shutdownChan =?=> { _ => ... }
    )
\end{scala}
\end{slide}

%%%%%

\begin{slide}
\heading{Blocking unsatisfiable requests}

Here's a simpler way to achieve the same effect.  Rather than the server
explicitly queueing pending requests, the clients can be blocked on the
|acquireRequestChan| channel until there is a resource free.

\end{slide}

%%%%%

\begin{slide}
\heading{Blocking unsatisfiable requests}

\begin{scala}
  private def server = thread{
    // Record whether resource i is available in free(i)
    val free = Array.fill(numResources)(true)
    // The number of free resources, equal to the number of trues in free.
    var numFree = numResources

    serve(
      numFree > 0 && acquireRequestChan =?=> { replyChan => 
	var r = 0; while(r < numResources && !free(r)) r += 1
	assert(r < numResources); free(r) = false; numFree -= 1
        replyChan!Some(r)
      }
      | returnChan =?=> { r => free(r) = true; numFree += 1 }
      | shutdownChan =?=> { ... } // as before
    )
  }
\end{scala}
\end{slide}

%%%%%

\begin{slide}
\heading{Blocking unsatisfiable requests}

Note that when the server has no free resources, all client threads wanting to
obtain a resource will be blocked on |acquireRequestChan|.

When a resource is returned, those clients will compete to communicate on the
channel.  The one that wins will obtain the resource.

This means that this version is probably less fair than the previous one.
Whether this is important depends on the use case.
\end{slide}

%%%%%%%%%%%%%%%%%%%%%%%%%%%%%%%%%%%%%%%%%%%%%%%%%%%%%%%


\begin{slide}
\heading{Introducing asynchrony}

At present, the client sends a request, and then stops and waits for a
response: this is rather like a procedure call.

In some cases, the client could do some useful work while waiting for the
request to be serviced:
%
\begin{scala}
<make request>
<other useful work>
<obtain response>
\end{scala}
%

We could change the interface of the object to support this, decoupling the
request for a resource from obtaining the resource.

The fact that we have used buffered reply channels means that the server can
send the response even if the client isn't yet ready for it.  


%% (This would change the interface of the corresponding object.)

%% What happens if the server is ready to return the response before the client
%% is ready to receive it?
\end{slide}

%% %%%%%

%% \begin{slide}
%% \heading{Using buffering}

%% CSO includes buffered communication channels.  

%% A sender can add values to a buffered channel, up to the capacity of the
%% channel, without waiting for a receiver to be ready.

%% Buffered channels should be used when there is no need for the
%% processes to synchronise.

%% |OneOneBuf[T](size)| returns a (one-one) buffered channel of type \SCALA{T}
%% that can hold up to \SCALA{size} messages.

%% |N2NBuf[T](size, writers, readers)| returns a buffered channel of type \SCALA{T}
%% that can hold up to \SCALA{size} messages, to be used by |writers| writers and
%% |readers| readers.

%% \end{slide}

%% %%%%%

%% \begin{slide}
%% \heading{Using buffering}

%% It makes sense to make the channels from the clients to the server buffered,
%% to reduce the likelihood of the clients having to wait for the
%% server; we give them capacity \SCALA{clients}.
%% \begin{scala}
%%   private val acquireRequestChan = 
%%     N2NBuf[ReplyChan](size = clients, readers = 1, writers = clients)
%%   private val returnChan = 
%%     N2NBuf[Resource](size = clients, readers = 1, writers = clients)
%% \end{scala}

%% It also makes sense to make the response channels buffered, so the server does
%% not have to wait for a client; they only need have capacity~1.
%% %
%% \begin{scala}
%%   val replyChan = OneOneBuf[Option[Resource]](size = 1)
%% \end{scala}
%% \end{slide}

%%%%%

\begin{slide}
\heading{Multiple request channels versus multiple request types}

We chose to use different channels for different requests to the server.  An
alternative is to use a single channel, and multiple types of request.

Such an approach seems more natural in a networked application.
\end{slide}

%%%%%


\begin{slide}
\heading{Multiple request channels versus multiple request types}

In the resource allocation we could define the following.
%
\begin{scala}
private trait Request
private case class Acquire(replyChan: ReplyChan) extends Request
private case class Return(r: Resource) extends Request
private case object Shutdown extends Request

private val requestChan = new BuffChan[Request](clients)
\end{scala}

The server receives on |requestChan|, and pattern matches on the value
received.
\begin{scala}
  requestChan?() match{
    case Acquire(replyChan) => ...
    case Return(r) => ...
    case Shutdown => ...
  }
\end{scala}
\end{slide}

