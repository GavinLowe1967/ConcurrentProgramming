\begin{slide}
\heading{A synchronous channel}

We will implement a synchronous channel, for use by an arbitrary number of
senders and receivers.  

We will require three signals between threads:
%
\begin{itemize}
\item Each sender will signal to a receiver that it has deposited a value,
  which the receiver can take;

\item The receiver will then signal back to the sender that it can continue;
  this makes the channel synchronous;

\item The receiver will signal to a sender for the next round that it has
  cleared the value, so that sender can deposit its value.
\end{itemize}
%
We will use a separate condition for each signal. 
\end{slide}

%%%%%

\begin{slide}
\heading{A synchronous channel}

\begin{scala}
class SharedSyncChan[A]{
  /** The current or previous value. */
  private var value = null.asInstanceOf[A]

  /** Is the current value of value valid, i.e. ready to be received? */
  private var full = false
  ...
}
\end{scala}
\end{slide}

%%%%%

\begin{slide}
\heading{A synchronous channel}

\begin{scala}
  /** Lock for controlling synchronisations. */
  private val lock = new Lock

  /** Condition for signalling to receiver that a value has been 
    * deposited. */
  private val slotFull = lock.newCondition

  /** Condition for signalling to current sender that it can continue. */
  private val continue = lock.newCondition

  /** Condition for signalling to the next sender that the previous value 
    * has been read. */
  private val slotEmptied = lock.newCondition
\end{scala}
\end{slide}

%%%%%

\begin{slide}
\heading{A synchronous channel}

\begin{scala}
  def send(x: A) = lock.mutex{
    slotEmptied.await(!full)  // wait for previous value to be consumed (1)
    value = x; full = true   // deposit my value
    slotFull.signal()         // signal to receiver at (2)
    continue.await()          // wait for receiver (3)
  }

  def receive: A = lock.mutex{
    slotFull.await(full)     // wait for sender (2)
    continue.signal()        // signal to current sender at (3) 
    full = false             // clear value
    slotEmptied.signal()     // signal to next sender at (1)
    value
  }
\end{scala}
\end{slide}

%%%%%

\begin{slide}
\heading{A synchronous channel}

Note that is necessary for the sender to recheck |full| when it receives a
signal on |slotEmptied|, in case another sender has run in the meantime and
filled the slot.

Likewise with the receiver on |slotFull|.

However, there is no need for the sender to perform a similar check on
receiving a signal on |continue|, because it cannot be preempted by another
sender. 
\end{slide}

%%%%%

\begin{slide}
\heading{Testing}

We can test this implementation by running some threads that perform sends and
receives, and have them logging the call and return of each operation.  For
simplicity, we arrange that every send is of a different value.  This makes it
easy to identify matching sends and receives (if they exist).  We then check
that the two invocations overlap.

It is sound to assume the values sent are all different, because the
implementation is \emph{data independent} in the type |A| of data: values of
this type are stored, read, and returned, but no operation (e.g.~|==|) is
done on them.  This means that for any erroneous history, there would be a
corresponding erroneous history using distinct values.
\end{slide}
