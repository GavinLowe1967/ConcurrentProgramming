
\begin{slide}
\heading{Bag of tasks with replacements}

We have already seen the \emph{bag of tasks} pattern:
%
\begin{itemize}
\item
A controller process holds a bag of tasks that need completing;

\item
Worker processes repeatedly obtain a task from the controller, and complete
it.
\end{itemize}

We saw this in the numerical integration problem, where a task represented an
interval whose integral needed estimating; no tasks were ever returned to the
bag, so this made the controller very simple.

We will now look at a more interesting example, where tasks are returned to
the bag.
\end{slide}

%%%%%

\begin{slide}
\heading{Magic squares}

A \emph{magic square} of size~$n$ is an $n$ by $n$ square, containing the
integers from $1$ to $n^2$, such that the sums of all the rows, columns and
diagonals are the same.  For example:
\[
\begin{array}{\|cccc\|}
\hline
16 & 4 & 13 & 1  \\
2 & 5 & 12 & 15  \\
9 & 14 & 3 & 8  \\
7 & 11 & 6 & 10 \\\hline
\end{array}
\]
We will design a concurrent program, using the bag of tasks pattern, to find
magic squares.

Similar techniques can be used for many other search problems.
\end{slide}

%%%%%

\begin{slide}
\heading{Partial solutions}

Each task will correspond to a partial solution of the puzzle, i.e. where
some, but not necessarily all, of the locations have been filled in.  

A worker process will obtain a partial solution, and, if it is not complete:
%
\begin{itemize}
\item
choose an empty square;

\item
create all the partial solutions obtained by filling that square with a value
that might lead to a complete solution (there might be none);

\item
return those new partial solutions to the bag.
\end{itemize}
\end{slide}

%%%%%

\begin{selfnote}

For example, given the partial solution
\[
\begin{array}{|c|c|c|c|}
\hline
16 & 4 & 11 & 3\;\:  \\\hline
2 & 12 &  &   \\\hline
 &  &  &   \\\hline
&  &  &  \\ \hline
\end{array}
\]
The next position has to be at least 5 to make it possible for that row to add
up to 34; so could be 5, 6, 7, 8, 9, 10, 13, 14, 15.

If 5 is selected, the next value has to be 15.

Alternatively, if 8 were selected, no value is posible.
\end{selfnote}

%%%%%

\begin{slide}
\heading{Representing partial solutions}

We will represent partial solutions using objects of the following class.
%
\begin{scala}
class PartialSoln(n : Int){
  // Create an empty partial solution
  def empty = {...} 

  // Is the partial solution completed?
  def finished : Boolean = {...} 

  // Is it legal to play piece k in position (i,j)
  def isLegal(i: Int, j: Int, k: Int) : Boolean = {...}

  ...
}
\end{scala}
\end{slide}

%%%%%

\begin{slide}
\heading{Representing partial solutions}

\begin{scala}
class PartialSoln(n : Int){ ...
  // Return new partial solution obtained by placing 
  // k in position (i,j)
  def doMove(i: Int, j: Int, k: Int) : PartialSoln = {...}

  // Choose a position in which to play
  def choose : (Int,Int) = {...}

  // Print board
  def printBoard = {...}
}
\end{scala}

We won't discuss the implementation of \SCALA{PartialSoln} (although this
makes a big difference to the efficiency of the program).  We will concentrate
on the concurrent programming aspects of the problem. 
\end{slide}

%%%%%

\begin{slide}
\heading{Channels}

The workers will use the following channels:
\begin{scala}
val get = OneMany[PartialSoln]; 
// get jobs from the controller

val put = ManyOne[PartialSoln];
// pass jobs back to the controller

val done = ManyOne[Unit]; 
// tell the controller we've finished a job

val toPrinter = ManyOne[PartialSoln]; 
// channel to printer
\end{scala}
\end{slide}

%%%%%

\begin{selfnote}
Use a separate process for printing.  Why?
\end{selfnote}

%%%%%

\begin{slide}
\heading{A worker}

\begin{scala}
def Worker(n: Int, get: ?[PartialSoln], put: ![PartialSoln], 
           done: ![Unit], toPrinter: ![PartialSoln]) 
= proc("Worker"){
  repeat{
    val partial = get? ; // get job
    if(partial.finished) toPrinter!partial;  // done!
    else{
      val(i,j) = partial.choose;
      // Generate all next-states
      for(k <- 1 to n*n)
        if(partial.isLegal(i,j,k)) 
          put!partial.doMove(i,j,k); 
    }
    done!();
  }
}
\end{scala}
\end{slide}

%%%%%

\begin{slide}
\heading{The Controller}

The Controller will use a stack to store all the current partial solutions.

\emph{Question:} what difference would using a queue make?

The controller needs to know when it can terminate.  This will be when it is
holding no partial solutions, and none of the workers is busy; it therefore
needs to keep track of the number of busy workers (that's the reason for the
\SCALA{done} channel).
\end{slide}

%%%%%

\begin{slide}
\heading{The Controller}

\begin{scala}
def Controller(n: Int, get: ![PartialSoln], put: ?[PartialSoln], 
               done: ?[Unit], toPrinter: ![PartialSoln])
= proc("Controller"){
  // Initialise stack holding empty board
  val init = new PartialSoln(n);
  init.empty ;
  val stack = new scala.collection.mutable.Stack[PartialSoln];
  stack.push(init);

  var busyWorkers = 0; // # workers currently busy

  ...
}
\end{scala}
\end{slide}

%%%%%

\begin{slide}
\heading{The Controller}

\begin{scala}
def Controller(...) = proc{
  ...
  // Main loop
  serve(
    (!stack.isEmpty &&& get) -!-> 
      { get!(stack.pop) ; busyWorkers += 1; }
    | (busyWorkers>0 &&& put) -?-> 
      { val ps = put? ; stack.push(ps); }
    | (busyWorkers>0 &&& done) -?-> 
      { done? ; busyWorkers -= 1 }
  )
       
  // Finished, when stack.isEmpty and busyWorkers==0
  toPrinter.close; get.close;
}
\end{scala}
\end{slide}
