\documentclass[notes,color]{sepslide0}
\usepackage{graphicx}
\usepackage[overheads]{mysepslides}
\usepackage{tech,graphicx,url,tikz,mathsx,verbatim,scalalistings}

\title{Synchronous Data Parallel Programming} 
\author{Gavin Lowe}

%\everymath{\color{Plum}}
%\def\smaller{\small} 
\def\upto{\mathbin{..}}

\def\scalacolour{\color{violet}}

\begin{document}

\begin{slide}
  
  \Title

Reading: Andrews, Chapter 11.
\end{slide}

%%%%%


\begin{slide}
\heading{Introduction}

In this chapter we will study a particular style of data parallel
programming where the threads proceed \emph{synchronously}.  

The program proceeds in \emph{rounds}.  At the end of each round, there is a
global synchronisation, to ensure all threads have completed one round before
any proceeds to the next.
\end{slide}

%%%%%

\begin{slide}
\heading{Introduction}

Typically, each thread will operate on one section of the data, but  may
need to read data updated by other threads.  The synchronisation can be used
to ensure that one thread obtains the updates to the data made by other
threads on the previous round.

The data can be distributed between threads by two different techniques.
%
\begin{itemize}
\item
By sending messages; this works well when each piece of data has to be
passed to only a few other threads;

\item
By writing to shared variables.
%% ; in this case, there needs to be a global
%% synchronisation part way through each round, after threads have finished
%% reading the shared variables, and before they start writing to them.
\end{itemize}

%These algorithms are sometimes known as heart-beat algorithms.
\end{slide}

%%%%%

\begin{slide}
\heading{Applications}

\begin{itemize}
\item
Image processing;

\item
Solving differential equations, for example in weather forecasting or fluid
dynamics; 

\item
Matrix calculations;

\item
Cellular automata.
\end{itemize}
\end{slide}

%%%%%

\begin{slide}
\heading{Barrier synchronisation}

The global synchronisation at the end of each round is sometimes known as a
\emph{barrier synchronisation}, since it represents a barrier than no thread
may pass until all have reached that point.

Suppose we have |p| threads with identities $[\sm{0} \mathbin{..} \sm{p})$.
  Then a suitable barrier synchronisation object may be created by:
%
\begin{scala}
val barrier = new Barrier(p)
\end{scala}
%
%% (or |new Barrier(p)|)
%% where \SCALA{p} is the number of threads (with $\sm p > 1$).

A thread with identity~|me| performs the barrier synchronisation by executing
%
\begin{scala}
barrier.sync(me)
\end{scala}
%
No call to \SCALA{sync} will return until all \SCALA{p} threads have called
it. 
\end{slide}

%%%%%

\begin{slide}
\heading{A possible implementation of a barrier using a server}

\begin{scala}
class ServerBarrier(p: Int){
  private val arrive = ManyOne[Unit]
  private val leave = OneMany[Unit]

  def sync(me: Int) = { arrive!(); leave?() }

  private def server = thread{
    while(true){
      for(i <- 0 until p) arrive?()
      for(i <- 0 until p) leave!()
    }
  }

  server.fork
}
\end{scala}
\end{slide}

%%%%%


%%%%%

\begin{slide}
\heading{Testing the barrier implementation}

We can test the implementation using the technique of logging.  We arrange for
each of |p| workers to repeatedly: log that it is trying to synchronise; call
the |sync| operation; log that it has returned.

We then traverse the log, checking correctness.  At each point in the log, we
keep track of two (disjoint) sets of threads: (1)~those threads that are
trying to synchronise but cannot yet return, and (2)~those threads that may
return.  When the final thread tries to synchronise, we update to allow all
threads to return.  When a thread returns, we check that it is allowed to.

See the code on the course webpage.
\end{slide}

%%%%%

\begin{slide}
\heading{Barrier synchronisation}

With the actual implementation, each synchronisation takes time $O(\log
\sm{p})$.  (Creating the barrier object takes time $O(\sm{p})$.)

Exercise: implement a barrier with this running time.
\end{slide}

%%%%%

\begin{slide}
\heading{Example: prefix sums}

Suppose we have an array holding |n| integers:
\begin{scala}
val a = new Array[Int](n)
\end{scala}
%
We want to calculate the \emph{prefix sums} (i.e., the sums of the first $j$
elements, for $j = 1, \ldots, \sm n$) and store them in the array:
%
\begin{scala}
val sum = new Array[Int](n)
\end{scala}
%
So at the end we should have
\[
\forall k \in \set{0, \ldots, \sm n-1} \spot 
  \sm{sum}(k) = \textstyle\sum \sm{a}[0 \upto k] 
\]
We aim to do this using |n| threads, but in $\Theta(\log \sm n)$ rounds. 
\end{slide}

%%%%%

\begin{slide}
\heading{The idea of the algorithm}

Each thread~|me| sets $\sm{sum}(\sm{me})$ to $\sum \sm a[0 \upto \sm{me}]$.

The program proceeds in rounds, with a barrier synchronisation at the end of
each round.

Each thread will have a local variable~|s|, which is the sum it has
calculated so far.  At the end of round~$r$, thread~|me| will have 
\[
\sm s = \textstyle\sum \sm a(\sm{me}-\sm{gap} \upto \sm{me}]
   \mbox{ where } \sm{gap} = 2^r
\]
(and we invent fictitious elements $\sm a(i)=0$ for $i<0$).

On each round, thread~|me| obtains the value of thread
$(\sm{me}-\sm{gap})$'s |s|, which is
$\sum \sm a(\sm{me}-2.\sm{gap} \upto \sm{me}-\sm{gap}]$, and so calculates
$\sum \sm a(\sm{me}-2.\sm{gap} \upto \sm{me}]$, to maintain the above
invariant.

This continues until $\sm{gap} \ge \sm n$.
\end{slide}

%%%%%

\begin{slide}
\heading{Calculating prefix sums}

We will calculate the prefix sums using an object with the following
interface, and using |n| threads. 
%
\begin{scala}
class PrefixSums(n: Int, a: Array[Int]){
  require(n == a.size)

  /** Calculate the prefix sums. */
  def apply(): Array[Int] = ...
}
\end{scala}
\end{slide}

%%%%%

\begin{slide}
\heading{Synchronisation and communication}

The algorithm proceeds in rounds, so we need a barrier synchronisation object:
%
\begin{scala}
  /** Barrier synchronisation object. */
  private val barrier = new Barrier(n)
\end{scala}
%
We need channels to send values to threads:
\begin{scala}
  /** Channels on which values are sent, indexed by receiver's identity. */
  private val toSummers = Array.fill(n)(new BuffChan[Int](1))
\end{scala}
%
These are buffered to allow the send and receive to be asynchronous.  

Threads end up writing their prefix sums into the following array. 
\begin{scala}
  /** Shared array, in which sums are calculated. */
  private val sum = new Array[Int](n)
\end{scala}
\end{slide}

%%%%%

\begin{slide}
\heading{A thread}

\begin{scala}
  private def summer(me: Int) = thread("summer"+me){
    // Invariant: gap = £$2^r$£ and s = £$\sum$£ a(me-gap .. me]
    // (with fictious values a(i) = 0 for i < 0).  r is the round number.
    var r = 0; var gap = 1; var s = a(me)

    while(gap < n){
      if(me+gap < n) toSummers(me+gap)!s // pass my value up the line
      if(gap <= me){                    // receive from me-gap,
	val inc = toSummers(me)?()  // inc = £$\sum$£ a(me-2*gap .. me-gap]
	s = s + inc                     // s = £$\sum$£ a(me-2*gap .. me]
      }
      r += 1; gap += gap             // s = £$\sum$£ a(me-gap .. me]
      barrier.sync(me)
    }
    sum(me) = s
  }
\end{scala}
\end{slide}

%%%%%

\begin{slide}
\heading{Calculating the prefix sums}

\begin{scala}
  /** Calculate the prefix sums. */
  def apply(): Array[Int] = {
    run(|| (for (i <- 0 until n) yield summer(i)))
    sum
  }
\end{scala}
\end{slide}

%%%%%

\begin{slide}
\heading{The need for a barrier synchronisation}

The use of the barrier synchronisation ensures that different
threads hold the same value for the round number, and hence each thread can
depend upon the value it receives from its peer.

Suppose we didn't use a barrier synchronisation.  Consider a particular thread
|me| on round |r|, and let $\sm{gap} = 2^{\ss r}$.  It expects to receive from
thread~|me-gap|; but suppose that thread is slow.  And suppose
thread~|me-2.gap| is fast and has proceeded to round~|r+1|.  Then thread~|me|
will instead receive from thread~|me-2.gap|.  In particular, that means the
value it sends on the next round will be incorrect.
\end{slide}

%%%%%

\begin{slide}
\heading{Complexity}

The algorithm uses $\Theta(\log n)$ rounds.  Each barrier synchronisation
takes $\Theta(\log n)$ time.  This makes $\Theta((\log n)^2)$ in total
(ignoring the initialisation time).

\bigskip

\heading{Testing}

We can test this against a sequential implementation. 

% \emph{Question:} What goes wrong if we don't have the barrier
% synchronisation?

\bigskip

\heading{Exercise} 

Adapt the program so that the threads communicate via shared variables rather
than channels.
\end{slide}

%%%%%

\begin{selfnote}
Without the barrier synchronisation, a summer could receive values out of
order.  For example, \SCALA{Summer(3)} could first receive \SCALA{a(0)+a(1)}
from \SCALA{Summer(1)}, and then receive \SCALA{a(2)} from \SCALA{Summer(2)},
if Summer(1) happens to be faster than Summer(2).  Despite this
\SCALA{Summer(3)} ends up with the right answer; but it will pass
\SCALA{a(0)+a(1)+a(3)} to \SCALA{Summer(5)} in the second step (instead of
\SCALA{a(2)+a(3)}), so \SCALA{Summer(5)} ends up with the wrong value.
\end{selfnote}
 % intro; prefix sums

\begin{slide}
\heading{Jacobi iteration}

We now study a program to find an approximate solution to a large system of
simultaneous linear equations.

Given an $n$ by $n$ matrix~$A = (a_{ij})_{i,j = 0, \ldots, n-1}$, and a
vector~$b$ of size~$n$, we want to find a vector~$x$ of size~$n$ such that
$Ax = b$ (or, at least, $Ax \approx b$).
\end{slide}

%%%%%

\begin{slide}
\heading{Jacobi iteration}  

We decompose $A$ as $A = D+R$ where $D$ contains the diagonal entries of~$A$,
and $R$ contains the rest of the entries.  Then
\[
\begin{array}{cl}
& Ax = b \\
\iff & Dx + Rx = b \\
\iff & x = D^{-1} (b - Rx)
\end{array}
\]
%
provided $a_{ii} \ne 0$ for all~$i$ (so $ D^{-1}$ exists). 

This suggests calculating a sequence of approximations to the solution by
taking $x^{(0)}$ arbitrary (say all $0$s), and
\[
x^{(k+1)} = D^{ - 1} \left( b-R x^{(k)} \right).
\]
   
It can be shown that this iteration will converge on a solution if
\[
 \left \| a_{ii} \right \| > \sum_{j \ne i} \left \| a_{ij} \right \| 
  \qquad \mbox{for all $i$}. 
\]
\end{slide}

%%%%%

\begin{slide}
\heading{Jacobi iteration}  

It is convenient to rewrite the equation 
\[
x^{(k+1)} = D^{ - 1} \left( b-R x^{(k)} \right),
\]
in component form:
\[
x^{(k+1)}_i = \frac{1}{a_{ii}} \left(b_i -\sum_{j\ne i}a_{ij}x^{(k)}_j\right),\,
   i=0,\ldots,n-1. 
\]
\end{slide}

%%%%%

\begin{slide}
\heading{Implementing Jacobi iteration}

We will implement objects with the following signature.

\begin{scala}
trait Jacobi{
  val Epsilon = 0.000001 // tolerance

  /** Find x such that a x is approximately b, performing Jacobi iteration until
    * successive iterations vary by at most Epsilon.
    * Pre: a is of size n by n, and b is of size n, for some n. */
  def solve(a: Array[Array[Double]], b: Array[Double]): Array[Double]

  ...
}
\end{scala}

The |solve| function will return |x| such that |a x| is approximately |b|,
iterating until successive values are within |Epsilon| of one another. 
\end{slide}

%%%%%

\begin{slide}
\heading{Sequential solution}

We start by considering a sequential program for Jacobian iteration. 
On each iteration we need to \emph{simultaneously} set each
\SCALA{x(i)} by
\begin{scala}
  x(i) = update(a, b, x, i, n)
\end{scala}
where |update| calculates 
\( \frac{1}{a_{ii}} \left(b_i -\sum_{j\ne i}a_{ij}x^{(k)}_j\right) \).
\begin{scala}
  /** Calculate new value for x(i) based on the old value. */ 
  @inline protected def update(
    a: Array[Array[Double]], b: Array[Double], x: Array[Double], i: Int, n: Int)
      : Double = {
    var sum = 0.0
    for(j <- 0 until n; if j != i) sum += a(i)(j)*x(j)
    (b(i)-sum) / a(i)(i)
  }
\end{scala}
%
However, if we do this sequentially for each \SCALA{i}, we will not have
the same effect.
\end{slide}

%%%%%

\begin{slide}
\heading{Using a second array}

Instead we will use a pair of arrays |oldX| and |newX|.  On each iteration, we
will update |newX| based on |oldX|.  
\begin{scala}
  newX(i) = update(a, b, oldX, i, n)
\end{scala}
%
We will then swap the arrays for the following iteration.

We terminate when |Math.abs(oldX(i)-newX(i)) < Epsilon|, for all |i|.
\end{slide}

%%%%%

\begin{slide}
\heading{Sequential version}

\begin{scala}
/** A sequential implementation of Jacobian iteration. */
object SeqJacobi extends Jacobi{
  def solve(a: Array[Array[Double]], b: Array[Double]): Array[Double] = {
    val n = a.length
    require(a.forall(_.length == n) && b.length == n)
    var oldX, newX = new Array[Double](n); var done = false

    while(!done){
      done = true
      for(i <- 0 until n){
        newX(i) = update(a, b, oldX, i, n)
	done &&= Math.abs(oldX(i)-newX(i)) < Epsilon
      }
      if(!done){ val t = oldX; oldX = newX; newX = t } // swap arrays
    }
    newX
} }
\end{scala}
\end{slide}


%%%%%

\begin{slide}
\heading{Concurrent solution}

For the concurrent solution, suppose we use \SCALA{p} workers.

We will split \SCALA{oldX} and \SCALA{newX} into \SCALA{p} disjoint
segments, and arrange for each worker to complete one segment.

For simplicity, we will assume $\sm n \bmod \sm p = 0$, and take each
segment to be of height
\begin{scala}
val height = n/p
\end{scala}
\end{slide}

%%%%%

\begin{slide}
\heading{Concurrent solution}

The concurrent solution will proceed in rounds, each round corresponding to
one iteration of the sequential solution.  In each round, all workers can read
all of \SCALA{oldX} and each worker can write its own segment of \SCALA{newX}.
The roles of the arrays swaps between rounds.

We need to avoid race conditions, so we perform a barrier synchronisation at
the end of each stage.

%% An alternative approach is to use a single array, and to split each round
%% into two sub-rounds.  In the first sub-round, all threads read from shared
%% variables into local variables (but write no shared variables).  In the second
%% sub-round 
\end{slide}

%%%%%

\begin{slide}
\heading{Termination}

On each iteration, each thread can test whether its segment of \SCALA{x} has
converged, and store the result in a (thread-local) variable \SCALA{myDone}.

The iteration should terminate when \emph{all} the threads have $\sm{myDone} =
\sm{true}$.  We can test this as part of the barrier synchronisation of each
round.

A \emph{combining} barrier 
\begin{scala}
  val combBarrier = new CombiningBarrier(p, f)
\end{scala}
%
where |f: (A,A) => A| is an associative function.  Each thread contributes
some piece of data~$x_i: \sm{A}$ to each synchronisation.  They all receive
back  the value
\[
\sm f(x_0, \sm f(x_1, \sm f(x_2, \ldots, \sm f(x_{p-2}, x_{p-1})\ldots).
\]
where $x_0, \ldots, x_{p-1}$ are the data provided, in some order.  If |f| is
commutative, the order doesn't matter. 
\end{slide}

%%%%%

\begin{slide}
\heading{Termination}

In this case, we can define the combining barrier by:
%
\begin{scala}
private val combBarrier = new CombiningBarrier[Boolean](p, _ && _)
\end{scala}
%
Each thread can execute
\begin{scala}
done = combBarrier.sync(me, myDone)
\end{scala}
%
Each thread will receive back |true| if (and only if) all threads pass in
|true|. 

This form of combining barrier is very common, so CSO contains a built-in
equivalent form
\begin{scala}
private val combBarrier = new AndBarrier(p)
\end{scala}
\end{slide}

%%%%%

\begin{slide}
\begin{scala}
/** A concurrent implementation of Jacobian iteration, using shared variables. */
class ConcJacobi(p: Int) extends Jacobi{
  private val combBarrier = new AndBarrier(p)

  def solve(a: Array[Array[Double]], b: Array[Double]): Array[Double] = {
    val n = a.length
    require(a.forall(_.length == n) && b.length == n && n%p == 0)
    val height = n/p // height of one strip
    var x0, x1 = new Array[Double](n)
    var result: Array[Double] = null // ends up storing final result

    // Worker to handle rows [me*height .. (me+1)*height).
    def worker(me: Int) = thread{ ... }

    // Run system
    run(|| (for (i <- 0 until p) yield worker(i)))
    result
} }
\end{scala}
\end{slide}

%%%%%

\begin{slide}
\heading{A worker}

\begin{scala}
    def worker(me: Int) = thread{
      val start = me*height; val end = (me+1)*height
      var oldX = x0; var newX = x1; var done = false
      while(!done){
        var myDone = true
        for(i <- start until end){
          newX(i) = update(a, b, oldX, i, n)
          myDone &&= Math.abs(oldX(i)-newX(i)) < Epsilon
        }
        done = combBarrier.sync(me, myDone)
        if(!done){ val t = oldX; oldX = newX; newX = t } // swap arrays
      }
      // worker 0 sets result to final result
      if(me == 0) result = newX
    }
\end{scala}

    %% def worker(start: Int, end: Int) = thread{
    %%   var oldX = x0; var newX = x1; var done = false
    %%   while(!done){
    %%     var myDone = true
    %%     for(i <- start until end){
    %%       newX(i) = update(a, b, oldX, i, n)
    %%       myDone &&= Math.abs(oldX(i)-newX(i)) < Epsilon
    %%     }
    %%     done = combBarrier.sync(myDone)
    %%     if(!done){ val t = oldX; oldX = newX; newX = t } // swap references
    %%   }
    %%   // worker 0 sets result to final result
    %%   if(start == 0) result = newX
    %% }
\end{slide}

%%%%%

\begin{slide}
\heading{Workers}

Note that all workers use the same arrays for |oldX| and |newX| on each round,
because of the use of the barrier synchronisation.  This avoids races. 

Also one worker is responsible for writing the final value of |newX| into
|result|, so the |solve| function can return it.
\end{slide}

%%%%%

\begin{slide}
\heading{Testing}

We can test the concurrent version by comparing its results against those for
the sequential version.

\heading{Experimental results}

For small values of $n$, the sequential version is faster: the overheads of
the synchronisation outweigh the benefits of parallelisation.  But for larger
values of $n$, the concurrent version is faster.
\end{slide}

%%%%%

\begin{selfnote}
The computation time for each round is $\Theta(n^2)$ for the sequential
version, or $\Theta(n^2/p)$ for the concurrent version.  But there is a
communication time of $\Theta(n)$ for the concurrent version, to keep the
values in the caches up to date, and this is overwhelming for small values
of~$n$.
\end{selfnote}

%%%%%

\begin{slide}
\heading{An alternative approach}

An alternative approach is to use a single array, and to split each round
into two sub-rounds. 
\begin{itemize}
\item In the first sub-round, all threads read from the shared array into
  local variables (but write no shared variables).

\item In the second sub-round, all threads write to their part of the shared
  array. 
\end{itemize}
%
This requires an extra barrier synchronisation, between the two sub-rounds. 
\end{slide}

%%%%%%%%%%%%%%%%%%%%%%%%%%%%%%%%%%%%%%%%%%%%%%%%%%%%%%%%%%%%

\begin{slide}
\heading{A message-passing version}

We can convert the shared-memory program into a message-passing program,
with no shared variables.  Each thread has its own copy of~\SCALA{x} (and all
threads should have the same value for this).  On each iteration, each
worker calculates the next value of its share of~\SCALA{x}, and then
distributes it to all other workers.  More precisely, each thread sends a
triple:
%
\begin{itemize}
\item 
its own identity;

\item
the part of the next value of~\SCALA{x} that it has just calculated;

\item
a boolean that indicates whether it is willing to terminate.
\end{itemize}
%
\begin{scala}
type Msg = (Int, Array[Double], Boolean)
\end{scala}

We use buffered channels to pass these messages, and a |Barrier| object to
provide synchronisation.
\end{slide}

%%%%%

\begin{slide}
\heading{The first message-passing version}

\begin{scala}
/** A concurrent implementation of Jacobian iteration, using message
  * passing. */
class JacobiMP0(p: Int) extends Jacobi{
  private val barrier = new Barrier(p)
  /** Messages are triples: worker identity, new segment of x, is that worker
    * willing to terminate? */
  private type Msg = (Int, Array[Double], Boolean)
  /** Channels to send messages to workers: indexed by receiver's identity;
    * buffered. */
  private val toWorker = Array.fill(p)(new BuffChan[Msg](p-1))

  def solve(a: Array[Array[Double]], b: Array[Double]): Array[Double] = ...
}
\end{scala}
\end{slide}

%%%%%
\begin{slide}
\heading{The {\scalashape solve} function}
 
\begin{scala}
  def solve(a: Array[Array[Double]], b: Array[Double]): Array[Double] = {
    val n = a.length
    require(a.forall(_.length == n) && b.length == n && n%p == 0)
    val height = n/p // height of one strip
    var result: Array[Double] = null // will hold final result

    // Worker to handle rows [me*height .. (me+1)*height).
    def worker(me: Int) = thread{ ... } 

    // Run system
    run(|| (for(i <- 0 until p) yield worker(i)))
    result
  }
\end{scala}
\end{slide}

%%%%%

\begin{slide}
\heading{A worker}

\begin{scala}
    def worker(me: Int) = thread{
      val start = me*height; val end = (me+1)*height; var done = false
      val x = new Array[Double](n)

      while(!done){
        done = true
        // newX(i) holds the new value of x(i+start)
        val newX = new Array[Double](height)
        // Update this section of x, storing results in newX
        for(i <- start until end){
          newX(i-start) = update(a, b, x, i, n)
          done &&= Math.abs(x(i)-newX(i-start)) < Epsilon
        }
        ...
      }
      if(me == 0) result = x // copy final result  
    } 
\end{scala}
\end{slide}

%%%%%

\begin{slide}
\heading{A worker}

\begin{scala}
      while(!done){
        ...
        // Send this section to all other threads
        for(w <- 1 until p) toWorker((me+w)%p)!(me, newX, done)
        // Copy newX into x
        for(i <- 0 until height) x(start+i) = newX(i)
        // Receive from others
        for(k <- 0 until p-1){
          val (him, hisX, hisDone) = toWorker(me)?()
          for(i <- 0 until height) x(him*height+i) = hisX(i)
          done &&= hisDone
        }
        // Synchronise for end of round
        barrier.sync(me)
      }
      if(me == 0) result = x // copy final result      
\end{scala}
\end{slide}


%%%%%

\begin{slide}
\heading{Barrier synchronisation}

The  barrier synchronisation is necessary.

Suppose one thread is really fast.  It could complete one round, do its
calculations for the next round, and send its value of \SCALA{newX} while some
slow thread is still doing some sends from the previous round.

Hence a third thread could receive the fast thread's value for the next
round before it receives the slow thread's value for the current round.
\end{slide}

%%%%%%%%%%%%%%%%%%%%%%%%%%%%%%%%%%%%%%%%%%%%%%%%%%%%%%%%%%%%

\begin{slide}
\heading{A more efficient version}

The previous version is a bit inefficient, since it involves copying the data
that is received into~\SCALA{x}.  Thus each round takes time $O(\sm n)$.

We can replace \SCALA{x}, in each worker, by a two-dimensional array
\SCALA{xs}:
%
\begin{scala}
val xs = Array.ofDim[Double](p, height)
\end{scala}
%
Each row of~|xs| corresponds to the segment operated on by a particular
worker.  The array~|xs| represents the array |x| formed by concatenating the
rows of |xs|.  That is, we have the abstraction:
\[
\sm{x} = \sm{xs.flatten}.
\]

Each worker will receive a block of data from another worker, and insert it
into its~|xs| with a single update.  Then each round takes time $O(\sm{height}
+ \sm{p})$.

This makes the code about 45\% faster.  See website for the code.
\end{slide}

%%%%%

%% \begin{slide}
%% \heading{A more efficient version}

%% Thread \SCALA{worker(k)} will work on the row \SCALA{xs(k)}.  
%% \SCALA{xs(k)(i)} holds the value previously stored in \SCALA{x(k*height+i)}.

%% This row will be sent to all other workers, who will copy it into their
%% own~|xs|.  This copying can be done by copying references.

%% Note that it's important for each worker to set its value of \SCALA{newX} to
%% be a \emph{new} array on each iteration.  If it re-uses the same array, other
%% workers will be using references to that array (which were passed on the
%% previous round), and so there will be race conditions.
%% \end{slide}

%% %%%%%

%% \begin{slide}
%% \heading{A worker}

%% \begin{scala}
%%     def worker(me: Int) = thread{
%%       val start = me*height; val end = (me+1)*height; var done = false
%%       val xs = Array.ofDim[Double](p,height)
%%       // xs represents the vector x = xs.flatten

%%       while(!done){ ... }
%%       if(me == 0) result = xs.flatten // copy final result
%%     } // end of worker
%% \end{scala}
%% \end{slide}    

%% %%%%%

%% \begin{slide}
%% \heading{A worker}

%% \begin{scala}
%%       while(!done){
%%         done = true
%%         // newX(i) holds the new value of x(i+start) = xs(me)(i)
%%         val newX = new Array[Double](height)
%%         // Update this section of x, storing results in newX
%%         for(i1 <- 0 until height){
%%           val i = start+i1
%%           var sum = 0.0
%%           for(k <- 0 until p; j1 <- 0 until height){
%%             val j = k*height+j1; if(j != i) sum += a(i)(j)*xs(k)(j1)
%%           }
%%           newX(i1) = (b(i)-sum) / a(i)(i)
%%           done &&= Math.abs(xs(me)(i1)-newX(i1)) < Epsilon
%%         }
%%         ...
%%       }
%% \end{scala}
%% \end{slide}    

%%%%%

%% \begin{slide}
%% \heading{A worker}

%% \begin{scala}
%%       while(!done){
%%         ...
%%         // Send this section to all other threads
%%         for(w <- 1 until p) toWorker((me+w)%p)!(me, newX, done)
%%         // Copy newX into x
%%         xs(me) = newX
%%         // Receive from others
%%         for(k <- 0 until p-1){
%%           val (him, hisX, hisDone) = toWorker(me)?()
%%           xs(him) = hisX; done &&= hisDone
%%         }
%%         // Synchronise for end of round
%%         barrier.sync
%%       }
%% \end{scala}
%% \end{slide}    


%% \begin{slide}
%% \heading{Experimental results}

%% Here are some experimental results, with $n = 12800$, and with 20 workers
%% on a 10-processor machine with hyperthreading (results give mean times and 95\% confidence intervals).
%% %
%% \begin{center}
%% \begin{tabular}{ll}
%% shared variables &	770$\pm$12ms \\
%% first message-passing &	745$\pm$18ms \\
%% second message-passing & 396$\pm$18ms
%% \end{tabular}
%% \end{center}

%% Note how avoiding the copying makes the second message-passing version about
%% 45\% faster than the first message-passing version. 

%% I think the shared-variable version is slower because of the use of shared
%% variables.  I suspect values are being re-read from main memory, rather than
%% using values in caches, in case they have been updated by other threads.
%% \end{slide}
 % Jacobi iteration

%%%%%%%%%%%%%%%%%%%%%%%%%%%%%%%%%%%%%%%%%%%%%%%%%%%%%%%%%%%%

%%%%%

\begin{slide}
\heading{Communication with neighbours}

Several applications work on a rectangular grid, where the state of a cell on
one round depends only on the state of its neighbouring cells at the previous
round.  Examples:
%
\begin{itemize}
\item
Cellular automata;

\item
Solutions to differential equations, e.g.~in weather forecasting or fluid
dynamics;

\item Image processing, e.g.~smoothing.
\end{itemize}

In such cases, it is natural to allocate a horizontal strip of cells to each
thread.  At the end of each round, each thread communicates the state of its
top row to the thread above it, and communicates the state of its bottom row
to the thread below it.  

%% These rows can be passed by reference, but it might be necessary to make a
%% copy to avoid sharing references, e.g. \SCALA{up ! myA(start).toArray}.
\end{slide}

%%%%%

\begin{slide}
\heading{Passing arrays in Scala}

Arrays in Scala are reference objects, and so passed by reference.  

Suppose a particular thread has a two-dimensional array \SCALA{myA}, and it
is responsible for rows \SCALA{[start..end)}, then it can send its top row to
the thread above it (say on channel \SCALA{up}) by
%
\begin{scala}
up ! myA(start);
\end{scala}
%
The thread above it can receive it (say on channel \SCALA{receiveUp}) as:
%
\begin{scala}
myA(end) = receiveUp?()
\end{scala}
%
However, the two threads now share references to this array, so updates made
by one thread will have an effect on the other thread --- a race!

The solution is either for the first thread to re-initialise
\SCALA{myA(start)}, e.g., by \SCALA{myA(start) = new Array[Int](N)}, or to
make a \emph{copy} of the array, e.g., by \SCALA{up ! (myA(start).clone)}.
\end{slide}

%%%%%%%%%%%%%%%%%%%%%%%%%%%%%%%%%%%%%%%%%%%%%%%%%%%%%%%%%%%%
% Commented out section
%%%%%%%%%%%%%%%%%%%%%%%%%%%%%%%%%%%%%%%%%%%%%%%%%%%%%%%

\begin{comment}
\begin{slide}
\heading{Particle computations}

We now consider the problem of simulating the evolution of a large collection
of $N$ particles (e.g. stars or planets) that evolve under gravity.  

We can do a discrete time simulation, with time quantum \SCALA{deltaT}.

In particular, we'll consider how to construct a concurrent program for this
task.

For each particle, we need to record its mass, position and velocity:
%
\begin{scala}
val mass = new Array[Double](N)
type Vector = (Double, Double, Double)
val position = new Array[Vector](N)
val velocity = new Array[Vector](N)
\end{scala}
\end{slide}

%%%%%

\begin{slide}
\heading{Some physics}

Particle~$i$ will exert a force on particle~$j$ of magnitude
$G.mass(i).mass(j)/distance^2$, where $G \approx 6.67 \times 10^{-11}$ is the
gravitational constant, and $distance$ is the distance between them.  This is
an attractive force, with direction along the vector from $position(j)$ to
$position(i)$.  

We could calculate the total force exerted on each particle by all other
particles, and store the results in 
%
\begin{scala}
val force = new Array[Vector](N)
\end{scala}

We could then update the velocity of each particle~\SCALA{i}
by:\footnote{assuming we have defined \SCALA{+}, \SCALA{*} and \SCALA{/} to
  operate over \SCALA{Vector}}
%
\begin{scala}
velocity(i) += deltaT*force(i)/mass(i)
\end{scala}

And we could update the position of particle~\SCALA{i} by
%
\begin{scala}
position(i) += deltaT*velocity(i)
\end{scala}
\end{slide}

%%%%%

\begin{slide}
\heading{Calculating the forces}

Note that the force exerted by particle $i$ on particle $j$ is the same as the
force exerted by particle $j$ on particle~$i$ (except in the opposite
direction).  For reasons of efficiency, we do not want to calculate this
quantity twice.  

What we will do is allocate each thread some set~$S$ of particles.  For each
particle~$i \in S$, the thread will calculate the forces between $i$ and all
particles~$j$ with $j>i$.  These will be added to the total forces on both $i$
and $j$.  Something like:
%
\begin{scala}
for(i <- S; j <- i+1 until N){
  val thisForce = ... // force exerted on i by j
  force(i) += thisForce
  force(j) -= thisForce
}
\end{scala}
\end{slide}

%%%%%

\begin{slide}
\heading{Avoiding race conditions}

The code on the previous slide has an obvious race condition: several
threads might be trying to write to the same \SCALA{force(i)}
simultaneously. 

Instead we arrange for each thread~$me$ to write to its own vector of forces.
%
\begin{scala}
val force1 = new Array[Array[Vector]](p,N)
\end{scala}
%
Something like:
%
\begin{scala}
for(i <- S; j <- i+1 until N){
  val thisForce = ... // force exerted on i by j
  force1(me)(i) += thisForce
  force1(me)(j) -= thisForce
}
\end{scala}
\end{slide}

%%%%%

\begin{slide}
\heading{Calculating the total forces}

Once all the \SCALA{force1} values have been calculated, the threads can
perform a barrier synchronisation.  

Then the thread with set of particles~$S$ can, for each particle~$i \in S$,
calculate the total force and update the velocity and position:
%
\begin{scala}
for(i <- S){
  var force: Vector = (0.0, 0.0, 0.0)
  for(k <- 0 until p) force += force1(k)(i)
  velocity(i) += deltaT*force/mass(i)
  position(i) += deltaT*velocity(i)
}
\end{scala}
%
We expect to have $p \ll N$, so the cost of this extra summation is
comparatively small. 

The threads can then perform another barrier synchronisation before the next
round. 
\end{slide}

%%%%%

\begin{slide}
\heading{The pattern of synchronisation}

This pattern of synchronisation is very common:
%
\begin{scala}
<initialisation>
barrier.sync
while(true){
  <read all variables>
  barrier.sync
  <write own variables>
  barrier.sync
}
\end{scala}
%
The final synchronisation on each iteration could be replaced by a
synchronisation using a combining barrier, to decide whether to continue.
\end{slide}

%%%%%

\begin{slide}
\heading{Load balancing}

We want to choose the sets $S$ allocated to different threads so as to
balance the total load.

Note that the cost of calculating all the forces for particle~$i$ is
$\Theta(N-i)$: not all particles are equal in this regard.

One way to balance the load is to split the $N$ particles into $2p$
segments, each of size $segSize = N/2p$.  Then we can allocate process~$me$
the segments~$me$ and $2p-me-1$, i.e.\ particles $[me.segSize \upto
  (me+1).segSize)$ and $[(2p-me-1).segSize \upto (2p-me).segSize)$.
\end{slide}
\end{comment}

%%%%%%%%%%%%%%%%%%%%%%%%%%%%%%%%%%%%%%%%%%%%%%%%%%%%%%% End of comment

%%%%%

\begin{slide}
\heading{Summary}

\begin{itemize}
\item 
Synchronous data parallel programming;
% Heart-beat algorithms;

\item
Barrier synchronisation; combining barrier synchronisation;

\item
Examples, using shared memory or message-passing.
\end{itemize}
\end{slide}

\end{document}


