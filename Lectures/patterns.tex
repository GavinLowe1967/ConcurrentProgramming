\documentclass[notes,color]{sepslide0}
\usepackage{graphicx}
\usepackage[overheads]{mysepslides}
\usepackage{tech,graphicx,url,tikz,scalalistings}

\title{Patterns of Concurrent Programming} 
\author{Gavin Lowe}

% \everymath{\color{Plum}}
% \def\smaller{\small} 
\def\scalacolour{\color{violet}}

\begin{document}

\begin{slide}
  
  \Title

\end{slide}

%%%%%

\begin{slide}
\heading{Patterns of concurrent programming} 

We've seen a number of patterns of concurrent programming so far:
%

\begin{itemize}
\item Concurrent datatypes;

\item
Bag-of-tasks (with or without replacement); 

\item
Synchronous computation; % heart-beat algorithms;

\item
Interacting peers: central controller; fully-connected topology; ring
topology; tree topology;

\item
Clients and servers.
\end{itemize}

In this chapter we will look at a few other patterns of concurrent programming.
\end{slide}

%%%%%


\begin{slide}
\heading{Recursive parallelism}

Many sequential programs use a recursive procedure: a procedure that calls
itself.

The idea of recursive parallelism is that the recursive calls are replaced by
spawning off parallel processes with the same effect.  This is particularly
useful where the procedure would make two or more recursive calls to itself
that are independent (operate on disjoint data): the corresponding recursive
parallel processes can then run concurrently.

A typical pattern for recursive parallelism is:
%
\begin{itemize}
\item
In some base case(s), calculate the result directly;

\item
In other cases, spawn off a parallel process corresponding to each recursive
call, together with a controller to tie things together.
\end{itemize}

\end{slide}


%% %%%%% 

\begin{slide}
\heading{Example: Quicksort}

We saw an implementation of Quicksort using recursive parallelism and message
passing in Chapter~2.

We will now implement Quicksort using recursive parallelism and shared
variables, to sort an array~|a|.  Different threads will work on disjoint
segments of~|a|, to avoid race conditions.

We can define a sequential partition function in the normal way.
%
\begin{scala}
  /** Partition the segment a[l..r).
    * Permute a[l..r) and return k s.t. a[l..k) <= a(k) < a[k+1..r).  */
  private def partition(l: Int, r: Int): Int = ... // standard definition
\end{scala}
\end{slide}

%%%%%

\begin{slide}
\heading{Example: Quicksort}

The recursive parallel definition creates one process to sort each subsegment.
%
\begin{scala}
  /** Sort a[l..r) using recursive parallelism. */
  private def qsort(l: Int, r: Int): ThreadGroup = thread{
    if(l+1 < r){
      val m = partition(l, r)
      run(qsort(l, m) || qsort(m+1, r))
    }
  }

  /** Sort a using recursive parallelism. */
  def apply() = run(qsort(0, a.length))
\end{scala}
\end{slide}

%%%%%


\begin{slide}
\heading{Limits of recursive parallelism}

In many applications, full recursive parallelism will lead to many more
processes than there are processors.  The overheads involved in spawning off
new processes, and context-switching between processes will mean that this is
very inefficient.

A better way is to limit the number of processes.  The simplest way to do this
is to switch to the sequential algorithm for small sub-problems.
\end{slide}

%%%%%

\begin{slide}
\heading{Recursive parallel quicksort with a limit}

\begin{scala}
  /** Minimum size of segment to sort with recursive parallelism. */
  val parLimit = a.length / 20

  /** Sort a[l..r) using recursive parallelism for segments of size at least
    * parLimit. */
  private def qsortLimit(l: Int, r: Int): ThreadGroup = thread{
    if(l+1 < r){
      val m = partition(l, r)
      if(r-l >= parLimit) run(qsortLimit(l, m) || qsortLimit(m+1, r))
      else{ run(qsortLimit(l, m)); run(qsortLimit(m+1, r)) }
    }
  }
\end{scala}
\end{slide}

%%%%%

\begin{slide}
\heading{Quicksort using the bag-of-tasks pattern}

An alternative concurrent implementation of quicksort would use a fixed number
of threads and the bag-of-tasks pattern, with replacement.  

The bag of tasks could be implemented by a concurrent stack or queue
(supporting termination).  Each task is a pair |(l,r)| representing that the
segment |a[l..r)| has to be sorted.
\end{slide}

%%%%%

\begin{slide}
\heading{Quicksort using the bag-of-tasks pattern}


Each thread repeatedly:
%
\begin{itemize}
\item Removes a task from the bag;

\item If the task is sufficiently small, sorts the segment sequentially;

\item Otherwise, it partitions the segment, and returns two tasks to the bag,
  representing that the two sub-intervals need to be sorted.
\end{itemize}

The bag is likely to prove a bottleneck.  Possible enhancements are:
%
\begin{itemize}
\item Don't return empty or singleton tasks to the bag;

\item Rather than returning two bags to the task, just return one, and
  continue working on the other;

\item Using a more sophisticated concurrent data structure.
\end{itemize}
\end{slide}


%%%%%

%% \begin{slide}
%% \heading{Parallel collections}

%% Scala's parallel collections
%% API\footnote{%
%% \url{https://docs.scala-lang.org/overviews/parallel-collections/overview.html}.}
%% aims to support parallelism in a way that means the programmer doesn't have to
%% think too much about parallelism.  Example:
%% %
%% \begin{scala}
%% scala> (1 to 10000).toList.par.sum
%% res0: Int = 50005000
%% \end{scala}

%% Internally, recursive parallelism is used.  To sum a long list, it can be
%% split into two parts; each is summed, in parallel (possibly leading to further
%% splits); then the subresults are summed. 

%% Of course, careless use can create race conditions.
%% \begin{scala}
%%   var sum = 0
%%   (1 to 10000).toList.par.foreach(sum += _)
%% \end{scala}

%% \vfill
%% \end{slide}

%
\begin{slide}
\heading{Bag of tasks with replacements}

We have already seen the \emph{bag of tasks} pattern:
%
\begin{itemize}
\item
A controller process holds a bag of tasks that need completing;

\item
Worker processes repeatedly obtain a task from the controller, and complete
it.
\end{itemize}

We saw this in the numerical integration problem, where a task represented an
interval whose integral needed estimating; no tasks were ever returned to the
bag, so this made the controller very simple.

We will now look at a more interesting example, where tasks are returned to
the bag.
\end{slide}

%%%%%

\begin{slide}
\heading{Magic squares}

A \emph{magic square} of size~$n$ is an $n$ by $n$ square, containing the
integers from $1$ to $n^2$, such that the sums of all the rows, columns and
diagonals are the same.  For example:
\[
\begin{array}{\|cccc\|}
\hline
16 & 4 & 13 & 1  \\
2 & 5 & 12 & 15  \\
9 & 14 & 3 & 8  \\
7 & 11 & 6 & 10 \\\hline
\end{array}
\]
We will design a concurrent program, using the bag of tasks pattern, to find
magic squares.

Similar techniques can be used for many other search problems.
\end{slide}

%%%%%

\begin{slide}
\heading{Partial solutions}

Each task will correspond to a partial solution of the puzzle, i.e. where
some, but not necessarily all, of the locations have been filled in.  

A worker process will obtain a partial solution, and, if it is not complete:
%
\begin{itemize}
\item
choose an empty square;

\item
create all the partial solutions obtained by filling that square with a value
that might lead to a complete solution (there might be none);

\item
return those new partial solutions to the bag.
\end{itemize}
\end{slide}

%%%%%

\begin{selfnote}

For example, given the partial solution
\[
\begin{array}{|c|c|c|c|}
\hline
16 & 4 & 11 & 3\;\:  \\\hline
2 & 12 &  &   \\\hline
 &  &  &   \\\hline
&  &  &  \\ \hline
\end{array}
\]
The next position has to be at least 5 to make it possible for that row to add
up to 34; so could be 5, 6, 7, 8, 9, 10, 13, 14, 15.

If 5 is selected, the next value has to be 15.

Alternatively, if 8 were selected, no value is posible.
\end{selfnote}

%%%%%

\begin{slide}
\heading{Representing partial solutions}

We will represent partial solutions using objects of the following class.
%
\begin{scala}
class PartialSoln(n : Int){
  // Create an empty partial solution
  def empty = {...} 

  // Is the partial solution completed?
  def finished : Boolean = {...} 

  // Is it legal to play piece k in position (i,j)
  def isLegal(i: Int, j: Int, k: Int) : Boolean = {...}

  ...
}
\end{scala}
\end{slide}

%%%%%

\begin{slide}
\heading{Representing partial solutions}

\begin{scala}
class PartialSoln(n : Int){ ...
  // Return new partial solution obtained by placing 
  // k in position (i,j)
  def doMove(i: Int, j: Int, k: Int) : PartialSoln = {...}

  // Choose a position in which to play
  def choose : (Int,Int) = {...}

  // Print board
  def printBoard = {...}
}
\end{scala}

We won't discuss the implementation of \SCALA{PartialSoln} (although this
makes a big difference to the efficiency of the program).  We will concentrate
on the concurrent programming aspects of the problem. 
\end{slide}

%%%%%

\begin{slide}
\heading{Channels}

The workers will use the following channels:
\begin{scala}
val get = OneMany[PartialSoln]; 
// get jobs from the controller

val put = ManyOne[PartialSoln];
// pass jobs back to the controller

val done = ManyOne[Unit]; 
// tell the controller we've finished a job

val toPrinter = ManyOne[PartialSoln]; 
// channel to printer
\end{scala}
\end{slide}

%%%%%

\begin{selfnote}
Use a separate process for printing.  Why?
\end{selfnote}

%%%%%

\begin{slide}
\heading{A worker}

\begin{scala}
def Worker(n: Int, get: ?[PartialSoln], put: ![PartialSoln], 
           done: ![Unit], toPrinter: ![PartialSoln]) 
= proc("Worker"){
  repeat{
    val partial = get? ; // get job
    if(partial.finished) toPrinter!partial;  // done!
    else{
      val(i,j) = partial.choose;
      // Generate all next-states
      for(k <- 1 to n*n)
        if(partial.isLegal(i,j,k)) 
          put!partial.doMove(i,j,k); 
    }
    done!();
  }
}
\end{scala}
\end{slide}

%%%%%

\begin{slide}
\heading{The Controller}

The Controller will use a stack to store all the current partial solutions.

\emph{Question:} what difference would using a queue make?

The controller needs to know when it can terminate.  This will be when it is
holding no partial solutions, and none of the workers is busy; it therefore
needs to keep track of the number of busy workers (that's the reason for the
\SCALA{done} channel).
\end{slide}

%%%%%

\begin{slide}
\heading{The Controller}

\begin{scala}
def Controller(n: Int, get: ![PartialSoln], put: ?[PartialSoln], 
               done: ?[Unit], toPrinter: ![PartialSoln])
= proc("Controller"){
  // Initialise stack holding empty board
  val init = new PartialSoln(n);
  init.empty ;
  val stack = new scala.collection.mutable.Stack[PartialSoln];
  stack.push(init);

  var busyWorkers = 0; // # workers currently busy

  ...
}
\end{scala}
\end{slide}

%%%%%

\begin{slide}
\heading{The Controller}

\begin{scala}
def Controller(...) = proc{
  ...
  // Main loop
  serve(
    (!stack.isEmpty &&& get) -!-> 
      { get!(stack.pop) ; busyWorkers += 1; }
    | (busyWorkers>0 &&& put) -?-> 
      { val ps = put? ; stack.push(ps); }
    | (busyWorkers>0 &&& done) -?-> 
      { done? ; busyWorkers -= 1 }
  )
       
  // Finished, when stack.isEmpty and busyWorkers==0
  toPrinter.close; get.close;
}
\end{scala}
\end{slide}
 %% omit??

%%%%%

\begin{slide}
\heading{Competition parallel}

The idea of competition parallel is to choose two (or more) different
algorithms for a problem, and to run them independently in parallel, and see
which finishes first.

This works well on the boolean satisfiability problem.  The problem is
NP-complete.  However,  there are a number of SAT-solving algorithms that work
well in many cases.  But all known algorithms have cases where they perform
badly, and different algorithms perform badly on different cases.  Therefore
running two algorithms in competition with one another can give better average
results than parallelising a single algorithm.
\end{slide}

%%%%%

\begin{selfnote}
Boolean satisfiability problem: given a boolean formula such as:
\[
(b_1 \lor b_2 \lor \lnot b_3) \land (\lnot b_1 \lor b_3) \land (\lnot b_1 \lor
b_2 \lor b_3)
\]
is there a way of choosing values for the boolean variables to make the
formula true?

Lots of problems can be mapped onto SAT-solving, e.g. constraint satisfaction,
model checking. 
\end{selfnote}

%%%%%

\begin{slide}
\heading{Task parallel programming}

Most of the concurrent programs we have seen so far have been \emph{data
  parallel}: the data has been split up between different processes, each of
which have performed the same task on its data.

An alternative is \emph{task parallel programming}\footnote{This is a
  different meaning of the word ``task'' than in the bag-of-tasks pattern.},
where different processes perform different operations on the same data,
typically in some kind of pipeline.

\vfill
\end{slide}

%%%%%

\begin{slide}
\heading{Examples of task parallel programming}

\begin{itemize}
\item
Compilers typically operate in a number of stages, e.g., lexical analysis,
syntactical analysis, semantic analysis, type checking, code generation,
optimisation.  Each stage can be implemented by a separate process, passing
its output to the next process. 

\item
Unix pipes, e.g.~\texttt{ls -R \| grep elephant \| more}.

\item 
Google queries: involve searching various indexes, combining the results,
generating adverts, logging, creating the HTML, etc.
\end{itemize}
\end{slide}

%%%%%

%% \begin{slide}
%% \heading{MapReduce}

%% MapReduce is a concurrent programming model introduced by researchers from
%% Google,\footnote{{\it MapReduce: Simplified Data Processing on Large Clusters},
%%   Jeffrey Dean and Sanjay Ghemawat,
%%   \url{http://labs.google.com/papers/mapreduce.html}} for execution on large
%% clusters of machines (hundreds or thousands), operating on large data sets
%% (terabytes).

%% \vfill
%% \end{slide}

%% %%%%%

%% \begin{slide}
%% \heading{MapReduce}

%% The programmer defines two functions:
%% %
%% \begin{scala}
%% def map(key1: Key1, val1: Val1) : List[(Key2, Val2)] = ...

%% def reduce(key2: Key2, vals2: List[Val2]) : Val3 = ...
%% \end{scala}
%% %
%% The effect of the corresponding MapReduce program is:
%% %
%% \begin{itemize}
%% \item
%% For each \SCALA{(key1, val1)} in the input (typically these are (filename,
%% contents) pairs), calculate \SCALA{map(key1, val1)} to produce a set of
%% intermediate pairs;

%% \item
%% For each \SCALA{key2} appearing in the intermediate pairs, let \SCALA{vals2}
%% be the list of \SCALA{val2} such that \SCALA{(key2, val2)} is in the
%% intermediate pairs, calculate \SCALA{reduce(key2, vals2)}; combine the results
%% to give a result of type \SCALA{List[(Key2, Val3)]} (ordered by the
%% \SCALA{Key2} elements).
%% \end{itemize}
%% \end{slide}

%% %%%%%

%% \begin{slide}
%% \heading{Example: counting occurences of words}

%% Suppose we want to count the number of occurrences of different words in a
%% large collection of documents.  We can define (where \SCALA{words} splits a
%% string into a list of words):
%% %
%% \begin{scala}
%% def map(filename: String, contents: String) : List[(String, Int)] =
%%   for(word <- words(contents)) yield (word, 1)

%% def reduce(word: String, counts: List[Int]) = counts.sum
%% \end{scala}
%% \end{slide}

%%%%%

%% \begin{slide}
%% \heading{MapReduce}

%% The run-time system parallelizes the implementation of the MapReduce, using
%% the bag-of-tasks pattern, where each task is either a map or reduce on some
%% block of data.

%% The run-time system takes care of issues such as implementing the controller,
%% fault tolerance, locality (so tasks are scheduled on machines where the data
%% is stored, as far as possible), and various optimisations.

%% The pattern seems to be quite generally applicable.  Further, programmers find
%% it easy to use, since most of the time the semantics is the same as for
%% sequential execution.  
%% \end{slide}

%%%%%


\begin{slide}
\heading{Futures}

A \emph{future} is a value that is computed in parallel with the main
computation, to be used at some time in the future.  For example:
%
\begin{scala}
val x = scala.concurrent.ops.future(
  <some lengthy computation>
)
...
val y = f(x())
\end{scala}
%
The expression \SCALA{x()} returns the result of the computation, waiting
until it is complete if necessary.

Of course, it is still necessary to avoid race conditions.
\end{slide}

%%%%%

\begin{slide}
\heading{A simple implementation of futures}

\begin{scala}
class Future[A](exp: => A){
  /** Variable to hold the result. */
  private var result = null.asInstanceOf[A]

  /** Is the result valid? */
  private var done = false

  /** Server process to calculate the result. */
  private def server = thread{
    synchronized{ result = exp; done = true; notifyAll() }
  }

  server.fork

  /** Get the value of exp. */
  def apply(): A = synchronized{ while(!done) wait(); result }
}
\end{scala}
\end{slide}

%%%%% 

\begin{slide}
\heading{Lock-free programming}

In this course, we've taken the view that shared variables should be accessed
without race conditions.  We've achieved this by different mechanisms:
%
\begin{itemize}
\item By using pure message passing, with different threads having disjoint
  variables;

\item By having different threads allowed to access each variable in different
  states, coordinated by message passing and/or barrier synchronisations;

\item By locking, using a monitor (or, in the next chapter, semaphores).
\end{itemize}
\end{slide}

%%%%%

\begin{slide}
\heading{Lock-free programming}

In a message-passing system, certain components, particularly servers, can be
a bottleneck.  Likewise, any object that uses locking can be a bottleneck.
These bottlenecks prevent scaling of performance.

An alternative approach is not to use locking, and, instead, to allow race
conditions in programs.  This is the approach taken in the Concurrent
Algorithms and Data Structures course.

However, this approach raises problems.  We've previously seen problems
concerning cache consistency and compiler optimisations.  But even without
these, it turns out that there are certain problems that cannot be solved
using standard registers.  For example, it can be proved that it's impossible
to implement a wait-free concurrent queue in this setting.
\end{slide}

%%%%%

\begin{slide}
\heading{Lock-free programming}

Instead, lock-free programming makes use of additional operations provided by
the hardware.  In particular, it uses a \emph{compare-and-set (CAS)} operation
on each register.  The CAS operation takes two parameters, an expected value,
and a new value; if the current value of the register equals the expected
value, it is replaced by the new value, and |true| is returned; otherwise the
register is unchanged and |false| is returned.  In pseudo-code:
%
\begin{scala}
def compareAndSet(expected: Int, newValue: Int): Boolean = atomically{
  if(current == expected){ current = newValue; true } else false
}
\end{scala}

The CAS operation can be used to implement many lock-free algorithms and data
structures. 
\end{slide}



%%%%%

\begin{slide}
\heading{Summary}

\begin{itemize}
\item
Recursive parallelism; limited recursive parallelism.

\item
Competition parallel.

\item 
Task parallel.

\item
Futures.

\item
Lock-free concurrent programming.
\end{itemize}
\end{slide}


\end{document}

