



%%% I don't like how this is set out.  I don't think the first version is
%%% sound.  There's a better version in
%%% /users/gavinl/Teaching/CP/Scala/ReadersWriters.

% {\advance\slideheight by 3mm
\begin{slide}
\heading{The readers and writers problem using semaphores}

We will now consider an implementation of the readers and writers problem
using semaphores. 

As before, we keep track of the number of readers and writers in the
critical section.
\begin{scala}
  /** Number of readers in the CS. */
  private var readers = 0 

  /** Number of writers in the CS. */
  private var writers = 0 
\end{scala}
\end{slide}

%%%%%

\begin{slide}
\heading{The readers and writers problem using semaphores}

We will use the technique of passing the baton.  We will have separate
semaphores for signalling (passing the baton) to readers and writers.  A
signal on these semaphores will indicate that a waiting thread can enter. 

\begin{scala}
  /** Semaphore to signal that reader can enter.
    * Indicates that writers == 0. */
  private val readerEntry = new SignallingSemaphore

  /** Semaphore to signal that writer can enter.
    * Indicates that writers == 0 && readers == 0. */
  private val writerEntry = new SignallingSemaphore
\end{scala}
\end{slide}

%%%%%

\begin{slide}
\heading{The readers and writers problem using semaphores}

We need to keep track of how many readers and writers are waiting, so an
exiting thread can know to whom to pass the baton.  We need a mutex semaphore
to protect the object variables.
%
\begin{scala}
  /** Number of readers waiting to enter. */
  private var readersWaiting = 0

  /** Number of writers waiting. */
  private var writersWaiting = 0

  /** Semaphore to protect shared variables. */
  private val mutex = new MutexSemaphore
\end{scala}
\end{slide}

%%%%%

\begin{slide}
\heading{The reader entering}

If a reader has to wait, when it is awoken, it signals to the next reader if
there is one. 
%
\begin{scala}
  def readerEnter = {
    mutex.down
    if(writers == 0){ // go straight in
      readers += 1; mutex.up
    }
    else{ // have to wait
      readersWaiting += 1; mutex.up
      readerEntry.down                         // wait to be released (1)
      assert(writers == 0); readersWaiting -= 1; readers += 1
      if(readersWaiting > 0) readerEntry.up // signal to next reader at (1)
      else mutex.up
    }
  }
\end{scala}
\end{slide}

%%%%%

\begin{slide}
\heading{The writer entering}

\begin{scala}
  def writerEnter = {
    mutex.down
    if(readers == 0 && writers == 0){ // go straight in
      writers = 1; mutex.up
    }
    else{ // have to wait
      writersWaiting += 1; mutex.up
      writerEntry.down              // wait to be released (2)
      writersWaiting -= 1
      assert(readers == 0 && writers == 0); writers = 1
      mutex.up
    }
  }
\end{scala}
\end{slide}

%%%%%

\begin{slide}
\heading{Readers and writers leaving}

A leaving reader can signal to a waiting writer.  A leaving writer can signal
to either a waiting reader or waiting writer.
%
\begin{scala}
  def readerLeave = {
    mutex.down
    readers -= 1; assert(readersWaiting == 0)
    if(readers == 0 && writersWaiting > 0)
      writerEntry.up                               // signal to writer at (2)
    else mutex.up
  }

  def writerLeave = {
    mutex.down
    writers = 0
    if(readersWaiting > 0) readerEntry.up        // signal to reader at (1)
    else if(writersWaiting > 0) writerEntry.up   // signal to writer at (2)
    else mutex.up
  }
\end{scala}
\end{slide}

%%%%%%%%%%%%%%%%%%%%%%%%%%%%%%%%%%%%%%%%%%%%%%%%%%%%%%% 

\begin{slide}
\heading{Starvation}

The previous version could cause a writer to be starved, if readers
continually enter and leave the critical section such that there is always at
least one reader.  To avoid this, we force a new reader to wait if there is
already a waiting writer.

\begin{scala}
  def readerEnter = {
    mutex.down
    if(writers == 0 && writersWaiting == 0){ // go straight in
      readers += 1; mutex.up
    }
    else{ ... } // as before
  }
\end{scala}

The assertion in |readerLeave| needs to be changed to 
\begin{scala}
  assert(readersWaiting == 0 || writersWaiting > 0)
\end{scala}
\end{slide}

%%%%%

\begin{slide}
\heading{Starvation}

On the assumption that no thread remains in the critical section for ever,
this ensures that writers collectively are not starved.  Once a writer starts
waiting, no more readers will enter the CS; eventually all readers will leave
the CS; and a writer will receive a signal.
\end{slide}

%%%%%

\begin{slide}
\heading{Fairness}

The implementation is potentially unfair to an \emph{individual} writer.  If
repeatedly several writers are waiting, a particular writer may never receive
the signal.  Likewise, a thread may never receive a signal on |mutex|
(although this is less likely). 

The implementation of a semaphore does not guarantee to be fair to each
individual waiting thread, although in practice it might be.

%% The semaphore implementation includes the option of providing fairness to each
%% thread waiting on it.
%% %
%% \begin{scala}
%%   private val mutex = MutexSemaphore(fair = true)
%%   private val readerEntry = SignallingSemaphore(fair = true)
%%   private val writerEntry = SignallingSemaphore(fair = true)
%% \end{scala}

%% ``Fair'' here means that if a thread~$t$ is waiting on semaphore~$s$, and an
%% infinite number of |up|s are performed on~$s$, then $t$ will receive one of
%% these signals.
%% %
%% Note that this is implied by the stronger property of first-in first-out.
%% %
%% Note also that there is a performance overhead associated with fairness.
\end{slide}

