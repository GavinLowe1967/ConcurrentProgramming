\documentclass[notes,color]{sepslide0}
\usepackage{graphicx}
\usepackage[overheads]{mysepslides}
\usepackage{tech,graphicx,url,csp-cm,tikz,scalalistings}

\title{Using Explicit Locks} 
\author{Gavin Lowe}

% \everymath{\color{Plum}}
\def\smaller{\small} 
\def\scalacolour{\color{violet}}

\begin{document}

\begin{slide}
  
  \Title

\end{slide}

%%%%%

\begin{slide}
\heading{Locks}

In this chapter we will see how to use explicit locks.  An SCL lock can be
created as
%
\begin{scala}
val lock = new Lock
\end{scala}

A lock has two principle operations:
%
\begin{itemize}
\item |acquire|: wait until no other thread holds the lock, and then acquire
  it;

\item |release|: release the lock.
\end{itemize}

We will see other uses of locks in the next chapter. 

Most other languages provide an implementation of locks with a similar
interface. 
\end{slide}

%%%%%

\begin{slide}
\heading{Using a lock}

Here's a counter object, using a lock to protect its private variable.
%
\begin{scala}
object Counter{
  private var x = 0
  private val lock = new Lock
  def inc = { lock.acquire; x = x+1; lock.release }
  def dec = { lock.acquire; x = x-1; lock.release }
  def getX = { lock.acquire; val result = x; lock.release; result }
}
\end{scala}
%
This ensures the operations are performed under mutual exclusion. 
\end{slide}

%%%%%

\begin{slide}
\heading{Mutual exclusion}

The |Lock| class also contains the following helper function, to perform some
computation under mutual exclusion. 
\begin{scala}
  /** Execute `comp` under mutual exclusion for this lock. */
  def mutex[A](comp: => A): A = {
    acquire; try{comp} finally{release}
  } 
\end{scala}
\end{slide}

%%%%%

\begin{slide}
\heading{Mutual exclusion}

Here's the counter using |mutex|.
%
\begin{scala}
object Counter{
  private var x = 0
  private val lock = new Lock
  def inc = lock.mutex{ x = x+1 }
  def dec = lock.mutex{ x = x-1 }
  def getX = lock.mutex{ x }
}
\end{scala}
%
Using |mutex| is more convenient, particularly if a function has several
branches, or if the function returns a value that requires mutual exclusion to
calculate. 

This plays a similar role to a |synchronized| block in a JVM monitor.
\end{slide}

%%%%%

\begin{slide}
\heading{Re-entry}

|Lock|s are re-entrant: a thread can acquire a |Lock| that it already holds;
and needs to release it a corresponding number of times before another
thread can acquire it.

For example:
\begin{scala}
def f(x: Int) = lock.mutex{ ...; g(y); ... }
def g(y: Int) = lock.mutex{ ... }
\end{scala}
\end{slide}


%%%%%

\begin{slide}
\heading{Example: using a monitor to protect an array}

Suppose we want to protect an array~|a| against race conditions.  We
can wrap it inside a monitor as follows.
%
\begin{scala}
class ArrayWrapper[T](a: Array[T]){
  private val lock = new Lock
  /** Get a(i). */
  def get(i: Int): T = lock.mutex{ a(i) }
  /** Set a(i) to x. */
  def set(i: Int, x: T) = lock.mutex{ a(i) = x }
}
\end{scala}
%
We might want more operations, for example to increment a particular
location. 

However, this could be a bottleneck: while one thread is accessing
the array, other threads have to wait, even if they want to access 
different locations in the array.
\end{slide}

%%%%%

\begin{slide}
\heading{Fine-grained locking}

An alternative is to use a separate lock for each entry in the array.
%
\begin{scala}
class ArrayWrapper[T](a: Array[T]){
  /** locks(i) protects access to a(i). */
  private val locks = Array.fill(a.size)(new Lock)

  /** Get a(i). */
  def get(i: Int): T = locks(i).mutex{ a(i) }

  /** Set a(i) to x. */
  def set(i: Int, x: T) = locks(i).mutex{ a(i) = x }
}
\end{scala}
%
Note how different threads can simultaneously access different entries
in the array.

However, this solution is potentially expensive on memory, as |a.size|
extra lock objects have to be created. 
\end{slide}

%%%%%

{\advance \slideheight by 4mm
\begin{slide}
\heading{Striped locking}

An alternative solution is to create some number |numLocks| of lock
objects (with $\sm{numLocks} < \sm{a.size}$) where each lock protects
several entries in the array.
%
\begin{scala}
class ArrayWrapper[T](a: Array[T], numLocks: Int){
  /** locks(j) protects access to each a(i) such that i%numLocks = j. */
  private val locks = Array.fill(numLocks)(new Lock)

  /** Get a(i). */
  def get(i: Int): T = locks(i%numLocks).mutex{ a(i) }

  /** Set a(i) to x. */
  def set(i: Int, x: T) = locks(i%numLocks).mutex{ a(i) = x }
}
\end{scala}
%
If |numLocks| is somewhat larger than the number of threads, then
threads will normally not have to wait (under reasonable assumptions).
\end{slide}}

%%%%%

\begin{slide}
\heading{Sharding}

Consider the concurrent set, again.  Suppose this proves to be a bottleneck.

We can split the set up into |n| distinct constituent sets or \emph{shards}.
The object represents the union of these shards.

We can insert a value |x| into the set based upon its hash code: we can place
|x| into the shard with index $\sm{x.hashCode} \bmod
\sm{n}$.\footnote{Be careful: the \% operation works oddly on negative
  values.}

Each shard can be locked independently, so values can be added to different
shards concurrently. 

%% This technique is known as \emph{sharding}.  Each constituent set is known as
%% a \emph{shard}. 

If |n| is a few times larger than the number of threads, and the threads
access shards independently, then most operations will not be blocked.

\vfill
\end{slide}

%%%%%

\begin{slide}
\heading{A sharded set}

\begin{scala}
class ShardedSet[A](n: Int){
  /** The consituent sets.  theSets(i) contains only values x such that
    * indexFor(x) = i.  This represents the union of the theSets(i). */
  private val theSets = Array.fill(n)(scala.collection.mutable.Set[A]())

  /** Locks.  Each locks(i) protects theSets(i). */
  private val locks = Array.fill(n)(new Lock)

  /** The index of the shard used for x. */
  private def indexFor(x: A) = (x.hashCode & 0x7FFFFFFF) % n

  /** Add x to this set.  Return true if x was not previously in the set. */
  def add(x: A): Boolean = {
    val index = indexFor(x); locks(index).mutex{ theSets(index).add(x) }
  }
}
\end{scala}
%
Other operations can be added in a similar way. 
\end{slide}

%%%%%

\begin{slide}
\heading{Fine-grained locking}

The periods during which a thread holds different locks can overlap in
arbitrary ways, for example:
%
\begin{scala}
  ...; lock1.acquire; ...; lock2.acquire; ...; lock1.release; ...; lock2.release
\end{scala}
%
One cannot achieve a similar effect using |synchronized| blocks, which have to
be well bracketed.
\end{slide}

%%%%%

\begin{slide}
\heading{Summary}

\begin{itemize}
\item Locks and mutual exclusion;

\item Advanced techniques: fine-grained locking; striped locking; sharding. 
\end{itemize}
\end{slide}

%% \begin{slide}
%% \heading{Alternative: explicit locks}

%% An alternative to using |synchronized| blocks is to use an explicit lock.  The
%% Java interface |java.util.concurrent.locks.Lock| defines locks with
%% operations:
%% %
%% \begin{itemize}
%% \item |lock()|: acquire the lock, blocking until it becomes available;
%% \item |unlock()|: release the lock;
%% \item |tryLock()|: try to acquire the lock, but give up if another thread
%%   holds it; return a boolean to indicate if successful. 
%% \end{itemize}
%% %
%% If |l| is a |Lock|, then the standard way to use |l| to protect a computation
%% |comp| is
%% %
%% \begin{scala}
%% l.lock(); try{ comp } finally{ l.unlock() }
%% \end{scala}
%% %
%% The use of |try{...} finally{...}| ensures that |l| is released on every
%% branch of~|comp|, including if it throws an exception or uses an explicit
%% |return| statement.  This is comparable to |synchronized{ comp }|.
%% \end{slide}

%% %%%%%

%% \begin{slide}
%% \heading{Alternative: explicit locks}

%% Other languages use similar mechanisms.  For example,
%% C++ |mutex|s act in the same way.
%% \end{slide}

\end{document}
