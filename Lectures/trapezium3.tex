
\begin{slide}
\heading{Equal shares?}

In the previous chapter, we chose to distribute the work equally between the
workers.

This is sensible if each worker will be executed on its own processor,
all the processors are the same speed, all are equally loaded with other
tasks, and it's possible to identify what an equal distribution is.

But what if this isn't the case?
\end{slide}

%%%%%

\begin{slide}
\heading{Smaller tasks}

An alternative is to use smaller tasks, with more tasks than workers.  

In the numerical integration example, each task will be to estimate the
integral over some subinterval.   We use \SCALA{nTasks} tasks, where
\SCALA{nTasks > nWorkers}.

The tasks are distributed between the workers, each receiving a new
task when it has finished the previous one.  In this way, the faster workers
will complete more tasks than the slower ones --- a form of load balancing.

This pattern is known as \emph{bag of tasks}: the controller holds a bag of
tasks, which it distributes to workers.  (Sometimes workers return sub-tasks
to the controller, but that's not the case here.)
\end{slide}

%%%%%

%% \begin{slide}
%% \heading{Signalling the end}

%% We need some way for the controller to signal to each worker that there are no
%% more tasks to complete, so each worker can terminate.
%% \begin{itemize}
%% \item
%% The controller could send a special value, like |null|, if it can be guaranteed
%% that no proper task will take that value.

%% \item
%% We could use an |Option[Task]| type, sending a value |Some(t)| to indicate a
%% proper task~|t|, or the value~|None| to indicate there are no more tasks.  But
%% this adds an overhead of packing and unpacking |Some| values. 

%% \item The paradigmatic CSO way of indicating that a stream of data is finished
%%   is to close the channel, preventing further communications.
%% \end{itemize} 
%% \end{slide}


%%%%%

\begin{slide}
\heading{A worker}

The controller will close the |toWorkers| channel to indicate that there are
no remaining tasks.

Each  worker process will repeatedly receive and process tasks, until the
|toWorkers|  channels is closed.  
%
\begin{scala}
  /** A worker, which repeatedly receives arguments from the distributor,
    * estimates the integral, and sends the result to the collector. */
  private def worker = thread("worker"){
    repeat{
      val (left, right, taskSize, delta) = toWorkers?() 
      val result = integral(left, right, taskSize, delta)
      toController!result
    }
  }
\end{scala}
\end{slide}

%%%%%

\begin{slide}
\heading{The controller}

The controller has to both distribute tasks (on |toWorkers|), and receive
results (on |toController|).

Implementing this using a single controller thread is tricky, because that
thread needs to deal with communications on two channels. 

But the issues of distributing the tasks and receiving the results are
independent.  So let's split them between two parallel processes!  

Also, this might also give better performance, if the controller acts as a
bottleneck.
\end{slide}

%%%%%

\begin{slide}
\heading{Distributing the tasks}

We again assume here that |n%nTasks == 0|. 
\begin{scala}
  /** A distributor, who distributes tasks to the clients. */
  private def distributor = thread("distributor"){
    require(n%nTasks == 0)
    val delta = (b-a)/n               // size of each interval
    val taskSize = n/nTasks          // number of points per task
    val taskRange = (b-a)/nTasks // range of each task
    var left = a                      // left hand boundary of next task
    for(i <- 0 until nTasks){
      val right = left+taskRange
      toWorkers!(left, right, taskSize, delta)
      left = right
    }
    toWorkers.endOfStream
  }
\end{scala}
\end{slide}

%%%%%

\begin{slide}
\heading{Collecting the results}

\begin{scala}
  /** This variable ends up holding the result. */
  private var result = 0.0

  /** A collector, that accumulates the sub-results into result. */
  private def collector = thread("collector"){
    result = 0.0
    for(i <- 0 until nTasks) result += (toController?())
  }
\end{scala}
\end{slide}

%%%%%

\begin{slide}
\heading{The complete system}

\begin{scala}
  /** The main system. */
  private def system = {
    val workers = || (for (i <- 0 until nWorkers) yield worker)
    workers || distributor || collector
  }

  def apply: Double = { run(system); result } 
\end{scala}
%
%% (The version on the website creates a new |toWorkers| channel for each run.)

\heading{Testing}

We can test this in the same way as in the previous chapter.
\end{slide}
