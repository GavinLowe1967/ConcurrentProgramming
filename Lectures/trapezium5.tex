
\begin{slide}
\heading{Encapsulation}

In the previous code, the design decisions concerning the use of
message-passing concurrency was interleaved with the computation.  If
we decided to implement the concurrency in a different way ---for
example, to use one of the techniques we'll see later in the course---
we would have to make changes in several places.  This is a very small
program, so it doesn't matter too much here; but it would be an issue
in bigger programs.

It would be better to encapsulate each relevant design decision within a
single object, so that changes will involve just that object.  Each such
object will be designed to allow concurrent calls of its methods, and to avoid
race conditions.  In particular, reasoning about correctness will (mostly)
involve reasoning about a single object at a time. 
\end{slide}

%%%%%

\begin{slide}
\heading{Encapsulation}

We will define concurrent objects with the following interfaces.
%
\begin{scala}
  /** The bag of tasks object. */
  private class BagOfTasks{
    /** Get a task.  
      * @throws Stopped exception if there are no more tasks. */
    def getTask: Task = ...
  }  

  /** A collector object that receives sub-results from the workers, and adds
    * them up. */
  private class Collector{
    /** Add x to the result. */
    def add(x: Double) = ...

    /** Get the result, once the computation has finished. */
    def get: Double = ...
  }
\end{scala}
\end{slide}

%%%%%

\begin{slide}
\heading{Encapsulation}

Then we can define the worker as follows. 
%
\begin{scala}
  private def worker(bag: BagOfTasks, collector: Collector) = thread{
    repeat{
      val (left, right, taskSize, delta) = bag.getTask
      val result = integral(left, right, taskSize, delta)
      collector.add(result)
    }
  }
\end{scala}
%
Note that this is independent of how the |BagOfTasks| and |Collector| are
implemented. 
\end{slide}

%%%%%

\begin{slide}
\heading{Encapsulating the bag of tasks}

\begin{scala}
  /** The bag of tasks object. */
  private class BagOfTasks(buffering: Int){
    /** Channel from the controller to the workers, to distribute tasks. */
    private val toWorkers = 
      if(buffering > 0) new BuffChan[Task](buffering) else new SyncChan[Task]

    /** Get a task.  
      * @throws Stopped exception if there are no more tasks. */
    def getTask: Task = toWorkers?()
    ...
  }
\end{scala}
\end{slide}  

%%%%%

\begin{slide}
\heading{Encapsulating the bag of tasks}

\begin{scala}
  private class BagOfTasks{
    ...
    /** A server process, that distributes tasks. */
    private def server = thread{
      require(n%nTasks == 0); val delta = (b-a)/n; val taskSize = n/nTasks
      val taskRange = (b-a)/nTasks; var left = a                  
      for(i <- 0 until nTasks){
        val right = left+taskRange
        toWorkers!(left, right, taskSize, delta)
        left = right
      }
      toWorkers.close
    }

    // Start the server running
    fork(server)
  }
\end{scala}
\end{slide}  

%%%%%

\begin{slide}
\heading{\scalashape fork}

If |t| is a |ThreadGroup|, then |fork(t)| or |t.fork| starts a new thread
running which executes~|t|.

Note that this is in parallel to the current thread.  On the previous slide,
the construction of the |BagOfTasks| returns after the |fork(server)|, but
leaves a thread running the |server| code. 
\end{slide}

%%%%%


\begin{slide}
\heading{Encapsulating the collector}

\begin{scala}
  /** A collector object that receives sub-results from the workers, and adds
    * them up. */
  private class Collector(buffering: Int){
    /** Channel from the workers to the controller, to return sub-results. */
    private val toController = 
      if(buffering > 0) new BuffChan[Double](buffering) else new SyncChan[Double]

    /** Channel that sends the final result. */
    private val resultChan = new SyncChan[Double]

    /** Add x to the result. */
    def add(x: Double) = toController!x

    /** Get the result. */
    def get: Double = resultChan?()
    ...
  }
\end{scala}
\end{slide}

%%%%%

\begin{slide}
\heading{Encapsulating the collector}

\begin{scala}
  private class Collector{
    ...
    /** A collector, that accumulates the sub-results. */
    private def server = thread{
      var result = 0.0
      for(i <- 0 until nTasks) result += (toController?())
      resultChan!result
    }

    // Start the server running
    fork(server)
  }
\end{scala}
\end{slide}

%%%%%

\begin{slide}
\heading{The main function}

\begin{scala}
  /** Calculate the integral. */
  def apply: Double = {
    val bag = new BagOfTasks(buffering); val collector = new Collector(buffering)
    val workers = || (for (i <- 0 until nWorkers) yield worker(bag, collector))
    run(workers)
    collector.get
  }
\end{scala}
\end{slide}  

%%%%%

\begin{slide}
\heading{An incorrect design}

A previous design stored the result in an object variable, and arranged for
|get| to return that variable.
%
\begin{scala}
  private class Collector{
    private var result = 0.0 // Holds final result

    /** A collector, that accumulates the sub-results into result. */
    private def server = thread{
      for(i <- 0 until nTasks) result += (toController?())
    }

    /** Get the result.  Pre: the computation has finished. */
    def get: Double = result
  }
\end{scala}
%
This can go wrong: the |run(workers)| in the |apply| function can terminate
while the |server| is still processing the final sub-result; thus the value
returned by |apply| can be missing this final sub-result.
% Why is this incorrect?
    %% private var done = false // Set to true when computation finished
    %%   done = true{ require(done); result }
\end{slide}

%%%%%

\begin{selfnote}
The main function calls |collector.get| after the workers have finished; but
the server inside the collector might not have finished.

Also, there's a potential cache consistency problem.  The collector might have
updated |result| only in its cache, and this might not yet have been written
back to main memory.

Note how using an explicit assertion detects the first point, and probably
catches the second point, making it easier to detect and correct the error.
\end{selfnote}



% %%%%%

% \begin{selfnote}
% More tasks gives better load balancing, but also gives extra overhead from the
% extra communications.

% In a distributed system, the communication overhead becomes more significant.
% \end{selfnote}

%%%%%

\begin{slide}
\heading{Bag of tasks with replacement}

A variant on the bag of tasks pattern allows new tasks to be placed back into
the bag. 

\emph{Adaptive quadrature} is an alternative approach that proceeds as
follows.  In order to calculate the integral $\int_a^b f(x) \mbox{d}x$,
compute the midpoint $mid = (a+b)/2$ and estimate three integrals, from $a$ to
$mid$, from $mid$ to~$b$, and from $a$ to~$b$, each using the trapezium rule
with a single interval.  If the sum of the former two estimates is within some
value $\epsilon$ of the third, then we take that third estimate as being the
result.  Otherwise, recursively estimate the integrals over the two
sub-intervals $a$ to~$mid$ and $mid$ to~$b$, and sum the results.

The latter case can be implemented by returning the two sub-intervals $(a,
mid)$ and $(mid, b)$ to the bag.

Exercise: implement this.
\end{slide}

%%%%%

\begin{slide}
\heading{Summary}

\begin{itemize}
%\item
%% Data parallel;

%% \item
%% Workers and controllers;

%% %% \item
%% %% Closing channels;

%% \item
%% Testing against a sequential implementation;

%% \item
%% Choosing the number of processes;

%% \item
%% Experimental design;

\item
Bag of tasks;

\item
Choosing the size of tasks;

\item
Communication is expensive: avoid having too many communications;

\item
Encapsulation;

\item Bag of tasks with replacement.
\end{itemize}
\end{slide}
