\begin{nontutequestion}
\Programming\ 
The \emph{readers and writers problem} is a classic synchronisation problem.
Consider a collection of processes, each of which want to access a database
concurrently, some for reading and some for writing.  It is allowable for
several readers to be accessing the database simultaneously, since their
transactions will be independent.  However, it is not allowable for two
writers to have simultaneous access, nor for a reader and writer to have
simultaneous access.

Implement a solution to the readers and writers problem using a server
process: each reader and writer should communicate with the server when
entering or leaving.

Is a reader or writer who is trying to gain entry to the database guaranteed
to succeed, or could it be locked out forever?  If the latter, how could this
be avoided? 
\end{nontutequestion}

%%%%%

\begin{nontuteanswer}
\Small
The server needs to keep track of the numbers of readers and writers currently
accessing the database, and maintain the invariant
\[
(readers=0 \land writers \le 1) \lor writers=0
\]

Avoiding starvation is tricky, since, say readers may always gain access,
locking the writers out for ever.  Below, priority is given to processes
wishing to leave, and when $readers=0 \land writers=0$, it alternates between
giving priority to a reader or a writer (according to the flag
\SCALA{writerPri}).    
%
\begin{scala}[showstringspaces=false]
// The readers and writers problem
import ox.CSO._

object ReadersWriters{
  val random = new scala.util.Random;

  // A reader, which just repeatedly enters and leaves
  def Reader(me: Int, enter: ![Int], leave: ![Int]) = proc{
    repeat{
      sleep(random.nextInt(2000)); enter!me;
      sleep(random.nextInt(1000)); leave!me
    }
  }

  // A reader, which does likewise
  def Writer(me: Int, enter: ![Int], leave: ![Int])= proc{
    repeat{
      sleep(random.nextInt(2000)); enter!me;
      sleep(random.nextInt(1000)); leave!me
    }
  }

  // A server which controls the entering and leaving, 
  // maintaining the invariant: (readers==0 && writers<=1) || writers==0
  // Priority is always given to leaving.  When readers==0 and writers==0, 
  // priority is given to a writer to enter next iff writerPri
  def Server(readerEnter: ?[Int], readerLeave: ?[Int],
	     writerEnter: ?[Int], writerLeave: ?[Int])
  = proc{
    var readers = 0; var writers = 0;
    def Report = { println("readers="+readers+"; writers="+writers); }
    var writerPri = true; // Does writer get priority?

    priserve(
      (readers>0 &&& readerLeave) --> {
	val id = readerLeave?; readers-=1; println("Reader "+id+" leaves"); Report
      }
      | 
      (writers>0 &&& writerLeave) --> {
	val id = writerLeave?; writers-=1; println("Writer "+id+" leaves"); Report
      }
      | 
      (writerPri && readers==0 && writers==0 &&& writerEnter) --> {
	val id = writerEnter?; writers+=1; writerPri=false;
	println("Writer "+id+" enters"); Report
      }
      |
      (writers==0 &&& readerEnter) --> {
	val id = readerEnter?; readers+=1; writerPri=true;
	println("Reader "+id+" enters"); Report
      }
      | 
      (!writerPri && readers==0 && writers==0 &&& writerEnter) --> {
	val id = writerEnter?; writers+=1; writerPri=false;
	println("Writer "+id+" enters"); Report
      }
    )
  }

  // Number of readers and writers:
  val readers = 5; val writers = 5

  // Declare the channels
  val readerEnter, readerLeave, writerEnter, writerLeave = ManyOne[Int];  

  def Readers = 
    || ( for(i <- 0 until readers) yield Reader(i, readerEnter, readerLeave) )
  def Writers =
    || ( for(i <- 0 until readers) yield Writer(i, writerEnter, writerLeave) )
  def System = 
    Readers || Writers || Server(readerEnter,readerLeave,writerEnter,writerLeave)

  def main(args : Array[String]) = { System() }
}
\end{scala}

This does not guarantee starvation freedom, since the readers may try to enter
sufficiently often that $readers$ never reaches zero, so no writer can gain
entry.  This can be avoided by having writers register their interest (by
another communication, that the server always accepts), and then not allowing
any more \SCALA{readerEnter} events before a \SCALA{writerEnter} event.
\end{nontuteanswer}

