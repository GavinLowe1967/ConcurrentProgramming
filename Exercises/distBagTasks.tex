\begin{question}
In the bag-of-tasks pattern, there is normally a single process that holds the
outstanding tasks.  However, that process can then become a bottle-neck.  The
aim of this question is to investigate an alternative, where each node holds
its own bag of tasks, but there is no central controller; i.e.,\ the bag of
tasks is distributed between the nodes.

One way to implement a distributed bag of tasks involves arranging the nodes
in a (logical) ring.  Each node can work on tasks from its bag, or pass tasks
to its clockwise neighbour.  Each node should accept a task from its
anti-clockwise neighbour only when it has no tasks in its own bag.

You should implement a system using the distributed-bag-of-tasks pattern to
solve some combinatorial puzzle.  For example, you could pick the \emph{Eight
  Queens Problem} (see \url{http://en.wikipedia.org/wiki/Eight_queens}).
However, you should not pick the \emph{Magic Squares} problem that was
considered in lectures.  (The main point of this question is the
distributed-bag-of-tasks pattern, so you should pick a puzzle with simple
rules; also avoid a puzzle that will take too long to solve.)  If you pick a
puzzle other than the Eight Queens Problem, please include a brief description
of the rules (this part of your report may be copied from a source, with
citation).

One issue you will need to address is termination: the system should terminate
when the puzzle if fully solved.  (This part is quite tricky; however, you can
still obtain a good mark if you do not achieve this.)

Your answer should include a description of your design and an explanation of
any interesting features.  In particular, you should explain the termination
protocol.

\marks{40}
\end{question}

%%%%%


\begin{answer}
My code is below.

The class \SCALA{Partial} represents a partial solution to the puzzle.  See
comments.

Each node is comprised of worker and bag processes.  The worker simply removes
a task from the local bag; if it is complete it prints it; otherwise it
generates a list of successor partial solutions which it returns to its bag,
and then sends a \SCALA{done} message to its bag.

When the bag is non-empty, it is willing to pass a partial solution to its own
worker, receive back new partial solutions, and then receive a \SCALA{done}
message.  Alternatively, it is willing to pass partial solutions to its
neighbour.  When the worker is busy, priority is given to this to speed up
distribution [3~marks for some consideration of priorities].

When the bag is empty, it is willing to receive a partial solution from its
neighbour, or take part in the termination protocol.

The termination protocol is initiated by node~0.  It sends a
\SCALA{Terminating} token round the ring, on the same channels as for the
passing of partial solutions.  Each subsequent node receives the token only
when it is idle, and passes it on.  If node~0 receives the token back and it
has been idle ever since starting the termination protocol, then the whole
system can terminate: it circulates the token again, with a flag set, to
indicate this.  If it becomes busy after initiating the termination protocol,
it has to wait until it receives the token back before it re-tries the
termination protocol. [15 marks for the termination protocol]

[Notes for examiners: The lectures included an example of solving a puzzle
  using a non-distributed bag of tasks.  A termination protocol like this was
  discussed in a class exercise, although without full details.]

\JavaSize{\footnotesize}
\begin{scala}
// The 8 queens problem

import ox.CSO._;

abstract class Msg;

// Class to represent a partial solution

case class Partial(N:Int) extends Msg{
  // Represent partial solution by a list of Ints; the ith 
  // entry represents the row number of the queen in column i 
  // (0 <= i < len).  len is the length of board
  private var board : List[Int] = Nil;
  private var len = 0;

  def finished : Boolean = (len==N);

  // Is it legal to play in column len, row j?
  private def isLegal(j : Int) = {
    // is piece (i1,j1) on different diagonal from (len,j)?
    def otherDiag(p:(Int,Int)) = { 
      val i1=p._1; val j1 = p._2; i1-len!=j1-j && i1-len!=j-j1; 
    };
    
    (board forall ((j1:Int) => j1 != j)) && 
      // row j not already used
    (List.range(0,len) zip board forall otherDiag) 
      // diagonals not used
  }

  // Return new partial resulting from playing in row j next
  private def doMove(j : Int) : Partial = {
    val newPartial = new Partial(N);
    newPartial.board = this.board ::: (j :: Nil);
    newPartial.len = this.len+1;
    return newPartial;
  }

  // Return list of successors, such that every solution to 
  // this is a solution of one of the successors
  def successors : List[Partial] = 
    for(j <- List.range(0, N) if(isLegal(j))) yield doMove(j);

  override def toString : String = {
    var st = "";
    for(i <- 0 until len) st = st + (i,board(i))+"\t";
    return st;
  }
}

// Class of messages used in testing for termination

case class Terminating(status:Boolean) extends Msg{ }
// Status of false means this is a probe if we can terminate; 
// status of true means terminate now

// The main object
object EightQueens{
  val N = 8; // size of board

  // A worker.  This node can get a partial solution from its 
  // anticlockwise neighbour on channel get; it can give a
  // partial solution to its clockwise neighbour on channel give
  def Node(me: Int, get: ?[Msg], give: ![Msg])
  = proc("Node"+me){

    // Process to maintain this bag of tasks
    def Bag(toWorker: ![Partial], fromWorker: ?[Partial], 
            done: ?[Unit]) 
    = proc("Bag"+me){
      // Store the tasks in a stack
      val stack = new scala.collection.mutable.Stack[Partial];
      if(me==0)	stack.push(new Partial(N)); // Starting position

      var workerBusy = false; // Is the local worker busy?
      var finished = false; // Becomes true when we can terminate

      // Status of termination protocol, node 0 only
      val NONTERM = 0; val TRYING = 1; val ABORTED = 2;
      var terminationStatus = NONTERM; 
      // NONTERM means not currently involved in termination 
      // protocol; TRYING means termination protocol started, 
      // and this node has been idle since then; ABORTED means 
      // termination protocol was started, and this node then 
      // became busy, but the protocol hasn't yet terminated. 
 
      // Main loop
      while(!finished){
	if(!workerBusy)
	  if(!stack.isEmpty){ // Send task to worker
	    val p = stack.pop; 
	    prialt(
	      toWorker -!-> { toWorker!p; workerBusy = true; }
	      | give -!-> { give!p; }
	    )
	  } 
	  else{ // Idle, so wait for message from neighbour
	    if(me==0 && terminationStatus==NONTERM){
	      // Initiate termination protocol
	      println("Termination protocol starting");
	      give!Terminating(false);
	      terminationStatus = TRYING;
	    }
	    val m:Msg = get?;
	    m match {
	      case Partial(n) => {
		toWorker!m.asInstanceOf[Partial]; 
		workerBusy = true;
		if(me==0 && terminationStatus==TRYING){
		  println("Termination protocol aborted");
		  terminationStatus = ABORTED;
		}
	      }
	      case Terminating(s) => {
		if(me==0){
		  if(terminationStatus==TRYING){
		    // We've received nothing since sending the 
		    // termination probe, so we can terminate
		    give!Terminating(true); 
                      // pass signal to next node
		    toWorker.close; // Tell worker to close
		    get?; // Receive termination signal back
		    finished = true; // Can terminate
		  }
		  else if(terminationStatus==ABORTED){
		    terminationStatus = NONTERM;
		    println("Ready to retry termination protocol");
		  }
		}
		else{ // me!=0 
		  give!m; // pass signal on
		  if(s){ finished = true; toWorker.close; } 
                    // terminating
		}
	      } // end of case Terminating(s)
	    } // end of match
	  }
	else // workerBusy
	  prialt(
	    (!stack.isEmpty &&& give) -!-> { give!(stack.pop); }
	    | fromWorker -?-> { 
                val p = fromWorker?; stack.push(p); 
              }
	    | done -?-> { done?; workerBusy = false; }
	  )
      }
    }

    // Process to work on partial solutions
    def Worker(toBag: ![Partial], fromBag: ?[Partial], 
               done: ![Unit])
    = proc("Worker"+me){
      repeat{
	val partial = fromBag? ; // get job
	if(partial.finished){ println(partial); } // done!
	else // Generate all next-states
	  for(p1 <- partial.successors) toBag!p1; 
	// sleep(10); 
	done!();
      }
    }

    // Put this node together
    val bagToWorker, workerToBag = OneOne[Partial];
    val done = OneOne[Unit];
    def ThisNode = (
      Bag(bagToWorker, workerToBag, done) || 
      Worker(workerToBag, bagToWorker, done)
    );
    ThisNode();
  }

  val p = 8; // Number of workers

  // Put system together

  val passPartial = OneOne[Msg](p); // indexed by recipient's id

  def System = 
    || ( for(i <- 0 until p) yield 
           Node(i, passPartial(i), passPartial((i+1)%p)) )

  def main(args: Array[String]) = System();

}
\end{scala}
\end{answer}
