\begin{question}
\label{Q:barrierLog}
Provide an implementation of barrier synchronisation that operates in
$\Theta(\log n)$ time, for $n$ processes.  If you like, each process may pass
its own identity to the \SCALA{sync} function.  Hint: base your implementation
on one of the patterns from the ``Interacting Peers'' chapter.  Test your code
by adapting the test framework on the course website. 
\end{question}

%%%%%

\begin{answer}
This implementation is based on the barrier tree synchronisation algorithm in
Section 3.4.2 of Andrews's book.  It is similar to one of the patterns
students saw in the ``Interacting Peers'' chapter.

The processes are arranged in a binary heap.  Each waits until its children
are ready, signals to its parent that all of its subtree is ready, waits for a
``go'' signal from its parent, and passes that signal on to its children.
%
\begin{scala}
class CombiningTreeBarrier(n: Int){
  /** Channels by which a process signals to its parent that it is ready.
    * Indexed by the child's identity. */ 
  private val ready = Array.fill(n)(OneOne[Unit])

  /** Channels by which a process signals to its children that it can
    * continue.  Indexed by the child's (receiver's) identity. */ 
  private val go = Array.fill(n)(OneOne[Unit])

  /** Barrier protocol for node with identity me, in a system of n nodes. */
  def sync(me: Int) = {
    val child1 = 2*me+1; val child2 = 2*me+2
    // Wait for ready signals from both children
    if(child1 < N) ready(child1)?()
    if(child2 < N) ready(child2)?()
    // Send ready signal to parent, and wait for go reply, 
    // unless this is the root
    if(me != 0){ ready(me)!(); go(me)? }
    // Send go signals to children
    if(child1 < N) go(child1)!()
    if(child2 < N) go(child2)!()
  }
}
\end{scala}

Possible topic for discussion: how could we turn this into a combining
barrier?  Answer: arrange for values to be passed up on the |ready| channels;
each thread combines its value with the values from its children and passes
the result up; the final value is passed back down on the |go| channels; note
that the values are combined following the shape of the heap, but the
combining function is assumed associative and commutative, so this doesn't
matter.
\end{answer}
