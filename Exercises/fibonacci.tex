\begin{question}
\Programming\ 
Recall the Fibonacci numbers, which can be defined by 
%
\begin{scala}
  def fib(n:Int) : Int = if(n<=1) n else fib(n-1)+fib(n-2)
\end{scala}
%
Implement this using recursive parallelism.  Consider versions with no limit
on the degree of recursion, and with some set limit.  Carry out some informal
timing tests on the two versions and on the sequential version above.
\end{question}

%%%%%

\begin{answer}
I'll just give the version with a limit on parallelism (omit the parameter
\SCALA{maxProcs} to get the version with no limit).
%
\begin{scala}
import ox.CSO._

object Fibonacci2{

  // Standard recursive definition
  def fib(n:Int) : Int = if(n<=1) n else fib(n-1)+fib(n-2)

  // Process to calculate fib n and send it on the out channel,
  // using at most maxProcs processes
  def Fib(n:Int, out: ![Int], maxProcs:Int) : PROC = proc{
    if(n<=1) out!n;
    else if(maxProcs<=1) out! fib(n);
    else{
      val res1, res2 = OneOne[Int];
      val k = maxProcs/2;
      ( proc{ out!((res1?)+(res2?)); out.close} || 
        Fib(n-1, res1, k) || Fib(n-2, res2, maxProcs-k)
      )()
    }
  }

  def System(n:Int) = {
    val out = OneOne[Int];
    Fib(n, out, 8) || ox.cso.Components.console(out);
  }

  def main(args:Array[String]) = {
    val n = Integer.valueOf(args(0)).intValue();
    val t0 = java.lang.System.currentTimeMillis()
    println(fib(n));
    println("Time taken: "+(java.lang.System.currentTimeMillis()-t0));
    val t1 = java.lang.System.currentTimeMillis()
    System(n)();
    println("Time taken: "+(java.lang.System.currentTimeMillis()-t1));
  }
}
\end{scala}

My experimental results are as follows, on an 8 processor machine.  The
version with no limit on parallelism struggles, taking about 10 seconds for $n
= 20$, compared with less than 20ms for the other versions.  For $n = 45$, the
parallel version with limited recursion is about 4 times faster than the
sequential version.
\end{answer}

