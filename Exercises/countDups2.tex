\begin{question}
\Programming\
Suppose you are given two arrays, as follows:
%
\begin{scala}
  val N = ...;
  val a = new Array[Int](N);
  val b = new Array[Int](N);
\end{scala}
%
Neither array contains any repetitions.  Write a concurrent program to count
the number of distinct values that appear in both arrays.  Briefly explain
your design.  \textbf{Hint:} use a hash table
(\url{scala.collection.mutable.HashSet}).   
\end{question}

%%%%%

\begin{answer}
We split \SCALA{a} into \SCALA{aSegs} segments, and \SCALA{b} into
\SCALA{bSegs} segments.  Each of the \SCALA{numWorkers = aSegs*bSegs} workers
wil work on a segment of \SCALA{a} and a segment of~\SCALA{b}, and count the
number of values that appear in both segments, and send the result to the
controller.  The elements of the \SCALA{b} segment are put into a hash table
to allow fast checking of membership.
%
\begin{scala}
// Given arrays a, b, with no repetitions, count the 
// number of distinct elements in both a and b.

import ox.CSO._;

object CountDups{
  val N = 500000; // size of arrays
  val a = new Array[Int](N);
  val b = new Array[Int](N);

  // Each worker will work on a segment of a and 
  // a segment of b
  val aSegs = 1; // number of segments of a
  val aSegSize = N/aSegs
  val bSegs = 8; // number of segments of b
  val bSegSize = N/bSegs
  val numWorkers = aSegs*bSegs; 

  // A single worker
  def Worker(me: Int, toController: ![Int]) = proc{
    // This worker will deal with a[aStart..aEnd) 
    // and b[bStart..bEnd) 
    def aStart = (me/bSegs) * aSegSize;
    def aEnd = (me/bSegs + 1) * aSegSize;
    def bStart = (me%bSegs) * bSegSize;
    def bEnd = (me%bSegs + 1) * bSegSize;

    // Put all b elements in hash table
    // val bHash = new java.util.HashSet[Int](bSegSize*2);
    // for(i <- bStart until bEnd) bHash.add(b(i));
    val bHash = new scala.collection.mutable.HashSet[Int];
    for(i <- bStart until bEnd) bHash += b(i);

    // Now iterate through segment of a, counting how 
    // many entries are in bHash
    var count = 0;
    for(i <- aStart until aEnd)
      if(bHash.contains(a(i))) count+=1;

    // Send result to controller
    toController!count;
  }

  // The controller
  def Controller(toController: ?[Int]){
    var count = 0;
    for(i <- 0 until numWorkers) count += (toController?);
    println(count);
  }

  // Initialise arrays
  def Init{
    // We set a(i)=i, b(i)=2*i;
    for(i <- 0 until N){ a(i)=i; b(i)=2*i; }
  }

  // Construct system
  val toController = ManyOne[Int];
  def Workers =
    || ( for(i <- 0 until numWorkers) yield 
           Worker(i, toController) )
  def System = Controller(toController) || Workers

  def main(args:Array[String]) = {
    Init; 
    val t0 = java.lang.System.currentTimeMillis();
    System();
    println("Time taken: "+
      (java.lang.System.currentTimeMillis()-t0)/1000.0);
  }
}
\end{scala}

My experiments suggest that this is fastest (on an 8 processor machine) when
we take \SCALA{aSegs = 1} and \SCALA{bSegs = 8}.

Curiously, the Scala hashtable seems considerably faster than the Java one. 
\end{answer}
