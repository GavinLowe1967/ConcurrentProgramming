\begin{nontutequestion}
Recall the following restriction on the use of \SCALA{alt}:
%
\begin{quote}
If the output port of a channel participates in an \SCALA{alt} then its input
end must not simultaneously participate in an(other) \SCALA{alt}.
\end{quote}
%
Suggest a reason for this restriction.
\end{nontutequestion}

%%%%%

\begin{nontuteanswer}
The restriction makes implementation of \SCALA{alt} much easier.  Consider a
program such as \SCALA{P || Q} where:
%
\begin{scala}
def P = proc{ alt( c -!-> ... | a -?-> ...) }
def Q = proc{ alt( c -?-> ... | b -?-> ...) }
\end{scala}
%
The two \SCALA{alt}s are executed by different threads.  There is a danger of
the \SCALA{alt} in \SCALA{P} selecting channel~\SCALA{c} at the same time that
the \SCALA{alt} in \SCALA{Q} selects channel~\SCALA{b}, for example; the two
threads would then evolve in ways inconsistent with the intended semantics.    

Solving these problems would require some non-trivial synchronisation between
the two threads.

This problem potentially becomes harder when we have more than two processes.

Very similar examples can be used to show why the input port of a OneMany
channel may not simultaneously participate in more than one \SCALA{alt}. 
\end{nontuteanswer}

%% Consider a ring of processes:
%% %
%% \begin{scala}
%% || ( for(i <- 0 until N) yield P(i) )
%% \end{scala}
%% %
%% where
%% %
%% \begin{scala}
%% def P(me) = proc{ alt( c((me+1)\%N) -!-> ... | c(me) -?-> ...) }
%% \end{scala}
