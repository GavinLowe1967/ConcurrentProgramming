\begin{question}
\Programming
\begin{enumerate}
\item
%\label{part:a}
Suppose $n^2$ worker processes are arranged in a $n$ by $n$ toroidal grid.
Each process starts with an integer value $x$.  The aim is for each process to
end up in possession of the maximum of these $n^2$ values.  Each process may
send messages to the processes above it and to the right of it in the grid
(where we treat the bottom row as being above the top row, and the left hand
column as being to the right of the right hand column).

Write code solving this problem, making use of a worker process:
%
\begin{scala}
def worker(i: Int, j: Int, x: Int, readUp: ?[Int], writeUp: ![Int], 
	   readRight: ?[Int], writeRight: ![Int]) = proc{ ... }
\end{scala}
%
Give your code the following signature, where |xss| gives the values initially
held by the processes:
%
\begin{scala}
class GridMax(n: Int, xss: Array[Array[Int]]){
  require(n >= 1 && xss.length == n && xss.forall(_.length == n))

  /** Run the system, and return array storing results obtained. */
  def apply(): Array[Array[Int]] = ...
}
\end{scala}
%
Test your code using the test harness on the course website.


%%%%%

%% \item
%% Explain what is means by a \emph{barrier synchronisation}.  

%% Show how to implement a barrier synchronisation between $M$ processes so that
%% each synchronisation takes $\Theta(\log M)$ steps.      %        \marks{9}

%%%%%

\item Now consider the same scenario as in the previous part, except assume
each process can send messages to any other processes.  We now want a solution
that takes $O(\log n)$ rounds.

Write code to solve this problem, making use of a worker process:
%
\begin{scala}
def worker(i: Int, j: Int, x: Int, read: ?[Int], write: List[List[![Int]]]) = proc{ ... }
\end{scala}
%
The process can send values to another process \SCALA{(i1, j1)} on the channel
\SCALA{write(i1)(j1)}, and can receive messages from other processes on the
channel~\SCALA{read}.  You should briefly explain your solution.  Again, test
your code using the test harness provided.

What is the total running time of the program (in $\Theta(\_)$ notation),
assuming as many processors as processes?

%% \item Write a test harness for your two programs. 
\end{enumerate}
\end{question}

%%%%%%%%%%%%%%%%%%%%%%%%%%%%%%%%%%%%%%%%%%%%%%%%%%%%%%%

\begin{answer}
%\Footnotesize
\begin{enumerate}
\item
We'll treat the channels as being buffered (alternatively, arrange for the
reads and writes, below, to be concurrent).
%
\begin{scala}
class GridMax(n: Int, xss: Array[Array[Int]]){
  require(n >= 1 && xss.length == n && xss.forall(_.length == n))

  /** Array to hold results. */
  private val results = Array.ofDim[Int](n,n)

  /** Worker with coordinates (i,j) and starting value x, with channels to allow
    * data to be passed upwards and rightwards. */
  private def worker(i: Int, j: Int, x: Int, readUp: ?[Int], writeUp: ![Int], 
	             readRight: ?[Int], writeRight: ![Int])
    = proc("worker"+(i,j)){
    var myMax = x
    // propogate values along rows
    for(i <- 0 until n-1){
      writeRight!myMax; myMax = myMax max readRight?()
    }
    // propogate values upwards
    for(i <- 0 until n-1){
      writeUp!myMax; myMax = myMax max readUp?()
    }
    // Store result
    results(i)(j) = myMax
  }

  /** Channels by which values are passed rightwards; indexed by coords of
    * recipients. */
  private val right = Array.fill(n,n)(OneOneBuf[Int](1))

  /** Channels by which values are passed upwards; indexed by coords of
    * recipients. */
  private val up = Array.fill(n,n)(OneOneBuf[Int](1))

  /** Run the system, and return array storing results obtained. */
  def apply(): Array[Array[Int]] = {
    val workers = 
      || (for(i <- 0 until n; j <- 0 until n) yield
            worker(i, j, xss(i)(j), up(i)(j), up(i)((j+1)%n), right(i)(j), right((i+1)%n)(j)))
    run(workers)
    results
  }
}
\end{scala}

%%%%%

%%%%%

\item
We perform the computation in two phases.  In the first phase, each worker
obtains the maximum of the values in its row, in $\lceil \log n \rceil$
rounds, using a variant of the prefix-sums example (this pattern works with
any associative and commutative operator).  In the second phase, the values
from the first phase are propagated up the columns, again using a variant of
the prefix-sums example, so each worker obtains the overall maximum in an
additional $\lceil \log n \rceil$ rounds.

In the first phase, we use a different barrier object for each row; and in the
second phase, we use a different barrier object for each column.  This has two
advantages over using a single barrier: it allows different rows/columns to
run at different speeds; each synchronisation involves fewer threads, so runs
faster.  However, it is necessary to use different barriers for the two
phases, or else a synchronisation may mix threads from the different phases.
It is also necessary to have a barrier synchronisation of all the threads in
each column before the second phase, or else the channel communications may
mix threads on different phases.  

\begin{scala}
/** This version uses about 2 log n rounds, log n in each dimension. */
class LogGridMax2(n: Int, xss: Array[Array[Int]]){
  require(n >= 1 && xss.length == n && xss.forall(_.length == n))

  /** Array to hold results. */
  private val results = Array.ofDim[Int](n,n)

  /** Barriers for synchronisation in the two rounds. */
  private val rowBarriers = Array.fill(n)(new Barrier(n))
  private val colBarriers = Array.fill(n)(new Barrier(n))
  /* Note: we use a different barrier for each row in phase 1, and a different
   * barrier for each column in phase 2; but we use different barriers for
   * each phase, or else the synchronisations can go wrong, with a particular
   * synchronisation mixing calls to sync from the two phases. */

  /** Worker with coordinates (i,j) and starting value x, a single incoming
    * channel, and channels to send to every other worker. */
  private def worker(i: Int, j: Int, x: Int, read: ?[Int], write: List[List[![Int]]])
    = proc("worker "+(i,j)){
      var myMax = x // max I've seen so far

      // Propagate along rows
      var r = 0; var gap = 1 // gap = 2^r
      // Inv: myMax = max xss(i)[j..j+gap) (looping round)
      while(gap < n){
        // Send to gap places left
        write(i)((j+n-gap)%n)!myMax
        // Receive from worker (i, (j+gap) mod n)
        myMax = myMax max read?() 
        // received value = max xss(i)[(j+gap) mod n .. (j+2gap) mod n)
        r += 1; gap += gap
        if(gap < n) rowBarriers(i).sync // Sync before next round
      }
      // myMax = max xss(i)

      // Wait for all workers in this column to finish phase 1.
      colBarriers(j).sync

      // Propagate up columns
      r = 0; gap = 1
      // Inv: myMax = max xss[i..i+gap) (looping round)
      while(gap < n){
        // Send to gap places down
        write((i+n-gap)%n)(j)!myMax
        // Receive from worker ((i+gap) mod n, j)
        myMax = myMax max read?() 
        // received value = max xss[(i+gap) mod n .. (i+2gap) mod n)
        r += 1; gap += gap
        if(gap < n) colBarriers(j).sync // Sync before next round
      }

      results(i)(j) = myMax
  }

  private val chans =
    List.fill(n,n)(N2NBuf[Int](size = 1, readers = 1, writers = n*n))

  /** Run the system, and return array storing results obtained. */
  def apply(): Array[Array[Int]] = {
    val workers = 
      || (for(i <- 0 until n; j <- 0 until n) yield
            worker(i, j, xss(i)(j), chans(i)(j), chans))
    workers(); results
  }
}
\end{scala}


%% The idea is that on round~$r$, each process holds the maximum of the initial
%% values from a $2^r$ by $2^r$ square (wrapping round in the obvious way), with
%% itself at the bottom-left corner.
%% %
%% \begin{scala}

%% class LogGridMax(n: Int, xss: Array[Array[Int]]){
%%   require(n >= 1 && xss.length == n && xss.forall(_.length == n))

%%   /** Array to hold results. */
%%   private val results = Array.ofDim[Int](n,n)

%%   /** Barrier for synchronisation. */
%%   private val barrier = new Barrier(n*n)

%%   /** Worker with coordinates (i,j) and starting value x, a single incoming
%%     * channel, and channels to send to every other worker. */
%%   private def worker(i: Int, j: Int, x: Int,
%%                      read: ?[Int], write: List[List[![Int]]])
%%     = proc("worker "+(i,j)){
%%     var myMax = x // max I've seen so far
%%     var r = 0 // Round number 
%%     var gap = 1 // 2^r
%%     // Invariant: myMax is the maximum of the values of nodes
%%     // [i..i+gap)[j..j+gap) (interpreted mod n), and gap = 2^r
%%     while(gap < n){
%%       // Send max to processes gap positions left and/or down of here
%%       val i1 = (i+n-gap)%n; val j1 = (j+n-gap)%n
%%       write(i)(j1)!myMax; write(i1)(j1)!myMax; write(i1)(j)!myMax
%%       // Receive from three processes gap right and/or above here
%%       for(i <- 0 until 3) myMax = myMax max read?()
%%       // Update r, max
%%       r += 1; gap += gap
%%       barrier.sync // Sync before next round
%%     }
%%     results(i)(j) = myMax
%%   }

%%   private val chans =
%%     List.fill(n,n)(N2NBuf[Int](size = 3, readers = 1, writers = n*n))

%%   /** Run the system, and return array storing results obtained. */
%%   def apply(): Array[Array[Int]] = {
%%     val workers = 
%%       || (for(i <- 0 until n; j <- 0 until n) yield worker(i, j, xss(i)(j), chans(i)(j), chans))
%%     run(workers)
%%     results
%%   }
%% }
%% \end{scala}

There are about $2 \log n$ rounds.  On each round, each step takes constant
time except the barrier synchronisation, which takes $\Theta(n)$.  So the
total time is $\Theta(n \log n)$.  If we replace the barrier by a logarithmic
version, as in question~\ref{Q:barrierLog}, then each barrier synchronisation
takes time~$\Theta(\log n)$.  So the total time is $\Theta((\log n)^2)$.

Here are a few alternatives.
%
\begin{itemize}
\item
Have a single phase.  The idea is that on round~$r$, each thread holds the
maximum of the initial values from a $2^r$ by $2^r$ square (wrapping round in
the obvious way), with itself at the bottom-left corner.  On round~$r$, each
thread sends its current value to the three threads $2^r$ steps left and/or
down from it, and receives from the three threads $2^r$ steps right and/or up
from it.  All $n^2$ threads have a barrier synchronisation at the end of each
round.  This requires about $\log n$ rounds, on each of which each thread
sends and receives three messages, and has a barrier synchronisation with
$n^2$ threads.  With the logarithmic barrier, this is again $\Theta((\log
n)^2)$, but requires 50\% more channel communications.

%% An alternative is to first find the maximum in each row in about $\log n$
%% rounds; and then to propagate the row maximums along the columns in about
%% $\log n$ rounds.  This takes twice as many rounds as the previous version, so
%% twice as many barrier synchronisations.  But it uses only about $2 \times \log
%% n$ channel communications per worker, compared with $3 \times \log n$ in the
%% previous version. 

\item
Give the worker at position $(i,j)$ an ``identity'' $n \times i + j \in [0
  .. n^2)$.  Note that different workers have different identities.  Then form
  the workers into a heap using these identities, and use the heap pattern
  from lectures.  This takes time $\Theta(\log(n^2)) = \Theta(\log n)$.

\item
Use identities as in the previous version.  Then follow the pattern of the
prefix sum example from lectures.  This takes about $\log(n^2) = 2 \log n$
rounds, and each round takes $\Theta(\log(n^2)) = \Theta(\log n)$, because of
the barrier synchronisation; this makes a total of $\Theta((\log n)^2)$.
\end{itemize}



%%%%%

%% \item I based my testing round the following function.
%% %
%% \begin{scala}
%%   /** Run a single test.
%%     * @param useLog should the logarithmic version be used? */
%%   def doTest(useLog: Boolean) = {
%%     val n = 1+Random.nextInt(10)
%%     val xss = Array.fill[Int](n, n)(Random.nextInt(1000))
%%     val results = if(useLog) new LogGridMax(n, xss)() else new GridMax(n, xss)()
%%     val expected = xss.map(_.max).max
%%     assert(results.forall(_.forall(_ == expected)))
%%   }
%% \end{scala}
\end{enumerate}
\end{answer}
