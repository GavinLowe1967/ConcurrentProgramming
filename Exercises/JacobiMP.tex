\begin{question}
\Programming\
The example of Jacobi iteration from the lectures made the array \SCALA{x} a
shared variable.  Convert the program into a message-passing program, where
each process calculates some of \SCALA{x} and sends it to the others.  The
only variables shared by the processes should be read-only variables.
\end{question}

%%%%%

\begin{answer}
In the solution below, each worker calculates its share of the next iteration
in \SCALA{newX}.  They then exchange these shares, so they all hold the same
value of~\SCALA{x}.

We use buffered channels, indexed by the receivers' identities.  Each
communication includes the portion of \SCALA{x} calculated, the identity of
the worker who produced it (so others know where it should be inserted
in~\SCALA{x}), and the value of the producing worker's \SCALA{done} variable.

Note that a barrier synchronisation is needed at the end of each iteration;
otherwise unfortunate scheduling could lead to a worker receiving in one
iteration values from the following iteration.  This could be avoided by each
worker ensuring that it has received from all the other workers before
continuing. 
%
\begin{scala}
import ox.CSO._

// Jacobi iteration to solve Ax = b

object JacobiMP0{
  val N = 4000; // size of array
  val A = new Array[Array[Double]](N,N);
  val b =  new Array[Double](N)

  val EPSILON = 0.000001; // tolerance
  val p = 8; // number of processes
  assert(N%p == 0);
  val height = N/p; // height of one strip

  // Synchronisation and communication
  val barrier = new Barrier(p)
  // Messages are triples: worker identity, new segment
  // of x, is that worker willing to terminate?
  type Msg = Tuple3[Int, Array[Double], Boolean];
  val toWorker = Buf[Msg](p-1,p)

  // Worker to handle rows start to end-1
  def Worker(me: Int) = proc{
    val start = me*height; val end = (me+1)*height;
    var done = false;
    val x = new Array[Double](N);
    val newX = new Array[Double](height); 
    // newX(i) holds the new value of x(i+start)

    while(!done){
      done = true; 
      // Update this section of x, storing results in newX
      for(i <- start until end){
	var sum:Double=0;
	for(j <- 0 until N) if(j!=i) sum += A(i)(j)*x(j);
	newX(i-start) = (b(i)-sum) / A(i)(i);
	done &&= Math.abs(x(i)-newX(i-start)) < EPSILON;
      }

      // Send this section to all other processes
      for(w <- 1 until p) 
        toWorker((me+w)%p)!(me, newX, done);

      // Copy newX into x
      for(i <- 0 until height) x(start+i) = newX(i);

      // Receive from others
      for(k <- 0 until p-1){
	val (him, hisX, hisDone) = toWorker(me)?;
	for(i <- 0 until height) x(him*height+i) = hisX(i);
	done &&= hisDone
      }

      barrier.sync;
    }
    if(me==0 && N<20) Print(x);
  }

  def System = || (for (i <- 0 until p) yield Worker(i));

  // Initialise so that A(i)(i) = (i+1)*N; A(i)(j) = 1 
  // for i!=j; b(i) = N;
  def Init = {
    for(i <- 0 until N) A(i)(i) = (i+1)*N;
    for(i <- 0 until N; j <- 0 until N) 
      if(i!=j) A(i)(j) = 1;
    for(i <- 0 until N) b(i) = N;
  }

  def Print(x:Array[Double]) = {
    for(i <- 0 until N){
      for(j <- 0 until N) printf("%2.4f\t", A(i)(j));
      printf("|%2.4f| = |%2.4f|", x(i), b(i));
      println; 
    }
    println
  }

  def main(args:Array[String]) = {
    val t0 = java.lang.System.currentTimeMillis();
    for(i <- 0 until 1){
      Init; System();
    }
    println("Time taken: "+
            (java.lang.System.currentTimeMillis()-t0));
  }
}
\end{scala}

That version is a bit inefficient, since it involves copying the data that is
received into~\SCALA{x}.  The following version replaces \SCALA{x} by a two
dimensional array \SCALA{xs}, indexed by the identity of the worker who
produced it, and the displacement within that segment.  The copying is now
replaced by copying of references.  

Note that it's important for each worker to set \SCALA{xs} to be a new array
on each iteration, since objects are passed on channels by reference, and we
need to ensure that different workers do not share references. 

\begin{scala}
def Worker(me: Int) = proc{
  val start = me*height; val end = (me+1)*height;
  var done = false;
  // val x = new Array[Double](N);
  val xs = new Array[Array[Double]](p,height)
  // xs(k)(i) holds x(k*height+i)

  while(!done){
    val newX = new Array[Double](height); 
    // newX(i) holds the new value of x(i+start)
     
    done = true; 
    // Update this section of x, storing results in newX
    for(i1 <- 0 until height){
      val i = start+i1
      var sum:Double=0;
      for(k <- 0 until p; j1 <- 0 until height){
        val j = k*height+j1; 
        if(j!=i) sum += A(i)(j)*xs(k)(j1);
      }
      newX(i1) = (b(i)-sum) / A(i)(i);
      done &&= Math.abs(xs(me)(i1)-newX(i1)) < EPSILON;
    }

    // Send this section to all other processes
    for(w <- 1 until p) 
      toWorker((me+w)%p)!(me, newX, done);

    // Copy newX into x
    xs(me) = newX;

    // Receive from others
    for(k <- 0 until p-1){
      val (him, hisX, hisDone) = toWorker(me)?;
      xs(him) = hisX;
      done &&= hisDone
    }
      
    barrier.sync;
  }
  if(me==0 && N<20) Print(xs);
}
\end{scala}
%
(\SCALA{Print} needs to be adapted to deal with the new type.)

In my experiments (parameters as above; eight-processor machine) the first
message-passing program is similar in speed to the shared-variable version
from lectures, but the second one is about 25\% faster.

\end{answer}
