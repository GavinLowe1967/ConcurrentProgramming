\begin{question}
\Programming\ Recall the following synchronisation problem from the previous
problem sheet.  Each process has an integer-valued identity.  A particular
resource should be accessed according to the following constraint: a new
process can start using the resource only if the sum of the identities of
those processes currently using it is divisible by~3.  Implement a class to
enforce this, with operations \SCALA{enter(id: Int)} and \SCALA{exit(id:
  Int)}.  Your class should use semaphores. 

Test your code.
\end{question}

%%%%%

\begin{answer}
My code is below.  The procedure \SCALA{signal} implements the baton-passing,
passing the baton either to a waiting process, or lifting the mutual
exclusion. 
\begin{scala}
class Mod3S extends Mod3{
  /** sum of identites of processes currently in critical region. */
  private var current = 0

  /** Number of processes waiting. */
  private var waiting = 0

  /** Semaphore for blocked processes to wait on.  
    * A signal implies current%3 = 0. */
  private val entry = SignallingSemaphore()

  /** Semaphore for mutual exclusion. */
  private val mutex = MutexSemaphore()
    
  /** Pass the baton, either signalling for another process to enter, or lifting
    * the mutex. */
  private def signal = {
    if(current%3 == 0 && waiting > 0) entry.up
    else mutex.up
  }

  def enter(id: Int) = {
    mutex.down
    if(current%3 != 0){
      waiting += 1; mutex.up; entry.down // wait for a signal
      assert(current%3 == 0); waiting -= 1
    }
    current += id
    signal
  }

  def exit(id: Int) = {
    mutex.down
    current -= id
    signal
  }
}
\end{scala}

Here's another solution.
\begin{scala}
class Mod3S2 extends Mod3{
  /** sum of identites of processes currently in critical region. */
  private var current = 0

  /** Semaphore to protect shared variable. */
  private val mutex = MutexSemaphore()

  /** Semaphore on which threads may have to wait. */
  private val canEnter = MutexSemaphore()
  /* Invariant: if canEnter is up then current%3 = 0.  A thread that does
   * canEnter.down will be the next to enter. */

  def enter(id: Int) = {
    canEnter.down 
    assert(current%3 == 0)
    // at most one thread can be waiting here (+)
    mutex.down
    current += id
    if(current%3 == 0) canEnter.up
    mutex.up
  }

  def exit(id: Int) = {
    mutex.down
    current -= id // does this maintain the invariant? (*)
    if(current%3 == 0 && id%3 != 0) canEnter.up
    mutex.up
  }
}
\end{scala}
%
This has an intended invariant:
\begin{quote}
 if |canEnter| is up then $\sm{current}\%3 = 0$.
\end{quote}
Most steps obviously maintain this.  To see that |(*)| maintains it, it is
necessary to observe that the critical section contains at most one thread
with identity |id| such that $\sm{id} \% 3 \ne 0$ (once one such is in the
critical section, another can't enter it).  Hence, if |canEnter| is up when a
call to |exit(id)| begins, then $\sm{current}\%3 = 0$, so $\sm{id} \% 3 = 0$.
Hence |(*)| maintains the invariant.

It is also important that there is at most one thread waiting at |(+)|: if
there are two, then the first to get the mutex could set $\sm{current}\%3 \ne
0$, and then the second could enter against the rules of the question.  The
form of the |if| statement in |exit| is necessary to enforce this: only a
thread that makes $\sm{current}\%3 = 0$ does  |canEnter.up| to allow a
thread to reach |(+)|.

Of course, the testing code from the previous version can be reused.
\end{answer}
