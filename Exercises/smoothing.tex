\begin{question}
Suppose we represent a graphical image by a value of type \SCALA{Image} where 
\begin{scala}
  type Row = Array[Boolean]
  type Image = Array[Row]
\end{scala}
Each \SCALA{Boolean}-valued entry indicates whether the corresponding pixel is
set.

\emph{Smoothing} is an operation that is sometimes applied as part of image
processing, to remove small irregularities.  The operation proceeds in rounds.
In each round, each pixel becomes set if the majority of it and its neighbours
are set (where each pixel has up to eight neighbours); otherwise it becomes
unset.  This is repeated until no more changes occur, or some limit on the
number of rounds is reached (the latter condition is necessary, because there
are some images that never converge).  Within each round, each pixel is
treated independently.

Give a concurrent implementation of smoothing that uses shared variables.
Give your implementation the signature
\begin{scala}
/** Smooth image a, using p workers, with at most maxIters iterations. */
class SmoothShared(a: Array[Array[Boolean]], p: Int, maxIters: Int){
  def apply() = ...
}
\end{scala}
If you like, you may assume that the number of rows is a multiple of the
number of worker threads that you use.  Test your code using the test harness
on the course webpage.
\end{question}

%%%%%%%%%%%%%%%%%%%%%%%%%%%%%%%%%%%%%%%%%%%%%%%%%%%%%%%

\begin{answer}
My code is below.  Each worker deals with a strip of \SCALA{height = N/p}
rows.  This version uses two arrays: on each round, the workers cooperate to
do a round of smoothing on one array, writing the results into the second
array; the roles of the arrays swap between rounds.  A barrier synchronisation
is used at the end of each round to coordinate, and also to decide whether
they can terminate, i.e.~if no changes happened on this round.  The code below
uses the function |Smooth.majority|, which was provided to students in the
testing code.
%
\begin{scala}
class SmoothShared2(a: Array[Array[Boolean]], p: Int, maxIters: Int){
  private val n = a.length; private val w = a(0).length // height and width
  assert(a.forall(_.length == w))
  assert(n%p == 0); private val height = n/p // height of each strip

  // The two arrays.  We set a1 to be the original a. 
  private val a1 = a; private val a2 = Array.ofDim[Boolean](n,w)

  /** Barrier used at the end of each round, and also to decide termination. */
  private val combBarrier = new lock.AndBarrier(p)

  /** Worker process. */
  private def worker(me: Int) = proc("worker"+me){
    val start = me*height; val end = (me+1)*height
    // This worker is responsible for rows [start..height)
    var oldA = a1; var newA = a2  // Old and new versions of the image
    var done = false; var iters = 0

    while(!done && iters < maxIters){
      var myDone = true // have no values changed yet?
      for(i <- start until end; j <- 0 until w){
	newA(i)(j) = Smooth.majority(oldA, i, j)
	myDone &&= (newA(i)(j) == oldA(i)(j))
      }
      iters += 1; done = combBarrier.sync(myDone) // Synchronise with others
      if(!done && iters < maxIters){ val t = oldA; oldA = newA; newA = t }
    }

    // Copy newA into a, if not already equal. 
    if(newA != a) for(i <- start until end) a(i) = newA(i)
  }

  /** Smooth the image. */
  def apply() = run( || (for (i <- 0 until p) yield worker(i)) )
}
\end{scala}


\end{answer}

%%  Two barrier
%% synchronisations are used in each round; before the first, all processes may
%% read the shared variables; between the first and second synchronisations, each
%% process may write to its share of the shared variable.  The second barrier
%% synchronisation is also used to decide whether they can terminate, i.e.~if no
%% changes happened on this round.  It's important to avoid reference sharing; I
%% do this by each worker creating new arrays for its rows on each iteration.
%% %
%% \begin{scala}
%% /** Smooth image a, using p workers, with at most maxIters iterations. */
%% class SmoothShared(a: Array[Array[Boolean]], p: Int, maxIters: Int){
%%   private val n = a.length
%%   private val w = a(0).length; assert(a.forall(_.length == w))
%%   assert(n%p == 0); private val height = n/p // height of each strip

%%   /** Barrier used at the end of each reading stage. */
%%   private val barrier = new Barrier(p)

%%   /** Barrier uses at the end of each writing state, and to decide termination. */
%%   private val combBarrier = new lock.AndBarrier(p)
   
%%   /** Test if majority of neightbours of a(i)(j) are set. */
%%   def majority(i: Int, j: Int): Boolean = {
%%     var sum = 0 // # set neighbours so far
%%     var count = 0 // # neighbours so far
%%     for(i1 <- i-1 to i+1; if i1 >= 0 && i1 < n;
%%         j1 <- j-1 to j+1; if j1 >= 0 && j1 < w){
%%       count += 1; if(a(i1)(j1)) sum += 1
%%     }
%%     2*sum >= count
%%   }

%%   /** Worker process. */
%%   private def worker(me: Int) = proc("worker"+me){
%%     val start = me*height; val end = (me+1)*height
%%     // This worker is responsible for rows [start..height)
%%     var done = false; var iters = 0

%%     while(!done && iters < maxIters){
%%       // This thread will calculate next smoothed values in newA.  newA(i) will
%%       // hold the new value for myA(start+i).
%%       val newA = Array.ofDim[Boolean](height, w)
%%       var myDone = true // have no values changed yet?
%%       for(i <- start until end; j <- 0 until w){
%% 	newA(i-start)(j) = majority(i, j)
%% 	myDone &&= (newA(i-start)(j) == a(i)(j))
%%       }
%%       barrier.sync

%%       // Copy into a
%%       for(i <- start until end) a(i) = newA(i-start)
%%       // Synchronise with other workers
%%       done = combBarrier.sync(myDone) 
%%       iters += 1
%%     }
%%   }

%%   /** Smooth the image. */
%%   def apply() = run( || (for (i <- 0 until p) yield worker(i)) )
%% }
%% \end{scala}

%% Alternatively, we could use two arrays, one for reading and one for writing,
%% with the roles swapping at the end of each round.  This means only one barrier
%% synchronisation is necessary. 


%% %% Testing is best done by generating random images, and comparing the result
%% %% against the results for a sequential algorithm (on a clone of the image).
%% \end{answer}

