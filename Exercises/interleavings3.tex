% based on Andrews 2.15
\begin{question}
Consider the code below.
%
\begin{scala}
var x = 0; var y = 10

def p = proc{ while(x != y) x = x+1 }

def q = proc{ while(x != y) y = y-1 }

def system = p || q
\end{scala}
%
Will the process \SCALA{system} terminate?  Explain your answer.
\end{question}
%
\begin{answer}
It might or might not terminate.

Consider an execution where we reach a state with \SCALA{x=4} and \SCALA{y=5},
both processes evaluate \SCALA{x!=y} in this state and enter the body of the
loop.  Now suppose each loop body is executed, giving \SCALA{x=5} and
\SCALA{y=4}.  Subsequently we always have \SCALA{x>y}, so neither process
terminates.

Alternatively, again suppose we reach a state with \SCALA{x=4} and
\SCALA{y=5}, both processes evaluate \SCALA{x!=y} in this state and enter the
body of the loop.  Now suppose \SCALA{p} runs, setting \SCALA{x=5}, evaluates
the guard which is now true, and terminates.  However, when \SCALA{q} runs, it
sets \SCALA{y=4}, so never terminates.  Similarly, it is possible
for~\SCALA{q} to terminate, but not~\SCALA{p}.

There are also problems that are caused by the use of caching.  For example,
|p| could reach a state with |x = 6|, while working with |y = 10| in its
cache; and |q| could reach a state with |y = 4|, while working with |x = 0| in
cache.  When they do refresh their caches, they will again run forever.  


It is easy to find executions where both terminate.
\end{answer}
