\begin{question}
Consider a class representing a bounded queue of integers with the following
interface. 
%
\begin{scala}
class Queue{
  def add(value:Int) 
  // Add value to the queue.  
  // Precondition: the queue is not full.

  def remove : Int
  // Remove and return the first element of the queue.
  // Precondition: the queue is not empty.

  def isFull : Bool
  // Test whether the queue is full.

  def isEmpty : Bool
  // Test whether the queue is empty.
}
\end{scala}
%
Assume that each operation is executed atomically.

\begin{enumerate}
\item
Which pairs of operations are independent?

\item
Suppose the queue is shared by two processes:
\begin{itemize}
\item
A \emph{producer}, which (repeatedly) waits until the queue is not full
(tested by repeatedly calling \SCALA{isFull}) and then adds a value to the
queue; 

\item
A \emph{consumer}, which (repeatedly) waits until the queue is not empty
(tested by repeatedly calling \SCALA{isEmpty}) and then removes a value from
the queue.
\end{itemize}
%
Define a \emph{bad} interleaving of the executions of the processes to be
where an operation is called when its precondition does not hold.  Explain why
no bad interleaving can occur.

\item
Now suppose there are two consumers and one producer.  Show that bad
interleavings can occur.
\end{enumerate}


\end{question}

%%%%%

\begin{answer}
\begin{enumerate}
\item
The following pairs of operations are \emph{not} independent: 
\begin{scala}
(ifFull, remove), (isEmpty, add), 
(add, add), (remove, remove), (add, remove)
\end{scala}
The last pair is independent when executed such that the preconditions hold. 

\item
The producer waits until \SCALA{isFull} returns false before attempting
\SCALA{add}.  There may actions by the consumer between those two producer
actions, but they will not make the queue full.  Hence the precondition of
\SCALA{add} holds.

Similarly, actions by the producer between the consumer testing
\SCALA{isEmpty} and calling \SCALA{remove} cannot make the queue empty, so the
precondition of \SCALA{remove} will hold.

\item
Suppose the queue contains one item.  Both consumers test \SCALA{isEmpty},
which returns false.  Now the first consumer calls \SCALA{remove}, which
succeeds, but leaves the queue empty.  Now when the second consumer calls
remove, it fails.
\end{enumerate}
\end{answer}
