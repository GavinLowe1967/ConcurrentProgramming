\begin{question}
\Programming\
Consider the following synchronisation problem.  Each process has an
integer-valued identity.  A particular resource should be accessed according
to the following constraint: a new process can start using the resource only
if the sum of the identities of those processes currently using it is
divisible by~3.  Implement a monitor with procedures \SCALA{enter(id: Int)} and
\SCALA{exit(id: Int)} to enforce this.  

Write a testing framework for your monitor using the linearizability
framework.  \textbf{Hint:} one approach is to use an immutable |Set[Int]| as
the sequential specification object, representing the set of identities of
processes currently using the resource, and to include a suitable precondition
for the sequential version of |enter|.
\end{question}

%%%%%

\begin{answer}
Here's my code.
%
\begin{scala}
class Mod3M{
  /** Sum of identites of processes currently in critical region. */
  private var current = 0

  /** Enter the critical region. */
  def enter(id: Int) = synchronized{
    while(current%3 != 0) wait()
    current += id
    if(current%3 == 0) notify()
  }

  /** Leave the critical region. */
  def exit(id: Int) = synchronized{
    current -= id
    if(current%3 == 0) notify()
  }
}
\end{scala}

Alternatively, one can note that the following is invariant: at most one
process whose identity is not divisible by 3 is present at any time.
Therefore it would be enough to store a boolean recording whether such a
process is currently present.

Here's my testing code, using the linearizability framework. 
%
\begin{scala}
import ox.cads.testing._

object Mod3MTest{
  var iters = 200 // # iterations by each worker; each iteration is two ops
  var reps = 1000  // # runs
  var numWorkers = 4 // # workers

  // Sequential specification: the identities of threads currently in the
  // critical region
  type S = scala.collection.immutable.Set[Int]

  // Corresponding sequential operations. 
  def seqEnter(id: Int)(current: S): (Unit, S) = {
    require(current.sum%3 == 0); ((), current+id)
  }

  def seqExit(id: Int)(current: S): (Unit, S) = {
    require(current.contains(id)); ((), current-id)
  }

  // A worker thread
  def worker(me: Int, log: GenericThreadLog[S,Mod3M]) = {
    for(i <- 0 until iters){
      log.log(_.enter(me), "enter("+me+")", seqEnter(me))
      log.log(_.exit(me), "exit("+me+")", seqExit(me))
    }
  }

  // The main method
  def main(args: Array[String]) = {
    for(i <- 0 until reps){
      val mon = new Mod3M
      val tester = LinearizabilityTester.JITGraph[S, Mod3M](
        0, mon, numWorkers, worker _, 2*iters)
      assert(tester() > 0)
      if(i%10 == 0) print(".")
    }
    println
  }
}
\end{scala}

Alternatively, the testing could be based on a sequential specification object
that just records the sum of the identities of current threads.
%
\begin{scala}
  type S = Int

  def seqEnter(id: Int)(current: S): (Unit, S) = {
    require(current%3 == 0); ((), current+id)
  }

  def seqExit(id: Int)(current: S): (Unit, S) = {
    require(id <= current); ((), current-id)
  }
\end{scala}
\end{answer}
