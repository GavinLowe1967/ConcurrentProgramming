\begin{question}
\Programming\ 
Re-implement the prefix sum example, so as to use shared variables, rather
than message-passing.  You need to think carefully how to avoid race
conditions.  

Test your code by adapting the test harness for the prefix sums example from
lectures. 
\end{question}

%%%%%

\begin{answer}
We use two arrays, with the role of the arrays swapping at the end of each
round. 
%
\begin{scala}
class PrefixSumsShared2(n: Int, a: Array[Int]){
  require(n == a.size)

  /** Shared arrays, in which sums are calculated. */
  private val sum1, sum2 = new Array[Int](n) 

  /** Holds the final result. */
  private var result: Array[Int] = null

  /** Barrier synchronisation object. */
  private val barrier = new Barrier(n)

  /** An individual thread.  summer(me) sets sum[me] equal to sum(a[0..me]). */
  def summer(me : Int) = proc("Summer "+me){
    // Invariant: gap = 2^r and oldSum(me) = sum a(me-gap .. me] 
    // (with fictious values a[i]=0 for i<0).  r is the round number.
    
    var oldSum = sum1; var newSum = sum2
    var r = 0; var gap = 1; oldSum(me) = a(me)

    while(gap < n){ 
      barrier.sync()
      val inc = if(gap <= me) oldSum(me-gap) else 0 
                                             // inc = sum a(me-2*gap .. me-gap]
      newSum(me) = oldSum(me) + inc  // newSum(me) = sum a(me-2*gap .. me]
      r += 1; gap += gap;                  // newSum(me) = sum a(me-gap .. me]
      // Swap the arrays for the next round
      val t = oldSum; oldSum = newSum; newSum = t 
                                           // oldSum(me) = sum a(me-gap .. me]
    }

    if(me == 0) result = oldSum        // Set final result
  }

  // Put system together
  def apply(): Array[Int] = {
    run(|| (for (i <- 0 until n) yield summer(i)))
    result
  }
}
\end{scala}


%% Each round uses two barrier synchronisations, one at the start and one in the
%% middle: between the two synchronisations, all \SCALA{sum} variables are
%% treated as read-only; after the second synchronisation, \SCALA{sum(k)} may be
%% written by \SCALA{summer(k)}.
%% %
%% \begin{scala}
%% /** Calculate prefix sums of an array of size n in poly-log n (parallel) steps.
%%   * Based on Andrews Section 3.5.1. */
%% class PrefixSumsShared(n: Int, a: Array[Int]){
%%   require(n == a.size)

%%   /** Shared array, in which sums are calculated. */
%%   private val sum = new Array[Int](n) 

%%   /** Barrier synchronisation object. */
%%   private val barrier = new Barrier(n)

%%   /** An individual thread.  summer(me) sets sum[me] equal to sum(a[0..me]). */
%%   def summer(me : Int) = proc("summer "+me){
%%     // Invariant: gap = 2^r and sum(me) = sum a(me-gap .. me] 
%%     // (with fictious values a[i] = 0 for i < 0).  r is the round number.
%%     var r = 0; var gap = 1; sum(me) = a(me)
%%     while(gap<n){ 
%%       barrier.sync()
%%       if(gap<=me){                
%% 	val inc = sum(me-gap)    // inc = sum a(me-2*gap .. me-gap]
%% 	barrier.sync()
%% 	sum(me) = sum(me) + inc  // s = sum a(me-2*gap .. me]
%%       }
%%       else barrier.sync()
%%       r += 1; gap += gap;        // s = sum a(me-gap .. me]
%%     }
%%   }

%%   // Put system together
%%   def apply(): Array[Int] = {
%%     run(|| (for (i <- 0 until n) yield summer(i)))
%%     sum
%%   }
%% }
%% \end{scala}

%% Alternatively, we could use two arrays.  On each round, threads read from one
%% array and write to the other.  The roles swap at the end of each round.  This
%% means we need only a single synchronisation on each round.

%% This can be tested the same way as the version in lectures (generate random
%% |n| and |a|; use |PrefixSumsShared| to calculate the prefix sums; compare the
%% result with sequentially calculated prefix sums; repeat).
\end{answer}
