% Loosely based on A7.9
\begin{question}
\Programming\ Consider a setting with two kinds of processes, which we will
call Male and Female; suppose we have $N$ Males and $N$ Females.  The aim is
to pair the processes off for some purpose, so each Male is paired with a
Female, and vice versa.  Each process should end up in possession of its
partner's identity.  Design and implement a protocol to achieve this.  The
protocol should be such that the pairings are decided nondeterministically (so
it's not acceptable for the protocol to simply pair each Male off with the
Female with the same index, for example).  The protocol should not use any
process other than the Males and Females.  Note: you need to be careful to
follow the restrictions on the use of \SCALA{alt}s.
\end{question}

%%%%%

\begin{answer}
We will arrange (arbitrarily) for the Female to send a ``proposition'' to a
Male, and she is willing for any Male to receive this; each Male will accept a
single proposition, and is willing to accept it from any Female.  (I'm not
sure which gender this is less polite towards!)   

It is worth considering the type of the proposition channel.  One approach
that doesn't work is to have $N^2$ \SCALA{OneOne} channels, so that Female~$i$
propositions Male~$j$ on channel \SCALA{prop(i)(j)}; this would mean that
Female~$i$ would use an \SCALA{alt} over the output ports of the
\SCALA{prop(i)(j)}, for $j = 0, \ldots, N-1$; and that Male~$j$ uses an
\SCALA{alt} over the input ports of the \SCALA{prop(i)(j)}, for $i = 0,
\ldots, N-1$; but that breaks one of the restrictions on the use of
\SCALA{alt}.

Another approach that doesn't work is to give each Female a \SCALA{OneMany}
proposition channel, on which it can send a proposition to an arbitrary Male.
However, then each Male would use an \SCALA{alt} over the input ports of those
channels, breaking the other restriction on the use of\SCALA{alt}.

The approach we do take is to use a single \SCALA{ManyMany}
channel~\SCALA{prop}, on which an arbitrary Female may proposition an
arbitrary Male.  The Female can send her identity within the proposition.  The
Male can then respond, sending his own identity; we give each Female a
\SCALA{ManyOne} channel for this.
%
\begin{scala}
import ox.CSO._

object MaleFemale{
  val N = 5; // number of males and females

  def Female(me: Int, prop: ![Int], resp: ?[Int]) 
  = proc("Female "+me){
    prop!me;
    val him = resp?;
    report!("Female "+me+" coupled with male "+him);
  }

  def Male(me: Int, prop: ?[Int], resp: Seq[![Int]]) 
  = proc("Male "+me){
    val her = prop?;
    resp(her)!me;
    report!("Male "+me+" coupled with female "+her);
  }

  val prop = ManyMany[Int];
  val resps = ManyOne[Int](N);
  val report = ManyOne[String]

  val System = (
    || ( for(i <- 0 until N) 
           yield Female(i, prop, resps(i)) )
    || 
    || ( for (i <- 0 until N) 
           yield Male(i, prop, resps) )
    || ox.cso.Components.console(report) 
  );

  def main(args : Array[String]) = System();
}
\end{scala}
\end{answer}

    
