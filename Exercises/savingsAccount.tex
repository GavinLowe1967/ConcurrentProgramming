\begin{question}
\Programming\ (Based on Andrews, Exercise 7.8.)  
%
A savings account is shared by several people.  Each person may deposit or
withdraw funds from the account, but the balance may never become negative.  

Develop a server to model this problem.  Withdrawal operations must be queued
until there are sufficient funds available.  (You might like to use a
\url{scala.collection.mutable.Queue} to hold such queued withdrawals.)
\end{question}

%%%%%

\begin{answer}
\begin{scala}
import ox.CSO._

object SavingsAccount{
  type ClientId = Int;
 
  abstract class Request;
  case class Deposit(amount:Int, c:ClientId) extends Request; 
  case class WithDrawal(amount:Int, c:ClientId) extends Request;

  abstract class Response;
  case class Ok() extends Response; 
 
  // The server
  def Server(req: ?[Request], resp: Seq[![Response]]) = proc{
    var balance = 0; // current balance
    // queue of withdrawal requests not yet serviced
    val queue = new scala.collection.mutable.Queue[Request] 

    repeat{
      val request = req?;
      request match{
	case Deposit(amount, c) => {
	  balance += amount;
	  resp(c) ! new Ok();
	  println("Deposit of "+amount+"; balance now "+balance);
	  // See if any queued withdrawals can be serviced
	  var done = false;
	  while(! queue.isEmpty && !done){
	    queue.front match{
	      case WithDrawal(amount1, c1) => {
		if(amount1<=balance){
		  queue.dequeue; balance -= amount1; 
                  resp(c1) ! new Ok();
		  println("Withdrawal of "+amount1+ 
                    " dequeued; balance now "+balance);
		} 
		else done = true;
	      }
	    }
	  }
	}
	case WithDrawal(amount, c) => {
	  if(amount<=balance){
	    balance -= amount; resp(c) ! new Ok();
	    println("Withdrawal of "+amount+
                    "; balance now "+balance);
	  }
	  else{
	    queue += request;
	    println("Withdrawal of "+amount+" queued");
	  }
	}
      }
    }
  }

  val random = new scala.util.Random;

  def Client(me: ClientId, req: ![Request], resp: ?[Response]) 
  = proc{
    repeat{
      if(random.nextInt(2)==0){	// Make deposit
	req ! new Deposit(random.nextInt(10), me);
	resp?; ()
      }
      else{ // Make withdrawal
	req ! new WithDrawal(random.nextInt(10), me);
	resp?; ()
      }
      sleep(random.nextInt(2000))
    }
  }

  val numClients = 5;

  val req = ManyOne[Request];
  val resp = OneOne[Response](numClients);
	  
  def Clients =
    || (for (c <- 0 until numClients) yield
          Client(c, req, resp(c)) )

  def System = Clients || Server(req, resp)

  def main(args:Array[String]) = System();
}
\end{scala}

There are a number of possible variations.  The above implementation services
a new withdrawal request, even if there are others queued; alternatively it
could be added to the back of the queue.  The above implementation only tries
to service queued withdrawal requests in queue order, until one is found that
is for too much; there might be withdrawal requests for smaller amounts
further down the queue that could be serviced immediately.

The system can deadlock if all clients try to make withdrawals for which
insufficient funds are available.  This seems unavoidable.
\end{answer}
	  
