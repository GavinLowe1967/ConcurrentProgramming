\begin{question}\label{Q:menWomenM}
\Programming\ 
Recall the following synchronisation problem from an earlier sheet.  There are
two types of client process, which we shall call \emph{men} and \emph{women}.
These processes need to pair off for some purpose, with each pair containing
one process of each type.  Design a monitor to support this, with public
methods
\begin{scala}
  def manSync(me: String): String = ...
  def womanSync(me: String): String = ...
\end{scala}
Each process should pass its name to the monitor, and receive back the name of
its partner.

Test your code.
\end{question}

%%%%%

\begin{answer}
My code is below.  (I've arranged for it to have the same signature as
for the similar question on an earlier sheet, to allow the testing
code to be re-used.)

%%% I think the first solution is less clear than the second.

% Each man or woman tries to write his/her name to a monitor variable |oMan| or
% |oWoman|.  He/she will then be the next man or woman to be paried.  We use
% |Option| variables here, to have a convenient way to indicate no current
% name.  Access to these variables is controlled by the conditions |manProceed|
% and |womanProceed|.

% Whoever is first out of a (prospective) couple will wait.  The second will
% signal to the first (on |manFinished| or |womanFinished|).  The first will
% then clear the two names, and signal to the next couple. 

% A typical execution of a couple (ignoring the initial waiting) will be either
% as follows, or with the genders reversed:
% \begin{enumerate}
% \item man writes his name, waits for a signal
% \item woman writes her name, signals to man, finishes
% \item man clears names and signals to next man and woman, finishes.
% \end{enumerate}



% \begin{scala}
% /** A solution using a monitor. */
% class MenWomenMonitor extends MenWomen{
%   /** Optionally the man currently waiting. */
%   private var oMan: Option[String] = None

%   /** Optionally the woman currently waiting. */
%   private var oWoman: Option[String] = None

%   /** Monitor to control the synchronization. */
%   private val monitor = new Monitor

%   /** Condition on which a man waits before writing to oMan. */
%   private val manProceed = monitor.newCondition

%   /** Condition on which a woman waits before writing to oWoman. */
%   private val womanProceed = monitor.newCondition

%   /** Condition on which a man waits for a woman.  A signal indicates oWoman has
%     * been written. */
%   private val manFinish = monitor.newCondition

%   /** Condition on which a woman waits for a man.  A signal indicates oMan has
%     * been written. */
%   private val womanFinish = monitor.newCondition

%   /** Clear both names, and signal to next man and woman that they can write
%     * their names. */
%   private def clearNames() = {
%     oMan = None; manProceed.signal
%     oWoman = None; womanProceed.signal
%   }

%   /** A man tries to find a partner. */
%   def manSync(me: String): String = monitor.withLock{
%     // Wait to write my name
%     manProceed.await(oMan.isEmpty); oMan = Some(me)
%     if(oWoman.nonEmpty){
%       womanFinish.signal() // signal to her
%       oWoman.get // get her name
%     }
%     else{
%       manFinish.await // wait for a signal from a woman
%       assert(oWoman.nonEmpty)
%       val her = oWoman.get // record her name
%       clearNames() // clear names and signal to next man, woman
%       her
%     }
%   }

%   /** A woman tries to find a partner. */
%   def womanSync(me: String): String = monitor.withLock{
%     // Wait to write my name
%     womanProceed.await(oWoman.isEmpty); oWoman = Some(me)
%     if(oMan.nonEmpty){
%       manFinish.signal() // signal to him
%       oMan.get // get his name
%     }
%     else{
%       womanFinish.await // wait for a signal from a man
%       assert(oMan.nonEmpty)
%       val him = oMan.get // record his name
%       clearNames() // clear names and signal to next man, woman
%       him
%     }
%   }
% }
% \end{scala}

%%%%%

This solution is asymmetric.  It uses a single variable to
hold the current status of the protocol, and three conditions, to signal
between different participants.  A typical execution of a couple will be:
\begin{itemize}
\item man waits for slot to become empty, writes his name, signals to woman,
(on |manWaiting|), and waits;

\item woman waits for signal, reads man's name, writes her name, signals to
man (on |womanDone|), and finishes;

\item man reads woman's name, clears name, signals to next man (on
|slotEmpty|), and finishes.
\end{itemize}
%
\begin{scala}
/** Another solution using a monitor. */
class MenWomenMonitor2 extends MenWomen{
  /** Information about who is currently waiting. */
  private abstract class Status

  /** The man "him" is currently waiting. */
  private case class ManWaiting(him: String) extends Status

  /** The woman "her" has just paired with the current man. */
  private case class WomanDone(her: String) extends Status

  /** Nobody is currently waiting. */
  private case object Nobody extends Status

  /** A slot which holds information about the current status of the exchange. */
  private var slot: Status = Nobody

  /** Monitor to control the synchronization. */
  private val monitor = new Monitor

  /** Condition to signal that the slot has been cleared, ready for the next
    * couple. */
  private val slotEmpty = monitor.newCondition

  /** Condition to signal that a man's name is in the slot, and he is waiting
    * for a woman. */
  private val manWaiting = monitor.newCondition

  /** Condition to signal that a woman has written her name, and is paired with
    * the waiting man. */
  private val womanDone = monitor.newCondition

  /** A man tries to find a partner. */
  def manSync(me: String): String = monitor.withLock{
    // Wait to write my name
    slotEmpty.await(slot == Nobody)          // wait for signal (1)
    slot = ManWaiting(me); manWaiting.signal  // signal to woman waiting at (2)
    womanDone.await                          // wait for woman (3)
    val WomanDone(her) = slot
    slot = Nobody; slotEmpty.signal         // signal to man waiting at (1)
    her
  }

  /** A woman tries to find a partner. */
  def womanSync(me: String): String = monitor.withLock{
    manWaiting.await(slot.isInstanceOf[ManWaiting]) // wait for signal (2)
    val ManWaiting(him) = slot
    slot = WomanDone(me); womanDone.signal   // signal to man waiting at (3)
    him
  }
}
\end{scala}

There are other solutions.  My impression is that symmetric solutions are
less clear.

As hinted above, testing can be done precisely as on the earlier sheet.
\end{answer}



% There are two types of
% process, which we shall call Men and Women.  These processes need to pair off
% for some purpose, with each pair containing one process of each type.  A
% monitor is required, with a procedure for each type of process.  The procedure
% should block processes until a process of the other type is ready, at which
% point both should continue.
% %
% \begin{enumerate}
% \item
% The following monitor was suggested.  Explain why this does not meet the
% requirement. 
% %
% \begin{scala}
% object Monitor1{
%   var menWaiting = 0; var womenWaiting = 0; 
%     // number of men, women currently waiting
%   var manCommitted, womanCommitted = false;
%     // Have the next man, woman to return been decided 
%     // but not yet paired?

%   def ManEnter = synchronized{
%     if(womenWaiting==0 && !womanCommitted){
%       menWaiting += 1;
%       while(womenWaiting==0 && !womanCommitted) 
%         wait(); // wait for a woman
%       menWaiting -= 1; womanCommitted = false;
%     }
%     else{ 
%       manCommitted = true; notify();  // wake up a man
%     }
%   }

%   def WomanEnter = synchronized{
%      if(menWaiting==0 && !manCommitted){
%         womenWaiting += 1;
%         while(menWaiting==0 && !manCommitted) 
%           wait(); // wait for a man
%         womenWaiting -= 1; manCommitted = false;
%      }
%      else{ 
%        womanCommitted = true; notify(); // wake up a woman
%      }
%   }
% }
% \end{scala}

% \item
% Now the following monitor is suggested.  Explain why this doesn't meet the
% requirement, either.
% %
% \begin{scala}
% object Monitor2{
%   var menWaiting = 0; var womenWaiting = 0; 
%     // number of men, women currently waiting
%   var manCommitted, womanCommitted = false;
%     // Have the next man, woman to return been decided 
%     // but not yet paired?

%   def ManEnter = synchronized{
%     if(womenWaiting==0 && !womanCommitted || manCommitted){
%       menWaiting += 1;
%       while(womenWaiting==0 && !womanCommitted 
%             || manCommitted) 
%         wait(); // wait for a woman
%       menWaiting -= 1; womanCommitted = false;
%     }
%     else{ 
%       manCommitted = true; notifyAll();  // wake up a man
%     }
%   }

%   def WomanEnter = synchronized{
%     if(menWaiting==0 && !manCommitted || womanCommitted){
%       womenWaiting += 1;
%       while(menWaiting==0 && !manCommitted
%             || womanCommitted)
%         wait(); // wait for a man
%       womenWaiting -= 1; manCommitted = false;
%     }
%     else{ 
%       womanCommitted = true; notifyAll(); // wake up a woman
%     }
%   }
% }
% \end{scala}

% \item
% Suggest a definition for the module that does meet the requirement.
% \end{enumerate}
% \end{question}

% %%%%%%%%%%

% \begin{answer}
% \begin{enumerate}
% \item
% Consider the following execution.
% %
% \begin{enumerate}
% \item A man enters and waits (so \SCALA{menWaiting = 1}).

% \item
% A woman enters, finds \SCALA{menWaiting>0}, so sets \SCALA{womanCommitted =
%   true}, does \SCALA{notify()} and exits.

% \item
% A second woman enters, before the waiting man is scheduled, finds
% \SCALA{menWaiting>0}, so sets \SCALA{womanCommitted = true}, does
% \SCALA{notify()} and exits.

% \item
% Eventually the man is scheduled and exits.
% \end{enumerate}
% %
% So the poor man ends up with two women.  The point is that the second woman
% has no way of knowing that another woman has already paired off with the man. 

% %%%%%

% \item
% Consider the following execution.
% %
% \begin{enumerate}
% \item
% Men 1 and 2 enter and wait (so \SCALA{menWaiting = 2}).

% \item
% Woman 3 enters, finds \SCALA{menWaiting>0}, so sets \SCALA{womanCommitted =
%   true}, does \SCALA{notifyAll()} and exits.

% \item
% Woman 4 enters (before any man is scheduled), and waits (since
% \SCALA{womanCommitted = true}), setting \SCALA{womenWaiting = 1}.

% \item
% Man 1 is scheduled, sets \SCALA{womanCommitted = false}, \SCALA{menWaiting =
%   1}, and exits (paired with woman 3).

% \item
% Man 2 is scheduled, finds \SCALA{womenWaiting = 1} and so exits.
% \end{enumerate}
% %
% So poor woman 3 ends up with two men.  The point is that man 2 should wake up
% woman 4. 

% %%%%%

% \item
% I think the following version works.  Each process tests (in the final
% \SCALA{if} statement) whether it is the first or second of the pair to be
% woken. 
% %
% \begin{scala}
% object Monitor3{
%   var menWaiting = 0; var womenWaiting = 0; 
%     // number of men, women currently waiting
%   var manCommitted, womanCommitted = false;
%     // Have the next man, woman to return been decided 
%     // but not yet paired?

%   def ManEnter = synchronized{
%     menWaiting += 1;
%     while(womenWaiting==0 && !womanCommitted || manCommitted) 
%       wait(); // wait for a woman
%     menWaiting -= 1; manCommitted = true; 
%     if(!womanCommitted) notifyAll() 
%     else{ manCommitted = false; womanCommitted = false; }
%   }

%   def WomanEnter = synchronized{
%     womenWaiting += 1;
%     while(menWaiting==0 && !manCommitted || womanCommitted) 
%       wait(); // wait for a woman
%     womenWaiting -= 1; womanCommitted = true; 
%     if(!manCommitted) notifyAll() 
%     else{ manCommitted = false; womanCommitted = false; }
%   }
% }
% \end{scala}
% \end{enumerate}
% \end{answer}
