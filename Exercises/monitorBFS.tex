\begin{question}
Adapt the |TerminatingPartialQueue| class from lectures so as to use a
monitor.  (If you like, you can change the signature, if it's more convenient
for the following part.)

Now adapt the concurrent graph search program from lectures so as to use
shared variables and a monitor.  Test your code by adapting the word path
program. 
\end{question}

%%%%%%%%%%%%%%%%%%%%%%%%%%%%%%%%%%%%%%%%%%%%%%%%%%%%%%%

\begin{answer}
I've changed the signature so that |dequeue| returns an |Option| type.  (The
implementation in lectures threw an exception in the termination case.)  
%
\begin{scala}
/** A partial queue that terminates if all worker threads are attempting to
  * dequeue, and the queue is empty.  This implementation uses a monitor internally.
  * @param numWorkers the number of worker threads. */
class MonitorTerminatingPartialQueue[A](numWorkers: Int){
  /** The queue itself. */
  private val queue = new Queue[A]

  /** The number of threads currently waiting to perform a dequeue. */
  private var waiting = 0

  /** Has the queue been shut down? */
  private var done = false

  /** Enqueue x.  */
  def enqueue(x: A) = synchronized{ 
    if(!done){
      queue.enqueue(x)
      if(waiting > 0) notify()
    }
  }

  /** Attempt to dequeue a value.  
    * @return None if the queue has been shut down, or it the queue is empty
    * and all threads are trying to dequeue. */
  def dequeue: Option[A] = synchronized{
    if(!done && queue.isEmpty){
      if(waiting == numWorkers-1) shutdown() // System should terminate
      else{ // we need to wait
        waiting += 1
        while(queue.isEmpty && !done) wait()
        waiting -= 1
      }
    }
    if(done) None 
    else Some(queue.dequeue)
  }

  /** Shut down this queue. */
  def shutdown() = synchronized{ done = true; notifyAll() }
}
\end{scala}
%
%% Note that if a |dequeue| operation was waiting, has been woken up, but not yet
%% decremented received the lock, then it counts as still waiting as far as the
%% test ``|waiting == numWorkers-1|'' is concerned: that's what we want. 

Here's the graph search program.  The main changes are:
\begin{itemize}
\item The use of the previous class, and straightforward changes to deal with
  it returning an |Option| type;

\item The termination case: a thread that finds a solution stores it and
  arranges for other threads to terminate; the relevant variable is protected
  by a |synchronized| block to avoid races; note that there is no need for a
  separate coordinator.
\end{itemize}
 %
\begin{scala}
/** A class to search Graph g concurrently. */
class MonitorConcGraphSearch[N](g: Graph[N]) extends GraphSearch[N](g){
  /**The number of workers. */
  val numWorkers = 8

  /** Try to find a path in g from start to a node that satisfies isTarget. */
  def apply(start: N, isTarget: N => Boolean): Option[List[N]] = {
    // All nodes seen so far.
    val seen = new ConcSet[N]; seen.add(start)
    // Queue storing edges and the path leading to that edge
    val queue = new MonitorTerminatingPartialQueue[(N, List[N])](numWorkers)
    queue.enqueue((start, List(start)))    
    // Variable that ends up holding the result.
    var result: Option[List[N]] = None

    // A single worker
    def worker = proc("worker"){
      var done = false
      while(!done) queue.dequeue match{
        case Some((n, path)) =>
          for(n1 <- g.succs(n)){
            if(isTarget(n1)) synchronized{
              if(result.isEmpty){ // we're the first one to find a result
                result = Some(path :+ n1) // store result
                queue.shutdown // cause others to terminate
                done = true
              }
            }
            else if(seen.add(n1)) queue.enqueue((n1, path :+ n1))
          }
        case None => done = true
      } // end of while(...) ... match
    }

    // Release the workers. 
    run(|| (for(_ <- 0 until numWorkers) yield worker))
    result
  }
}
\end{scala}

\end{answer}
