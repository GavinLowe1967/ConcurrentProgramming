\begin{question}
\begin{enumerate}
\item
Write a definition of a process
%
\begin{scala}
def ListBuilder(in: ?[Int], out: ![List[Int]]) = proc{ ... }
\end{scala}
%
with the following behaviour.  The process should be willing to 
%
\begin{itemize}
\item
receive \SCALA{Int}s on \SCALA{in};

\item
send on \SCALA{out} a list containing (in some order) the \SCALA{Int}s it has
received since the last communication on \SCALA{out}.
\end{itemize}
%
(In Scala, you can initialise a variable \SCALA{xs} to the empty list using
the code \SCALA{var xs:List[Int] = Nil}; you can add an element \SCALA{x} onto
\SCALA{xs} using \SCALA{xs = x :: xs}).  \marks{5}
\end{enumerate}

\medskip
%%%%%

\noindent
The rest of this question concerns a program to perform a breadth-first
traversal of a directed graph.  We use a class with the following signature to
represent nodes of the graph:
%
\begin{scala}
class Node(id:Int){
  val successors : List[Int] = ...; // list of successors
  def Visit : Unit = ...;
}
\end{scala}
%
The field \SCALA{id} represents the identity of the node.  The list
\SCALA{n.successors} contains the identities of those nodes \SCALA{n1} such
that there is an edge from this node to~\SCALA{n1}.  

The nodes themselves are stored in an array
%
\begin{scala}
val nodes : Array[Node];
\end{scala}
%
of some size~$N$, so that a node with identity~\SCALA{id} is stored in
\SCALA{nodes(id)}.

%%%%%

\begin{enumerate}
\item[(b)]  
%
Write a concurrent program, to perform a traversal of the graph, executing
\SCALA{Visit} on each node reachable from \SCALA{nodes(0)}.  This should be
done in breadth-first order; i.e., \SCALA{Visit} should be executed on all
nodes at distance~$k$ from the starting node before any node at distance
$k+1$.

Your program should use $p$ worker processes (you may assume $N \mod p = 0$).
Each worker $i$ should be responsible for nodes $[i \times height,
  (i+1)*height)$, where $height = N/p$.  The worker should record which of
those nodes have been previously visited, and call \SCALA{Visit} upon them
  appropriately.  When a worker encounters a new node, the identity of that
  node should be passed on to the worker responsible for it, to be visited
  on the next round.  {\bf Hint:} use an instance of \SCALA{ListBuilder} as a
  subprocess within each worker to receive these identities.

You should briefly explain your design, including an explanation of the
following points:
%
\begin{enumerate}
\item
How your program ensures that the nodes really are visited in breadth-first
order; 

\item
How your program terminates when all reachable nodes have been visited.
\end{enumerate}
                                                                 \marks{20}

%%%%%

%% \item[(c)]
%% %
%% Briefly discuss ways in which your program could be made more efficient.
%%                                                                  \marks{5}
\end{enumerate}


\end{question}

%%%%%%%%%%%%%%%%%%%%%%%%%%%%%%%%%%%%%%%%%%%%%%%%%%%%%%%%%%%%

%%% File GraphSearch/Search1.scala
\begin{answer}
\begin{enumerate}
\item
\begin{scala}
def ListBuilder(in: ?[Int], out: ![List[Int]]) = proc{
  var xs:List[Int] = Nil;
  serve(
    in -?-> { xs = (in?) :: xs; }
    | out -!-> { out!xs; xs = Nil; }
  )
}
\end{scala}

%%%%%
\item
Each worker comprises a \SCALA{Visitor} process, that does the actual
visiting, and a \SCALA{ListBuilder} process that receives identities of nodes
from other workers.

On each round $k$ (starting from 0), the nodes at distance $k$ from the
starting node are visited.  The successors are forwarded to the
\SCALA{ListBuilder}s of the appropriate other workers.  Once all such nodes
have been visited, the workers perform a barrier synchronisation, the
\SCALA{Visitor}s receive the next list of nodes from their
\SCALA{ListBuilder}s, and they perform another barrier synchronisation (the
need for two barrier synchronisations is not obvious. 

On each round, each worker keeps track of whether it has visited any new
nodes.  The second barrier synchronisation calculates whether no worker has
visited any new nodes on this round.  If so, each worker terminates (the
\SCALA{Visitor} closes the channels to the \SCALA{ListBuilder} to make it
terminate). 


\begin{scala}
import ox.CSO._

object Search1{
  val N = 16; // number of nodes
  val p = 8; // number of workers
  assert(N%p==0); 
  // height of strip each node is responsible for:
  val height = N/p;

  // Class to represent nodes
  class Node(id:Int){ ... }

  // Initialise the nodes
  val nodes = new Array[Node](N);
  for(i <- 0 until N) nodes(i) = new Node(i);

  def ListBuilder(in: ?[Int], out: ![List[Int]]) = ...;

  // Combining barrier
  def and(b1:Boolean,b2:Boolean) = b1 && b2
  val combBarrier = new CombiningBarrier(p, true, and);
  val barrier = new Barrier(p);

  // Find worker responsible for node n
  def workerOf(n: Int) = n/height;

  // Channel to send indices of new nodes to workers
  val toWorker = ManyOne[Int](p);

  // Worker to deal with nodes [start..end)
  def Worker(me: Int) = proc{
    val start = me*height; val end = start+height;

    // Channel to communicate between components
    val getNewSeen = OneOne[List[Int]];

    // Process to do the visiting
    def Visitor = proc("Visitor"+me){
      // Store status for node i in visited(i-start)
      val visited = new Array[Boolean](end-start);
      for(i <- start until end) visited(i-start) = false;

      // List of nodes to be considered on the next iteration
      var seen:List[Int] = Nil;
      if(start==0) seen = 0 :: seen;

      var done = false; // are we finished?
      
      // Main loop
      while(!done){
	var myDone = true; 
        // myDone is true if we've visited no new nodes 
        // on this round
	for(i <- seen){
	  if(! visited(i-start)){
	    // Can now visit this node
	    nodes(i).Visit; visited(i-start) = true; 
            myDone = false;
	    // Set all successors as seen
	    for(j <- nodes(i).successors) 
              toWorker(workerOf(j)) ! j
	  }
	}
	barrier.sync;
	seen = getNewSeen? ;
	done = combBarrier.sync(myDone);
      }

      // Now close down listener
      toWorker(me).close; getNewSeen.close; 
    }

    // Put it together
    (Visitor || ListBuilder(toWorker(me), getNewSeen))()    
  }
	  
  // Put the system together
  def System = || (for (i <- 0 until p) yield Worker(i))

  def main(args:Array[String]) = System();
}
\end{scala}

%%%%%

%% \item
%% The current design uses lots of messages.  It's better for each worker to send
%% a single message to each other worker on each round, containing the
%% appropriate \emph{list} of new nodes.

%% With the current design, it would be better to buffer the \SCALA{toWorker}
%% channels.  In this case, each \SCALA{ListBuilder} needs to use a
%% \SCALA{priserve} to ensure the input channel is flushed before sending on the
%% list of next nodes.
\end{enumerate}
\end{answer}
