\begin{question}
Consider a system constructed from $N$ similar processes, with identities
$0, \ldots, N-1$.  Each process can send messages to any other process, on
synchronous channels.

Each process is either \emph{active} or \emph{passive}.  When a process is
passive, it simply waits to receive a message, at which point it becomes
active.  When a process is active, it can send messages to any other
processes, but may eventually become passive.

The aim of this question is to consider a technique for determining whether
all the processes are passive, in which case the system can terminate.

\begin{enumerate}
\item Consider the following scheme.  Process~0, when it is passive, sends a
token to process~1, containing a boolean value, initially $true$.  Each
process~$k$, when it receives a token of this form, sends a similar token to
process~$(k+1) \bmod N$, so the token circulates, as if the processes are
arranged in a ring; note, though that other messages don't need to follow the
ring.  Process~$k$ sets the value in the token to be $false$ if it is active,
or passes on the value it receives if it is passive; in each case, the process
retains its previous status (active or passive).  When the token returns to
process~0, if its value is $true$, it assumes that all processes are passive,
so sends a message to all processes telling them to terminate.

Explain why this scheme does \emph{not} achieve the desired goal.

%%%%%

\item
Describe how to adapt this scheme so that it does achieve the desired goal.
Explain why your scheme works.  Make clear what (safety and liveness)
properties your scheme achieves.

%%%%%

\item
Now suppose the normal messages do follow the ring topology.  Describe how to
adapt the scheme so that it achieves the desired goal. 

% \item
% Does it make a difference if the channels are asynchronous (i.e.~buffered)? 
\end{enumerate}
\end{question}

%%%%%%%%%%%%%%%%%%%%%%%%%%%%%%%%%%%%%%%%%%%%%%%%%%%%%%%

\begin{answer}
\begin{enumerate}
\item
Consider a ring of four processes, with process~3 initially active.  Suppose
the token circulates to process~2, so the boolean is still $true$.  Then
suppose process~3 sends a message to process~1 and becomes passive, but
process~1 remains active.  Then the token is passed on to process~3, then
back to process~0, with the boolean still $true$.  Hence process~0 assumes all
processes are passive, even though process~1 is active.

%%%%%

\item Arrange for the token to go round the ring \emph{twice} (with some field
indicating whether it is on its first or second circuit).  On the second
circuit, each process sets the boolean to be $false$ if it has been active at
any time since it saw the token the first time.

Consider the time~$t$ at which the token first returns to process~0.  If
any process is active at time~$t$, then it will subsequently set the boolean
to $false$.  Hence, if the value ends up $true$, then all processes were
passive at time~$t$, and so all remain passive.  (This is a safety property:
the system terminates only if it is correct to do so.) 

It is possible for all the processes to become passive while the token is
circulating, but after at least one has seen it for the first time, in which
case the scheme does not lead to overall termination.  However, if all are
passive when the token starts circulating for the first time, then termination
will result.  (This is a liveness property: the system does terminate,
under the stated conditions.)


%%%%%

\item Arrange for the termination messages and the data messages to go round
the ring in the same direction.  Sending the token only once round the ring is
now sufficient: process~0 can choose to terminate provided the boolean comes
back as $true$ and it has been continually passive since initiating the token.
The following property is invariant: if process~0 remains passive, and the
boolean in the token is still $true$, then all processes between~0 and the
current token-holder are passive.

\emph{Alternatively} send the termination messages round the ring in the
\emph{opposite} direction, i.e.~node~$i$ sends the termination token to
node~$(i-1) \bmod N$.   The following property is invariant: if process~$i$
sends the token with value $true$, then all processes in the range $[i ..
N-1) \union \set{0}$ are passive.


% \item
% If the channels are asynchronous, then each passive process must check that
% its input channels are empty before passing the token.  Otherwise, such a
% process could subsequently become active as a result of receiving a message
% that was in the channel.

\end{enumerate}
\end{answer}
