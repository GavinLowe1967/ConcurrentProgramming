\begin{question}\label{Q:consumerM}
\Programming\ 
%
Consider the following producer-consumer problem.  Several producers each
produce one piece of (integer) data at a time, which they place into a buffer
using a method \SCALA{put(item: Int)}.  These are accumulated into an array of
size $n$.  A consumer can receive that array, once it is full, using the
method \SCALA{get : Array[Int]}.  

Implement the buffer as a monitor.  The consumer should be blocked until the
array is full, and the producers should be blocked once the array is full.

Write testing code for your implementation.  (If you use the
linearizability testing framework, then remember that the results of
the concurrent and sequential operations need to be comparable using
``|==|'', which won't be the case for |Array[Int]|; I suggest you cast
the results to |List[Int]|.)
\end{question}

%%%%%

\begin{answer}
My code is below.
%
\begin{scala}
class ProducerConsumer(n: Int){
  require(n >= 1)

  /** Array to hold the data. */
  private var a = new Array[Int](n)

  /** Number of items added to a so far. */
  private var count = 0

  /** Monitor to control synchronisation. */
  private val monitor = new Monitor

  /** Condition for signalling that the old buffer has been taken. */
  private val notFull = monitor.newCondition

  /** Condition for signalling that the buffer has been filled. */
  private val full = monitor.newCondition

  /** Add x to the buffer, waiting until it is not full. */
  def put(x: Int) = monitor.withLock{
    notFull.await(count < n)
    a(count) = x; count += 1
    if(count == n) full.signal() // signal to consumer
  }

  /** Get the contents of the buffer, once it is full. */
  def get: Array[Int] = monitor.withLock{
    if(count < n) full.await()
    val result = a
    a = new Array[Int](n); count = 0
    notFull.signalAll() // signal to all producers 
    result    
  }
}
\end{scala}
%
Note that the producer has to re-check the condition each time it is awoken,
since there are multiple producers, and it is possible that other producers
have filled the new buffer in the meantime.  However, there is no need for the
consumer to re-check the condition when re-awoken, since there is only one
consumer, and the implementation of conditions guards against spurious
wake-ups. 

My testing code is below.
%
\begin{scala}
import ox.cads.testing._

object ProducerConsumerTest{
  var iters = 200 // # iterations by each worker
  var maxValue = 20  // max value added to the stack
  var reps = 1000 // # runs
  var n = 4  // Size of each buffer; also the number of producers (to avoid deadlock).

  // Sequential specification type.  Note the sequential specification type
  // needs to be immutable, and comparable using "==", so we can't use an 
  // array here.
  type S = List[Int]

  // Sequential put operation
  def seqPut(x: Int)(buffer: S): (Unit, S) = {
    require(buffer.length < n); ((), buffer :+ x)
  }
  // It might be better to build buffer in reverse, to avoid the O(n) ":+"

  // Sequential get operation.
  def seqGet(buffer: S): (List[Int], S) = {
    require(buffer.length == n); (buffer, List[Int]())
  }

  // A worker thread
  def worker(me: Int, log: GenericThreadLog[S,ProducerConsumer]) = {
    val random = new scala.util.Random
    for(i <- 0 until iters){
      if(me == 0) // consumer
        log.log(_.get.toList, "get", seqGet)
      else{
        val x = random.nextInt(maxValue)
        log.log(_.put(x), "put("+x+")", seqPut(x))
      }
    }
  }
  // Note that with the get method, we cast the result to a List, to allow
  // comparison using "==".

  // The main method
  def main(args: Array[String]) = {
    for(i <- 0 until reps){
      val concBuffer = new ProducerConsumer(n)
      val seqBuffer = List[Int]()
      val tester = LinearizabilityTester.JITGraph[S, ProducerConsumer](
        seqBuffer, concBuffer, n+1, worker _, iters)
      assert(tester() > 0)
      if(i%10 == 0) print(".")
    }
    println; exit
  }
}
\end{scala}
%
Note that we use a |List| for the sequential specification, since an
\emph{immutable} datatype should be used here.  Also, we cast the result of
the concurrent |get| to a |List|, and arrange for the sequential version to
also return a |List|, so that the two can be compared using ``|==|'' (over
arrays, ``|==|'' is reference equality, which would be wrong here). 
\end{answer}
