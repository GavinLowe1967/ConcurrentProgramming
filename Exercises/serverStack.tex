\begin{question}
\Programming\ Implement a concurrent total stack providing the following
operations. 
% 
\begin{scala}
  /** Push x onto the stack. */
  def push(x: T)

  /** Optionally pop a value from the stack.
    * @return Some(x) where x is the value popped, or None if the stack is empty. */
  def pop: Option[T]
\end{scala}
% 
Your implementation should use a server process internally.

Test your implementation using the linearizability testing framework.  The
course website contains an implementation of an immutable stack, which you
might want to use (unfortunately the Scala API no longer contains such a
class).
\end{question}

% ------------------------------------------------------------------

\begin{answer}
My code is below.  This is a straightforward adaptation of the concurrent
queue from lectures.
%
\begin{scala}
/** A total stack, implemented using a server. */
class ServerStack[T]{
  /** Channel for pushing. */
  private val pushC = ManyOne[T]

  /** Channel for popping. */
  private val popC = OneMany[Option[T]]

  /** Push x onto the stack. */
  def push(x: T) = pushC!x

  /** Optionally pop a value from the stack.
    * @return Some(x) where x is the value popped, or None if the stack is empty. */
  def pop: Option[T] = popC?()

  private def server = proc{
    val stack = new scala.collection.mutable.Stack[T]
    serve(
      pushC =?=> { x => stack.push(x) }
      | popC =!=> { if(stack.isEmpty) None else Some(stack.pop) }
    )
  }

  server.fork

  /** Shut down the stack, terminating the server thread. */
  def shutdown = { pushC.close; popC.close }
}
\end{scala}

My testing code is below.  Again this is a straightforward adaptation of the
corresponding code for a queue in lectures.   It would be better if the
parameters could be set on the command line.  
%
\begin{scala}
import ox.cads.testing._

object StackTest{
  var iters = 200      // # iterations by each worker
  var pushProb = 0.3 // probability of each operation being a push
  var maxValue = 20 // max value added to the stack
  var reps = 1000     // # runs

  // Type of sequential specification objects.
  type S = ImmutableStack[Int]

  // Type of concurrent object to be tested.
  type C = ServerStack[Int]

  // Sequential push operation
  def seqPush(x: Int)(stack: S): (Unit, S) = ((), stack.push(x))

  // Sequential pop operation
  def seqPop(stack: S): (Option[Int], S) =
    if(stack.isEmpty) (None, stack)
    else{ val(x, stack1) = stack.pop2; (Some(x), stack1) }

  // A worker thread
  def worker(me: Int, log: GenericThreadLog[S,C]) = {
    val random = new scala.util.Random
    for(i <- 0 until iters)
      if(random.nextFloat <= pushProb){
        val x = random.nextInt(maxValue)
        log.log(_.push(x), "push("+x+")", seqPush(x))
      }
      else log.log(_.pop, "pop", seqPop)
  }

  // The main method
  def main(args: Array[String]) = {
    for(i <- 0 until reps){
      val concStack = new ServerStack[Int]; val seqStack = new ImmutableStack[Int]
      // The tester: Add the parameter tsLog = false on Windows.
      val tester = LinearizabilityTester.JITGraph[S, C](seqStack, concStack, 4, worker _, iters)
      assert(tester() > 0)
      concStack.shutdown
      if(i%10 == 0) print(".")
    }
    println; exit
  }
}
\end{scala}
\end{answer}
