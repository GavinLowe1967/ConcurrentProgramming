\begin{nontutequestion}
(Optional.)
\begin{enumerate}
\item
Suppose $x$ and $y$ are random numbers, chosen with a uniform distribution on
$[0,1)$.  Show that $Prob(x^2+y^2<1) = \pi/4$.  Hint: use a geometric
  argument.

\item
\Programming\ 
Use this idea to write a concurrent program to estimate the value of $\pi$.

\item
(For those who have taken a course on probability:) Estimate the
  standard deviation of the results given by your program.
\end{enumerate}
\end{nontutequestion}

%%%%%

\begin{nontuteanswer}
For part (a): consider a quarter circle, with centre $(0,0)$, inscribed in the
quadrant $0 \le x,y < 1$.  Then
\begin{eqnarray*}
Prob(x^2+y^2<1) & = & Prob(\mbox{$(x,y)$ is in the quarter circle}) \\
 & = & \mbox{area of quarter circle}/\mbox{area of quadrant} \\
 & = & \pi/4.
\end{eqnarray*}

The program below uses the bag of tasks pattern.  Each task is to generate
\SCALA{taskSize} pairs $(x,y)$ and to count how many satisfy $x^2+y^2<1$.  The
controller compiles the results.
%
\begin{scala}
// Estimate pi, using the fact that if x, y are random 
// numbers in [0,1), then P(x^2+y^2)<1 = pi/4.

import ox.CSO._

object pi{
  val taskSize = 20000;
  val numTasks = 50;
  val numWorkers = 8;

  // val random = new scala.util.Random;

  // The worker receives a signal on toWorkers, generate
  // taskSize random pairs (x,y), and tell controller 
  // (on toController) how many had x^2+y^2 < 1.
  def Worker(toWorkers: ?[Unit], toController: ![Int]) 
  = proc{
    val random = new scala.util.Random;
    repeat{
      toWorkers?;
      var count=0;
      for(i <- 0 until taskSize){
	val x = random.nextDouble; 
        val y = random.nextDouble;
	if(x*x+y*y<1.0) count += 1;
      }
      toController!count;
    }
  }

  def Controller(toWorkers: ![Unit], toController: ?[Int]) 
  = proc{
    // Process to distribute numTasks tasks
    def Sender = proc{
      for(i <- 0 until numTasks) toWorkers!();
      toWorkers.close
    }

    // Process to receive counts and calculate result
    def Receiver = proc{
      var count:Int = 0;
      for(i <- 0 until numTasks) count += (toController?) ;
      println( 
        (4.0*(count:Double)/(taskSize*numTasks)).toString
      );
    }

    (Sender || Receiver)();
  }

  // Construct system
  val toWorkers = OneMany[Unit];
  val toController = ManyOne[Int];
  
  def Workers = 
    || ( for(i <- 0 until numWorkers) yield
           Worker(toWorkers, toController) );

  def System = 
    Controller(toWorkers, toController) || Workers

  def main(args:Array[String]) = {
    val t0 = java.lang.System.currentTimeMillis();
    System();
    println("Time taken: "+
      (java.lang.System.currentTimeMillis()-t0)/1000.0);
  }
}
\end{scala}

Each worker can be given their own random number generator, or they can share
a single one.  In my tests, the latter is about 20 times slower, which is quite
surprising.  I suspect the reason it is more than \SCALA{numWorkers} times
slower is that the state of the random number generator has to be copied to
the appropriate cache for each random number.

If we let $n$ be the number of samples, then the number of samples that
satisfy the property $x^2+y^2<1$ has binomial distribution $Binom(n,\pi/4)$,
which has variance $n.\pi/4.(1-\pi/4)$, standard deviation
$\sqrt(n.\pi/4.(1-\pi/4))$, and so a standard deviation on the final result of
$4.\sqrt(\pi/4.(1-\pi/4)) / \sqrt n \approx 1.64 / \sqrt n$.  For $n = 10^6$, as
above, this is about $0.00164$.  This is consistent with the experimental
observation that the program is normally accurate to about two decimal places.
\end{nontuteanswer}
