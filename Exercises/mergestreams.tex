\begin{question}
\Programming\
Write a definition for a process
%
\begin{scala}
def Merge(left: ?[Int], right: ?[Int], out: ![Int]) = proc ...
\end{scala}
%
that inputs two ascending non-empty streams of integers on \SCALA{left} and
\SCALA{right}, merges them into a single ascending stream, and outputs that on
\SCALA{out}.  The process should hold at most one value at a time from each of
the input streams.  You may assume that the input streams are never closed.
% You may not use \SCALA{ox.cso.Components.merge}.
\end{question}
%
\begin{answer}
%% When one of the input channels is closed, causing the program to break out of
%% the \SCALA{repeat} loop below, we need to know whether the values of \SCALA{l}
%% and \SCALA{r} have already been output; we use the flag \SCALA{invalid} for
%% this.
%
\begin{scala}
import ox.CSO._
import ox.cso.Components._

object MergeStreams{
  // Merge two non-empty ascending streams
  def Merge(left: ?[Int], right: ?[Int], out: ![Int]) 
  = proc("Merge"){
    // Invariant: l is last value read from left; 
    // r is last value read from right
    var l = left?; var r = right?;
    repeat{
      if(l<=r){ out!l; l=left? }
      else{ out!r; r=right? }
    }
    left.close; right.close; out.close;
  }

  val random = new scala.util.Random ;

  // Produce an ascending stream of n Ints on out
  def Producer(n:Int, out: ![Int]) = proc("Producer"){
    var current = 0;
    for(i <- 0 until n){ 
      current += random.nextInt(5); 
      println("producing "+current); 
      out!current;
    }
    out.close
  }

  val left = OneOne[Int]; 
  val right = OneOne[Int];
  val out = OneOne[Int];

  def System(n:Int) = (
    Merge(left, right, out) || Producer(n, left) 
    || Producer(n, right) || console(out) 
  )

  def main(args : Array[String]) = 
    System( 
      if(args.length>0)
        Integer.valueOf(args(0)).intValue()
      else 10 
    )();
}
\end{scala}
%
[Students should provide some sensible test results.]
\end{answer}
