\begin{question}
\def\A{\mbox{\SCALA{a}}} 
The aim of this question is to write concurrent
programs, using CSO, to update the $N$-by-$N$ array~\A\
%
%% \begin{scala}
%%   val A = new Array[Array[Int]](N,N);
%% \end{scala}
%
to hold values defined as follows:
%
\[
\begin{array}{rcll}
\A(0)(j) & = & j, & \mbox{for $0 \le j < N$,} \\
\A(i)(0) & = & \A(i-1)(0), & \mbox{for $1 \le i < N$,} \\
\A(i)(j) & = & f( \A(i-1)(j-1) ,\, \A(i-1)(j) ), &  
  \mbox{for $1 \le i < N$,\, $1 \le j < N$,}
\end{array}
\]
where $f$ is some given function. 

Note that you will need to ensure that your programs calculate the entries
of~\A\ in a suitable order.  
%
\begin{parts}
\part
Write a concurrent program that uses $N$ workers to solve this problem.  The
$j$th worker should be responsible for calculating $\A(i)(j)$ for $0 \le i <
N$.  Explain your design.  \marks{10}

\part
Write a concurrent program that uses $p$ worker processes, where $p<N$, to
solve this problem.  Your program should use the bag-of-tasks pattern, where
each task should involve calculating a \emph{single} entry in~\A.  Explain
your design.  \marks{15}
\end{parts}
% 
As always, you should aim to make your programs reasonably efficient.


%%%%%%%%%%%%%%%%%%%%%%%%%%%%%%%%%%%%%%%%%%%%%%%%%%%%%%%

\answer
% \begin{ansnote}
The difficulty is in ensuring that processes calculate values in a suitable
order.
%\end{ansnote}

\begin{parts}
\part
The solution below arranges for each worker $j$, for $j < N-1$, to signal on
the channel \SCALA{flag(j+1)} to worker $j+1$ after writing each value; the
$i$th such signal (counting from~0) indicates that $a(i)(j)$ has been written.
Worker~$j+1$ will calculate the next value only after receiving this signal.
We use buffered channels for efficiency. 
%
\begin{scala}
object Recurrence1{
  val N = 10
  val a = new Array[Array[Int]](N,N)

  val flag = Array.fill(N)(OneOneBuf[Unit](1))  // indexed by recipients 

  def f(m:Int, n:Int) = m+n  // for testing

  def worker(j:Int) = proc("Worker"+j){
    a(0)(j) = j
    for(i <- 1 until N){
      if(j < N-1) flag(j+1)!() // signal to j+1
      if(j > 0){
        flag(j)?() // wait for signal
        a(i)(j) = f(a(i-1)(j-1) , a(i-1)(j))
      } 
      else a(i)(j) = a(i-1)(j) // j = 0 doesn't need to wait
    }
  }

  def main(args: Array[String]) = { 
    run(|| (for(j <- 0 until N) yield worker(j)))
    for(j <- 0 until N) println(a(N-1)(j)) 
  }
}
\end{scala}

%%%%%

\part

In order to ensure that values are calculated in a suitable order, the
controller has an array of booleans indicating which values have been
calculated; tasks are added to its queue when the necessary prior tasks have
been completed.
%
\begin{scala}
object Recurrence2{
  val N = 10
  val a = new Array[Array[Int]](N,N)

  type Task = (Int,Int)
  // the task (i,j) tells the worker to calculate a(i)(j)

  val toWorkers = OneMany[Task] // channel for distributing tasks
  val done = ManyOne[Task] // channel to tell controller that task is done
  def f(m:Int, n:Int) = m+n // for testing

  def worker = proc{
    repeat{
      val(i,j) = toWorkers?()
      if(i == 0) a(i)(j) = j
      else if (j > 0) a(i)(j) = f(a(i-1)(j-1) , a(i-1)(j))
      else a(i)(j) = a(i-1)(j)
      done!(i,j)
    }
  }

  def controller = proc{
    val b = Array.fill[Boolean](N,N) // indicates which values are done; initialised to all false
    // queue stores the tasks that are ready to be done
    val queue = new scala.collection.mutable.Queue[Task]
    for(j <- 0 until N) queue += (0,j)
    var busyWorkers = 0 // number of busy workers

    // Main loop
    serve(
      (queue.nonEmpty && toWorkers) =!=> { busyWorkers += 1; queue.dequeue }
      | 
      (busyWorkers > 0 && done) =?=> { case (i,j) =>
        b(i)(j) = true; busyWorkers -= 1
        // add tasks to queue if possible
        if(i < N-1){
          if(j == 0 || b(i)(j-1)) 
            // (i,j-1) and (i,j) done, so can do task (i+1,j)
            queue += (i+1,j) 
          if(j < N-1 && b(i)(j+1))
            // (i,j) and (i,j+1) done so can do task (i+1,j+1)
            queue += (i+1,j+1) 
        }
      }
    )
    // Loop terminates when queue.isEmpty && busyWorkers = 0

    toWorkers.close
  }

  def main(args: Array[String]) = { 
    val p = 4 // number of workers
    val workers = || (for(j <- 0 until p) yield worker)
    run(workers || controller)
    for(j <- 0 until N) println(a(N-1)(j))
  }
}
\end{scala}
\end{parts}
\end{question}
 
