\begin{question}
Consider a single-lane road bridge, crossing a river from north to south.
Cars going in the same direction can cross the bridge simultaneously, but cars
going in different directions may not.  The aim of this question is to
implement classes, using CSO, to control the bridge, with the following
signature:
%
\begin{scala}
  class Bridge{
    /** Car arrives at north end. */
    def arriveNorth = {...}
    /** Car leaves at south end. */
    def leaveSouth = {...}
    /** Car arrives at south end. */
    def arriveSouth = {...}
    /** Car leaves at north end. */
    def leaveNorth = {...}
  }
\end{scala}
%
A car arriving at the north end of the bridge calls \SCALA{arriveNorth}; this
call should block until the car may enter the bridge.  When the car leaves the
south end of the bridge it calls \SCALA{leaveSouth}.  Cars travelling from
south to north call \SCALA{arriveSouth} and \SCALA{leaveNorth}, similarly.

The code should enforce both the safety condition that cars are never
simultaneously crossing the bridge in different directions, and also the
liveness condition that each car that calls one of the \SCALA{arrive}
procedures may eventually cross the bridge.  Hint: one way to achieve the
liveness condition is to not allow a car that arrives at the bridge to
immediately enter it if there is a car waiting at the other side.

You may find that your code for cars travelling from south to north
is very similar to the code for cars travelling from north to south.  If so,
you should give the code for one direction in full, but may omit the code for
the other direction.

The physical characteristics of the bridge will prevent one car from
overtaking another while on the bridge.  Your code does \emph{not} need to
enforce this. 

\begin{parts}
\part
Implement the \SCALA{Bridge} class by encapsulating a server process.
\marks{9}

\part
Now implement the \SCALA{Bridge} class as a monitor.
\marks{8}

\part
Finally implement the \SCALA{Bridge} class using semaphores.
\marks{8}
\end{parts}
%
Note that in each part you should use \emph{only} the concurrency primitive
mentioned.  You should explain your designs. 

%%%%%%%%%%%%%%%%%%%%%%%%%%%%%%%%%%%%%%%%%%%%%%%%%%%%%%%

\answer

Note: students are expected to elide about 45\% of the code below.

\begin{parts}
\part
Each call to an \SCALA{Arrive} procedure causes a reply channel to be sent to
the server.  The server sends a reply on this channel when the car should
proceed.
%
\begin{scala}
class BridgeServer{
  private type ReplyChan = OneOne[Unit]
  // channels to server
  private val arriveNorth,  arriveSouth = ManyOne[ReplyChan]
  private val leaveSouth, leaveNorth = ManyOne[Unit]

  def arriveNorth = {
    val c = OneOne[Unit]
    arriveNorth!c // register with server
    c?() // wait for acknowledgement to proceed
  }

  def leaveSouth = {
    leaveSouth!() // tell server we're done
  }

  def arriveSouth = {
    val c = OneOne[Unit]
    arriveSouth!c // register with server
    c?() // wait for acknowledgement to proceed
  }

  def leaveNorth = {
    leaveNorth!() // tell server we're done
  }

  private def server = proc{
    // Queues of ReplyChans for waiting cars
    val northQueue = new scala.collection.mutable.Queue[ReplyChan]
    val southQueue = new scala.collection.mutable.Queue[ReplyChan]
    var fromNorth = 0; var fromSouth = 0 // number on bridge
    // Invariant: fromNorth=0 or fromSouth=0

    serve(
      arriveNorth =?=> { c =>
        if(fromSouth > 0 || southQueue.nonEmpty) // need to wait
          northQueue += c
        else{ fromNorth += 1; c!() } // can go
      }
      | 
      arriveSouth =?=> { c =>
        if(fromNorth > 0 || northQueue.nonEmpty) // need to wait
          southQueue += c
        else{ fromSouth += 1; c!() } // can go
      }
      |
      leaveSouth =?=> { () => 
        fromNorth -= 1
        if(fromNorth == 0) // release cars queuing from south
          while(southQueue.nonEmpty){ southQueue.dequeue!(); fromSouth += 1 }
      }
      |
      leaveNorth =?=> { () =>
        fromSouth -= 1
        if(fromSouth == 0) // release cars queuing from south
          while(northQueue.nonEmpty){ northQueue.dequeue!(); fromNorth += 1 }
      }
    ) // end of serve
  }

  server.fork // fork off server process
}
\end{scala}

%%%%%

\part
Here's a version using a JVM monitor.
%
\begin{scala}
class BridgeMonitor{
  private val mon = new Monitor
  // Conditions for signalling that the route from south, resp. north, is clear.
  private val southClear, northClear = mon.newCondition
  
  private var fromNorth = 0; private var fromSouth = 0 // number on bridge
  private var waitingNorth = 0; private var waitingSouth = 0// numbers waiting

  def arriveNorth = monitor.withLock{
    if(fromSouth > 0 || waitingSouth > 0){ // need to wait
      waitingNorth+=1
      northClear.await
      assert(fromSouth == 0)
      waitingNorth-=1
    }
    // proceed
    fromNorth += 1
  }  

  def arriveSouth = monitor.withLock{
    if(fromNorth > 0 || waitingNorth > 0){ // need to wait
      waitingSouth+=1
      southClear.await
      assert(fromNorth == 0)
      waitingSouth-=1
    }
    // proceed
    fromSorth += 1
  }

  def leaveSouth = monitor.withLock{
    fromNorth -= 1
    if(fromNorth == 0 && waitingSouth > 0) southClear.signalAll()
  }

  def leaveNorth = monitor.withLock{
    fromSouth -= 1
    if(fromSouth == 0 && waitingNorth > 0) northClear.signalAll()
  }
}
\end{scala}
%
Note that once a waiting car~$c$ receives a signal, no car will enter from the
other direction until (at least) $c$ has left the bridge. 


%% \begin{scala}
%% class BridgeMonitor{

%%   private var fromNorth = 0; private var fromSouth = 0; // number on bridge
%%   private var waitingNorth = 0; private var waitingSouth = 0;// numbers waiting
%%   private val NORTH = 0; private val SOUTH = 1; private val ANY = -1;
%%   private var direction = ANY; // current direction of flow

%%   def ArriveNorth = synchronized{
%%     if(direction==NORTH || waitingSouth>0){ // need to wait
%%       waitingNorth+=1; 
%%       wait; while(direction==NORTH) wait;
%%       waitingNorth-=1;
%%     }
%%     // proceed
%%     fromNorth+=1; direction = SOUTH;
%%   }

%%   def ArriveSouth = synchronized{
%%     if(direction==SOUTH || waitingNorth>0){ // need to wait
%%       waitingSouth+=1; 
%%       wait; while(direction==SOUTH) wait;
%%       waitingSouth-=1;
%%     }
%%     // proceed
%%     fromSouth+=1; direction = NORTH;
%%   }

%%   def LeaveSouth = synchronized{
%%     fromNorth-=1;
%%     if(fromNorth==0){
%%       if(waitingSouth>0){  // wake up waiting processes
%%         direction = NORTH; notifyAll; 
%%       }
%%       else direction = ANY;
%%     }
%%   }

%%   def LeaveNorth = synchronized{
%%     fromSouth-=1;
%%     if(fromSouth==0){
%%       if(waitingNorth>0){  // wake up waiting processes
%%         direction = SOUTH; notifyAll; 
%%       }
%%       else direction = ANY;
%%     }
%%   }
%% }
%% \end{scala}

%% Note: the code needs to be careful to avoid waking up cars going in the wrong
%% direction.  

%%%%%

\part
\begin{scala}
class BridgeSemaphore{

  private var fromNorth = 0; private var fromSouth = 0 // number on bridge
  private var waitingNorth = 0; private var waitingSouth = 0// numbers waiting
  private val mutex = MutexSemaphore() // for mutual exclusion
  // Semaphores to signal to cars waiting at each end
  private val north = SignallingSemaphore()
  private val south = SignallingSemaphore()

  def arriveNorth = {
    mutex.down
    if(fromSouth > 0 || waitingSouth > 0){ // need to wait
      waitingNorth += 1; mutex.up; north.down // wait for signal
      waitingNorth -= 1
    }
    fromNorth+=1
    signalNorth // maybe signal to next car at this end
  }

  // Signal to cars waiting in the north, if any
  private def signalNorth = if(waitingNorth > 0) north.up else mutex.up

  def arriveSouth = {
    mutex.down
    if(fromNorth > 0 || waitingNorth > 0){ // need to wait
      waitingSouth += 1; mutex.up; south.down // wait for signal
      waitingSouth -= 1
    }
    fromSouth+=1
    signalSouth
  }

  // Signal to cars waiting in the south, if any
  private def signalSouth = if(waitingSouth > 0) south.up else mutex.up

  def leaveSouth = {
    mutex.down
    fromNorth -= 1
    if(fromNorth == 0) signalSouth else mutex.up
  }

  def leaveNorth = {
    mutex.down
    fromSouth -= 1
    if(fromSouth == 0) signalNorth else mutex.up
  }
}
\end{scala}
\end{parts}
\end{question}
