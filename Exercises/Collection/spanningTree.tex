\begin{question}
Consider a collection of nodes that are connected together in a directed
graph.  Each node has a variable
%
\begin{scala}
  neighbours: List[NodeId]
\end{scala}
listing its neighbours in the graph, where
\begin{scala}
  type NodeId = Int
\end{scala}
is the type of node identifiers.  

If Node~\SCALA{n1} has a neighbour~\SCALA{n2}, then \SCALA{n1} can send
\SCALA{n2} messages on the channel \SCALA{toNode(n2)}, and \SCALA{n2} can send
messages back on \SCALA{toNode(n1)}.  However, nodes can send messages only to
their neighbours. 

The aim of this question is to produce concurrent programs, using CSO, to find
a spanning tree of the graph.  For each part, you should give a brief
description of your design.  You should state any assumptions you make about
the \SCALA{toNode} channels.

\begin{parts}
\part\label{part:a}
Write a concurrent program to find a spanning tree.  The protocol should be
initiated by Node~0, which should become the root of the spanning tree.  Each
node except Node~0 should end with a variable \SCALA{parent} holding the
identity of its parent in the tree.  Node 0 should start by sending a message
to each of its neighbours.  The first time that a node receives a message, it
should accept the sender of the message as its parent, and should send a
message on to all of its other neighbours.  \marks{7}

\part
Now adapt your answer to part~(\ref{part:a}) so that, in addition, each
process ends up with a variable \SCALA{children: List[NodeId]} that holds the
identities of its children, in some order.  Further, the spanning tree should
be constructed by exploring the graph in a depth-first order.  \marks{12}

\part
Now describe how to adapt the program so that the spanning tree is constructed
by exploring the graph in a \emph{breadth-first} order.  Code is not required,
but you should give a clear explanation of how to achieve this goal.  
\marks{6}
\end{parts}


%%%%%%%%%%%%%%%%%%%%%%%%%%%%%%%%%%%%%%%%%%%%%%%%%%%%%%%

\answer

\begin{parts}
\part
Node~0 sends a \SCALA{Req} request message to each of its neighbours.  The
first time that a node receives a message, it accepts the sender of the
message as its parent, and sends a \SCALA{Req} message on to all of its other
neighbours.

Here's a complete program, although only about the first half is required.  
%
\begin{scala}
object SpanningTree0{

  val N = 10 // number of nodes
  type NodeId = Int // identities of nodes

  abstract class Msg
  case class Req(n: NodeId) extends Msg
    // suggestion from potential parent n

  val toNode = Array.fill(N)(OneOneBuf[Msg](N-1)) // channels to each node

  def node(me: Int, neighbours: List[NodeId]) = proc{
    var parent = 0

    if(me != 0){ // Wait to receive notification from parent
      toNode(me)?() match{ case Req(p) => parent = p }
    }
    println("Node "+me+" chooses parent "+parent)

    // Send out request to potential children
    for(n <- neighbours; if n != parent){ toNode(n)!Req(me) }

    // There's no real need to consume other requests here
  }

  /** Generate neighbours randomly. */
  val allNeighbours = new Array[List[NodeId]](N)
  def generateNeighbours = {
    // Generate links randomly
    val links = new Array[Array[Boolean]](N,N)
    for(i <- 0 until N; j <- i+1 until N){
      if(scala.util.Random.nextDouble < 0.5){ 
        links(i)(j) = true; links(j)(i) = true 
      }
    }
    // Now build lists of neighbours
    for(i <- 0 until N){
      allNeighbours(i) = List[NodeId]()
      for(j <- 0 until N) if (i != j && links(i)(j)) allNeighbours(i) ::= j
    }
  } 

  def main(args:Array[String]) = {
    generateNeighbours
    for(i <- 0 until N) println("neighbours("+i+") = "+allNeighbours(i))
    run( || (for (i <- 0 until N) yield node(i, allNeighbours(i))) )
  }
}
\end{scala}

%%%%%

\part
When a node first receives a \SCALA{Req} message, it sends \SCALA{Req}
messages to each of its neighbours in turn, waiting for a response from each
before sending to the next; it then sends a response back to its parent.  Each
\SCALA{Req} message is responded to with a \SCALA{Resp} message, than
indicates whether the sender of the \SCALA{Resp} has accepted the sender of
the \SCALA{Req} as its parent.  Note that a node needs to reply to subsequent
\SCALA{Req} messages negatively, until it has sent a \SCALA{Resp} to each of
its neighbours. 

\begin{scala}
  abstract class Msg
  case class Req(n: NodeId) extends Msg
    // suggestion from potential parent n
  case class Resp(n: NodeId, ok: Boolean) extends Msg
    // response, either positive or negative, from node n

  val toNode = Array.fill(ManyOne[Msg]) // channels to each node

  def Node(me:Int, neighbours:List[NodeId]) = proc{
    var parent = 0

    if(me != 0){ // Wait to receive notification from parent
      toNode(me)?() match{ case Req(p) => parent = p }
    }
    println("Node "+me+" chooses parent "+parent)

    var children = List[NodeId]()
    var reqs = if(me == 0) 0 else 1 // number of requests so far

    // Send out requests to potential children, and wait for reply
    for(n <- neighbours; if n != parent){ 
      toNode(n)!Req(me)
      var done = false
      while(!done){
        toNode(me)? match{ 
          case Req(n1) =>  // refuse new invitations
            reqs += 1; toNode(n1)!Resp(me, false)
          case Resp(n1,b) => 
            assert(n1 == n); done = true // received response
            if(b) children ::= n  // add to list of children if positive response
        } // end of match
      } // end of while
    } // end of for
    
    println("Node "+me+" has children "+children)

    if(me != 0) toNode(parent)!Resp(me,true) // send agreement to parent

    // Consume outstanding requests
    val len = neighbours.length-children.length // # expected reqs
    while(reqs < len){
      toNode(me)? match{ 
        case Req(n1) =>  // refuse new invitations
          reqs += 1; toNode(n1)!Resp(me, false)
      }
    }
  }
\end{scala}
%
Most students will fail to consume outstanding \SCALA{Req} messages at the
end, which will normally lead to deadlock [3 marks for this bit].

%%%%%

\part
The protocol proceeds in rounds, with the $n$th round exploring the graph to a
depth of~$n$.  The first time a node receives a \SCALA{Req}, it just sends
back a response, accepting its parent.  The next time it receives a
\SCALA{Req}, it sends a \SCALA{Req} on to all its other neighbours, to find if
they will be its children.  In order to achieve termination, the responses
back up the tree have to include a field indicating whether the search has
bottomed-out.  Perhaps the easiest way to implement this is for the root node
to detect when all branches have bottomed out, and to send a final
``terminate'' signal down the tree.  [A better way is for a node whose
  sub-tree has bottomed-out to terminate, and for its parent to not
  subsequently send it messages; with this approach, nodes also need to avoid
  sending messages to neighbours from whom they've previously received
  invitations that they've rejected (in case those nodes have terminated).]

%% Code isn't required, but here it is, anyway.
%% %
%% \begin{scala}
%%   abstract class Msg;
%%   case class Req(n:NodeId) extends Msg; 
%%     // suggestion from potential parent n
%%   case class Resp(n:NodeId, ok:Boolean, done:Boolean) extends Msg; 
%%     // response from node n; ok will signify acceptance or rejection of
%%     // invitation; done will indicate whether this sub-tree has
%%     // bottomed-out.

%%   val toNode = ManyOne[Msg](N); // channels to each node

%%   def Node(me:Int, neighbours:List[NodeId]) = proc{
%%     var children : List[NodeId] = Nil;
%%     var done = false;
%%     var childDone = new Array[Boolean](N); // is each child finished?

%%     if(me==0){
%%       // send initial request to children; should get back positive responses
%%       for(n <- neighbours; if !childDone(n)){
%%         toNode(n)!Req(me); 
%%         toNode(me)? match{ 
%%           case Resp(n1,b,d) => { assert(n1==n && b && !d); children ::= n; }
%%         } // end of match
%%       } // end of for
%%       println("Node "+me+" has children "+children);

%%       // Now continue to send out commands to explore graph one level deeper
%%       while(!done){
%%         done = true; // have all neighbours signalled willingness to terminate?
%%         for(n <- neighbours; if !childDone(n)){
%%           toNode(n)!Req(me); 
%%           toNode(me)? match{ 
%%             case Resp(n1,b,d) => { 
%%               assert(n1==n && b); done &&= d; childDone(n) = d;
%%             }
%%             case Req(n1) => Resp(me,false,true);
%%           } // end of match
%%         } // end of for
%%       } // end of while
%%     } // end of if(me==0)

%%     else {
%%       var parent = 0;
%%       toNode(me)? match{ case Req(p) => { parent = p; } }
%%       toNode(parent)!Resp(me,true,false);                       
%%       println("Node "+me+" chooses parent "+parent);

%%       var firstRound = true; // is this the first round of pooling children
%%       // repeatedly receive requests to explore graph one level deeper
%%       while(!done){
%%         toNode(me)? match{
%%           case Req(from) => {
%%             if(from==parent){
%%               done = true;
%%               // pass on to all neighbours
%%               for(n <- neighbours; if n!=parent && !childDone(n)){ 
%%                 toNode(n)!Req(me); 
%%                 // wait for reply
%%                 var replyRecd = false
%%                 while(!replyRecd){
%%                   toNode(me)? match{ 
%%                     case Req(n1) => { // refuse new invitations
%%                       toNode(n1)!Resp(me,false,true); 
%%                       childDone(from) = true; // never invite this node
%%                     }
%%                     case Resp(n1,b,d) => { // reply received
%%                       assert(n1==n); done &&= d; replyRecd = true; 
%%                       if(firstRound && b){ children ::= n; assert(!d); }
%%                       childDone(n) = d; 
%%                     }
%%                   } // end of match
%%                 } // end of while(!replyRecd)
%%               } // end of for
%%               toNode(parent)!Resp(me, true, done);
%%               if(firstRound){
%%                 println("Node "+me+" has children "+children);
%%                 firstRound = false;
%%               }
%%             } // end of if(from==parent)
%%             else{
%%               toNode(from)!Resp(me,false,true); // refuse new invitations
%%               childDone(from) = true; // never invite this node
%%             }
%%           } // end of case Req(from)
%%         } // end of match
%%       } // end of while(!done)
%%     } // end of else
%%   }
%% \end{scala}
\end{parts}
\end{question}
