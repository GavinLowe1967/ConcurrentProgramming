\begin{question}
\Programming\ 
%
Recall the producer-consumer buffer problem from
Question~\ref{Q:consumerM}.
% Consider the following producer-consumer problem.  Several producers each
% produce one piece of (integer) data at a time, which they place into a buffer
% using a method \SCALA{put(item: Int)}.  These are accumulated into an array of
% size $n$.  A consumer can receive that array, once it is full, using the
% method \SCALA{get : Array[Int]}.  
Implement the buffer using semaphores. 
% The consumer should be blocked until the array is full, and the
% producers should be blocked once the array is full.
Test your code.
\end{question}

%%%%%

\begin{answer}
The code below uses the technique of passing the baton.  The producer needs to
pass the baton to either another producer or the consumer, depending on
whether the buffer is now full.
%
\begin{scala}
class ProducerConsumerSemaphore(n: Int)  extends ProducerConsumer{
  require(n >= 1)

  /** Array to hold the data. */
  private var a = new Array[Int](n)

  /** Number of items added to a so far. */
  private var count = 0

  /** Semaphore to signal to a producer that the buffer is not full. */
  private val nonFull = MutexSemaphore()

  /** Semaphore to signal to a consumer that the buffer is full. */
  private val full = SignallingSemaphore()

  /** Add x to the buffer, waiting until it is not full. */
  def put(x: Int) = {
    nonFull.down
    assert(count < n)
    a(count) = x; count += 1
    if(count == n) full.up else nonFull.up // pass the baton appropriately
  }

  /** Get the contents of the buffer, once it is full. */
  def get: Array[Int] = {
    full.down
    assert(count == n)
    val result = a
    a = new Array[Int](n); count = 0
    nonFull.up // pass baton to a producer
    result
  }
}
\end{scala}

Testing can be done in the same way as for the similar question using monitors. 
\end{answer}
