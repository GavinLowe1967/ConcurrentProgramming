\begin{question}
\Programming\ Implement a bounded buffer, of size~$n$, using two counting
semaphores: one semaphore should prevent data from being added to the buffer
when it is full; the other should prevent data from being removed from the
buffer when it is empty.

Test your implementation using the linearizability framework. 
\end{question}

%%%%%

\begin{answer}
\begin{scala}
/** A queue implemented using a counting semaphore. */
class BoundedCountingSemaphorePartialQueue[T](n: Int) extends PartialQueue[T]{
  /** The queue itself. */
  private val queue = new scala.collection.mutable.Queue[T]

  /** Semaphore for enqueuing.  The state of the semaphore equals the number of
    * free spaces. */
  private val spaces = CountingSemaphore(n)

  /** Semaphore for dequeueing.  The state of the semaphore equals queue.length. */
  private val size = CountingSemaphore(0)

  /** Semaphore to provide mutual exclusion. */
  private val mutex = MutexSemaphore()

  def enqueue(v: T) = {
    spaces.down; mutex.down
    queue.enqueue(v)
    size.up; mutex.up
  }

  def dequeue: T = {
    size.down; mutex.down
    val result = queue.dequeue
    mutex.up; spaces.up
    result
  }
}
\end{scala}
%
Note that the order in which each thread acquires the two semaphores is
important.  If the |down|s where performed in the opposite order, the system
could become deadlocked, e.g.~if the queue is empty and a |dequeue| acquires
the |mutex| first.  Above, assuming no thread is permanently descheduled, a
thread will eventually proceed. 

This can be tested using the linearizability framework.  The only non-trivial
changes from the version in lectures is that both corresponding sequential
operations have preconditions.
\begin{scala}
  type SeqQueue = scala.collection.immutable.Queue[Int]
  def seqEnqueue(x: Int)(q: SeqQueue) : (Unit, SeqQueue) = {
    require(q.length < n); ((), q.enqueue(x))
  }
  def seqDequeue(q: SeqQueue) : (Int, SeqQueue) = {
    require(q.nonEmpty); q.dequeue
  }
\end{scala}
%
To avoid deadlocks, it's best to run this in a setting where half the workers
perform enqueues and half perform dequeues.  In order to obtain good coverage
of the tests, in particular to cover the case of the queue becoming full, some
tests should use a small value for $n$ (e.g.~by choosing $n$ as a random
number between 1 and 10).
\end{answer}
