\begin{question}
In the bag-of-tasks pattern, there is normally a single process that holds the
outstanding tasks.  However, that process can then become a bottle-neck.  The
aim of this question is to investigate an alternative, where each node holds
its own bag of tasks, but there is no central controller; i.e.,\ the bag of
tasks is distributed between the nodes.
%
More precisely, each node is composed of two subprocesses, one implementing
the bag, and the other working on data from the bag:
%
\begin{scala}
  def Bag(me: Int, bagToWorker: ![Data], workerToBag: ?[Data], 
          get: ?[Data], give: ![Data], done: ?[Unit]) = ...

  def Worker(me: Int, workerToBag: ![Data], 
             bagToWorker: ?[Data], done: ![Unit]) = ...
\end{scala}
%
The \SCALA{Worker} repeatedly gets a task from its bag (on channel
\SCALA{bagToWorker}) and, as a result, puts zero or more tasks back into the
bag (via channel \SCALA{workerToBag}), and then signals that it has completed
that task (on channel \SCALA{done}).  You are \emph{not} asked to provide an
implementation of \SCALA{Worker} for this question.

One way to implement the distributed bag of tasks involves arranging the nodes
in a (logical) ring.  The \SCALA{Bag} can pass data to its clockwise neighbour
on the channel \SCALA{give}.  It can accept a task from its anti-clockwise
neighbour on the channel \SCALA{get}, but only when it has no tasks in its
own bag.

\begin{enumerate}
\item
Give an implementation of the \SCALA{Bag} process.  You do not need to
consider termination for this part of the question.           \marks{15}

\item
Now explain how your answer to the previous part can be extended so that the
system terminates when there are no outstanding tasks.  You should give a
precise description of your design, but need not produce code.  \marks{10}
\end{enumerate}
%% You should implement a system using the distributed-bag-of-tasks pattern to
%% solve some combinatorial puzzle.  For example, you could pick the \emph{Eight
%%   Queens Problem} (see \url{http://en.wikipedia.org/wiki/Eight_queens}).
%% However, you should not pick the \emph{Magic Squares} problem that was
%% considered in lectures.  (The main point of this question is the
%% distributed-bag-of-tasks pattern, so you should pick a puzzle with simple
%% rules; also avoid a puzzle that will take too long to solve.)  If you pick a
%% puzzle other than the Eight Queens Problem, please include a brief description
%% of the rules (this part of your report may be copied from a source, with
%% citation).

%% One issue you will need to address is termination: the system should terminate
%% when the puzzle if fully solved.  (This part is quite tricky; however, you can
%% still obtain a good mark if you do not achieve this.)

%% Your answer should include a description of your design and an explanation of
%% any interesting features.  In particular, you should explain the termination
%% protocol.

%% \marks{40}
\end{question}

%%%%%


\begin{answer} \mbox{ }
{\bf Part a.}
%
\begin{scala}
  def Bag(me: Int, get: ?[Data], give: ![Data], 
          bagToWorker: ![Data], workerToBag: ?[Data]) 
  = proc("Bag"+me){
    // Store the tasks in a stack
    val stack = new scala.collection.mutable.Stack[Partial];
    /* Initialise stack in some way */

    var workerBusy = false; // Is the local worker busy?
 
    // Main loop
    while(true){
      if(!workerBusy)
        if(!stack.isEmpty){ // Send task to worker
	  val p = stack.pop; 
	  prialt(
	    toWorker -!-> { toWorker!p; workerBusy = true; }
	    | give -!-> { give!p; }
	  )
	} 
	else{ val p = get?; toWorker!p; workerBusy = true; }
      else // workerBusy
        prialt(
	  (!stack.isEmpty &&& give) -!-> { give!(stack.pop); }
	  | fromWorker -?-> { 
              val p = fromWorker?; stack.push(p); 
            }
	  | done -?-> { done?; workerBusy = false; }
	)
      }
    }
\end{scala}
%
The priorities are chosen so as try to speed up distribution of tasks.
[3 marks for some discussion of priorities.]

%%%%%

\noindent
{\bf Part b.}
%
We arrange to send messages corresponding to the termination protocol on the
\SCALA{get}/\SCALA{give} channels.  This is best done by changing the type of
those channels to some abstract type \SCALA{Msg}, with two subtypes:
\SCALA{Data} (as above) and \SCALA{Terminating} (for tokens in the termination
protocol). 

The termination protocol is initiated by Bag~0 when its stack is empty and its
worker is idle.  It sends a \SCALA{Terminating} token round the ring, on the
same channels as for the passing of partial solutions.  Each subsequent node
receives the token only when it is idle (the only time it receives on
\SCALA{Get}), and passes it on.  If node~0 receives the token back and it has
been idle ever since starting the termination protocol, then the whole system
can terminate: it circulates the token again, with a flag set, to indicate
this.  If it becomes busy after initiating the termination protocol, it has to
wait until it receives the token back before it re-tries the termination
protocol.  Therefore this node has to keep track of whether it is: not
currently involved in the termination protocol; involved in the termination
protocol; or has aborted the termination protocol and is waiting to receive
the token back.

Code is not required, but the code below illustrates this, with an application
to solve the Eight Queens Problem.  [Note to tutors: I suggest you give this
  to students to study in their own time.]
%
The class \SCALA{Partial} (replacing \SCALA{Data}) represents a partial
solution to the puzzle.  See comments.

\JavaSize{\footnotesize}
\begin{scala}
// The 8 queens problem

import ox.CSO._;

abstract class Msg;

// Class to represent a partial solution

case class Partial(N:Int) extends Msg{
  // Represent partial solution by a list of Ints; the ith 
  // entry represents the row number of the queen in column i 
  // (0 <= i < len).  len is the length of board
  private var board : List[Int] = Nil;
  private var len = 0;

  def finished : Boolean = (len==N);

  // Is it legal to play in column len, row j?
  private def isLegal(j : Int) = {
    // is piece (i1,j1) on different diagonal from (len,j)?
    def otherDiag(p:(Int,Int)) = { 
      val i1=p._1; val j1 = p._2; i1-len!=j1-j && i1-len!=j-j1; 
    };
    
    (board forall ((j1:Int) => j1 != j)) && 
      // row j not already used
    (List.range(0,len) zip board forall otherDiag) 
      // diagonals not used
  }

  // Return new partial resulting from playing in row j next
  private def doMove(j : Int) : Partial = {
    val newPartial = new Partial(N);
    newPartial.board = this.board ::: (j :: Nil);
    newPartial.len = this.len+1;
    return newPartial;
  }

  // Return list of successors, such that every solution to 
  // this is a solution of one of the successors
  def successors : List[Partial] = 
    for(j <- List.range(0, N) if(isLegal(j))) yield doMove(j);

  override def toString : String = {
    var st = "";
    for(i <- 0 until len) st = st + (i,board(i))+"\t";
    return st;
  }
}

// Class of messages used in testing for termination

case class Terminating(status:Boolean) extends Msg{ }
// Status of false means this is a probe if we can terminate; 
// status of true means terminate now

// The main object
object EightQueens{
  val N = 8; // size of board

  // A worker.  This node can get a partial solution from its 
  // anticlockwise neighbour on channel get; it can give a
  // partial solution to its clockwise neighbour on channel give
  def Node(me: Int, get: ?[Msg], give: ![Msg])
  = proc("Node"+me){

    // Process to maintain this bag of tasks
    def Bag(toWorker: ![Partial], fromWorker: ?[Partial], 
            done: ?[Unit]) 
    = proc("Bag"+me){
      // Store the tasks in a stack
      val stack = new scala.collection.mutable.Stack[Partial];
      if(me==0)	stack.push(new Partial(N)); // Starting position

      var workerBusy = false; // Is the local worker busy?
      var finished = false; // Becomes true when we can terminate

      // Status of termination protocol, node 0 only
      val NONTERM = 0; val TRYING = 1; val ABORTED = 2;
      var terminationStatus = NONTERM; 
      // NONTERM means not currently involved in termination 
      // protocol; TRYING means termination protocol started, 
      // and this node has been idle since then; ABORTED means 
      // termination protocol was started, and this node then 
      // became busy, but the protocol hasn't yet terminated. 
 
      // Main loop
      while(!finished){
	if(!workerBusy)
	  if(!stack.isEmpty){ // Send task to worker
	    val p = stack.pop; 
	    prialt(
	      toWorker -!-> { toWorker!p; workerBusy = true; }
	      | give -!-> { give!p; }
	    )
	  } 
	  else{ // Idle, so wait for message from neighbour
	    if(me==0 && terminationStatus==NONTERM){
	      // Initiate termination protocol
	      println("Termination protocol starting");
	      give!Terminating(false);
	      terminationStatus = TRYING;
	    }
	    val m:Msg = get?;
	    m match {
	      case Partial(n) => {
		toWorker!m.asInstanceOf[Partial]; 
		workerBusy = true;
		if(me==0 && terminationStatus==TRYING){
		  println("Termination protocol aborted");
		  terminationStatus = ABORTED;
		}
	      }
	      case Terminating(s) => {
		if(me==0){
		  if(terminationStatus==TRYING){
		    // We've received nothing since sending the 
		    // termination probe, so we can terminate
		    give!Terminating(true); 
                      // pass signal to next node
		    toWorker.close; // Tell worker to close
		    get?; // Receive termination signal back
		    finished = true; // Can terminate
		  }
		  else if(terminationStatus==ABORTED){
		    terminationStatus = NONTERM;
		    println("Ready to retry termination protocol");
		  }
		}
		else{ // me!=0 
		  give!m; // pass signal on
		  if(s){ finished = true; toWorker.close; } 
                    // terminating
		}
	      } // end of case Terminating(s)
	    } // end of match
	  }
	else // workerBusy
	  prialt(
	    (!stack.isEmpty &&& give) -!-> { give!(stack.pop); }
	    | fromWorker -?-> { 
                val p = fromWorker?; stack.push(p); 
              }
	    | done -?-> { done?; workerBusy = false; }
	  )
      }
    }

    // Process to work on partial solutions
    def Worker(toBag: ![Partial], fromBag: ?[Partial], 
               done: ![Unit])
    = proc("Worker"+me){
      repeat{
	val partial = fromBag? ; // get job
	if(partial.finished){ println(partial); } // done!
	else // Generate all next-states
	  for(p1 <- partial.successors) toBag!p1; 
	// sleep(10); 
	done!();
      }
    }

    // Put this node together
    val bagToWorker, workerToBag = OneOne[Partial];
    val done = OneOne[Unit];
    def ThisNode = (
      Bag(bagToWorker, workerToBag, done) || 
      Worker(workerToBag, bagToWorker, done)
    );
    ThisNode();
  }

  val p = 8; // Number of workers

  // Put system together

  val passPartial = OneOne[Msg](p); // indexed by recipient's id

  def System = 
    || ( for(i <- 0 until p) yield 
           Node(i, passPartial(i), passPartial((i+1)%p)) )

  def main(args: Array[String]) = System();

}
\end{scala}
\end{answer}
