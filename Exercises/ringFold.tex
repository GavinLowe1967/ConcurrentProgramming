\begin{question}
\Programming\ Suppose $n$ threads are connected in a ring.  Each process~$i$
(for $i = 0, \ldots, n-1$) initially holds a value~$x_i: T$.  Write a
concurrent program so that each thread ends up in possession of the value
\[
f(f(f(\ldots f(x_0, x_1), x_2) \ldots ), x_{n-1}),
\]
for some given function $f$; or using functional programming notation:
\[
\mathrm{foldl1}~f~ List(x_0, x_1, \ldots, x_{n-1})
\]
Give your program the following signature
%
\begin{scala}
class RingFold[T](xs: Array[T], f: (T,T) => T, outs: Array[Chan[T]]){
  def apply(): PROC = ...
}
\end{scala}
%
Each thread should output its final result on the appropriate |out| channel.
You should state properties that hold at key points of your program, for
example concerning the values passed on channels.

Test your code. 

What property of~$f$ will allow a different program that achieves the same
goal?  Explain your answer, and outline how this can be achieved.  (A full
implementation is not necessary.)
\end{question}

%%%%%

\begin{answer}
\Small
This is a polymorphic generalisation of an example seen in lectures.

A single token is passed twice round the ring.  Write $xs$ for the list of
all~$x$ values, and use (hopefully) obvious notation for subsequences.  On the first cycle the desired
value is built up: the value sent by node~$me$ equals
\[
\mathrm{foldl1}~f~ xs[0..me] = 
  f(f(f(\ldots f(x_0, x_1), x_2) \ldots ), x_{me}),
\]
This is easily proven by induction.  Hence node~0 receives back the desired
result, which then gets past round the ring on the second cycle.

My code is below.  For testing purposes, I've arranged for each node to output
its final value on one of the |outs| channels. 
%
\begin{scala}
/** A ring of node threads.  The ith node holds the value xs(i).  Each ends
  * up with the value of foldleft f xs.  Node i outputs its final value on
  * outs(i).  */
class RingFold[T](xs: Array[T], f: (T,T) => T, outs: Array[Chan[T]]){
  private val n = xs.length
  require(n >= 2 && outs.length == n)

  /** The channels connecting nodes, indexed by the recipient's identity:
    * node(i) receives on chans(i) and sends on chans((i+1)%n). */
  private val chans = Array.fill(n)(OneOne[T]) 

  /** A single node. */
  private def node[T](me: Int) = proc{
    val left = chans(me); val right = chans((me+1)%n)
    val out = outs(me); val x = xs(me)
    if(me == 0){
      right!x // Start things going; = foldl1 f xs[0..me]
      val result = left?() // Receive final result; = foldl1 f xs
      right!result  // Pass it on
      left?()       // Receive it back at the end
      out!result
    }
    else{
      val y = left?() // y = foldl1 f xs[0..me-1]
      right!f(y,x)    // = foldl1 f xs[0..me]
      val result = left?(); right!result // Receive final result and pass it on
      out!result
    }
  }  

  /** The complete ring. */
  def apply(): PROC = || (for(i <- 0 until n) yield node(i))
}
\end{scala}

Most of my testing code is below.  (This also works with the solution for the
final part of the program.)  A subsidiary |checker| process receives all the
final values, and checks that they are as expected.
%  
\begin{scala}
  /** A process that expects to receive expected on each channel in chans, and
    * throws an exception if an incorrect value is received. */
  def checker(chans: Array[Chan[Int]], expected: Int) = proc{
    for(chan <- chans){ val res = chan?(); assert(res == expected) }
  }

  /** A single test of either RingFold or RingFold1, using random values.
    * @param assoc use RingFold1 if true */
  def doTest(assoc: Boolean) = {
    val n = 2+Random.nextInt(20)
    val xs = Array.fill(n)(Random.nextInt(100))
    val outs = Array.fill(n)(OneOne[Int])
    def f(x: Int, y: Int) = if(assoc) x+y else 2*x+y
    val rf: PROC =
      if(assoc) new RingFold1[Int](xs, f, outs)()
      else new RingFold[Int](xs, f, outs)()
    (rf || checker(outs, xs.foldLeft(0)(f)))()
  }
\end{scala}


If $f$ is associative and commutative, then a different protocol is possible.
On the first round, each node sends its initial value to its neighbour.  On
each subsequent round, each node applies $f$ to the value it receives and its
own initial value, and passes the result on.  On each round~$r$ (counting
from~0), every node~$me$ sends the value
\[
foldl1~f~xs[me-r..me] =
  f(f(f(\ldots f(x_{me-r}, x_{me-r+1}), x_{me-r+2}) \ldots ), x_{me}),
\]
to node~$me+1$ (where all indices are interpreted mod $n$).  After $n$
rounds, all nodes have the desired final value.  The channels need to be
buffered to avoid deadlocks, obviously.  The changes from the previous code
are below.  
\begin{scala}
  /** The channels connecting nodes, indexed by the recipient's identity:
    * node(i) receives on chans(i) and sends on chans((i+1)%n).  The channels
    * need to be buffered. */
  private val chans = Array.fill(n)(OneOneBuf[T](1)) 

  /** A single node. */
  private def node[T](me: Int) = proc{
    val left = chans(me); val right = chans((me+1)%n)
    val out = outs(me); val x = xs(me); var y = x
    // Inv: at the start of round r, y = foldl1 f xs[me-r..me], where the indices
    // are interpreted mod n. 
    for(r <- 0 until n-1){
      right!y         // = foldl1 f xs[me-r..me]
      val z = left?() // = foldl1 f xs[me-1-r..me-1]
      y = f(z, x)     // = foldl1 f xs[me-(r+1)..me], maintaining invariant
    }
    // y = foldl1 f xs[me-(n-1)..me] = foldl1 f xs since f is AC.
    out!y
  }  
\end{scala}
%
The comment concerning |z| is justified by the fact that this value was sent
by node~|me-1|, which is running the same protocol, and must have sent this
value on its round |r| (since the channels are FIFO). 
\end{answer}
