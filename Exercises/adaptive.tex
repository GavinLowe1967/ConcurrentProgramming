\begin{question}
\def\integral#1#2{\int_{#1}^{#2} f(x)\,\mbox{d}x} 
%
\Programming\ Recall the trapezium rule, used for estimating integrals,
discussed in the third chapter.  \emph{Adaptive quadrature} is an
alternative approach that proceeds as follows.  In order to calculate
$\integral{a}{b}$, first compute the midpoint $mid = (a+b)/2$.  Then estimate
three integrals, $\integral{a}{mid}$, $\integral{mid}{b}$ and
$\integral{a}{b}$, each using the trapezium rule with a single interval.  If
the sum of the former two estimates is within some value $\epsilon$ of the
third, then we take that third estimate as being the result.  Otherwise,
recursively estimate the integrals over the two ranges $a$ to~$mid$ and $mid$
to~$b$, and sum the results.  The following sequential algorithm captures this
idea:
%
\begin{scala}
  /** Estimate the integral of f from a to b using adaptive quadrature. */
  def estimate(f: Double => Double, a: Double, b: Double) : Double = {
    val mid = (a+b)/2.0
    val fa = f(a); val fb = f(b); val fmid = f(mid)
    val lArea = (fa+fmid)*(mid-a)/2; val rArea = (fmid+fb)*(b-mid)/2
    val area = (fa+fb)*(b-a)/2
    if (Math.abs(lArea+rArea-area) < Epsilon) area
    else estimate(f,a,mid) + estimate(f,mid,b)
  }
\end{scala}

Write a concurrent program to implement adaptive quadrature.  Give your
program the following signature:
\begin{scala}
class Adaptive(f: Double => Double, a: Double, b: Double, Epsilon: Double, nWorkers: Int){
  require(a <= b)

  def apply(): Double = ...
}
\end{scala}
%
The program should use a bag of tasks with replacement: the two recursive
calls in the sequential version can be implemented by returning tasks to the
bag.  The bag of tasks needs to store the tasks still to be performed: use
either a |Queue| or a |Stack| for this.  You will have to think carefully
about when the system should terminate.

A test harness for this program is on the course website (making use of the
class |TrapeziumTest| from lectures).  Use this to test your code. 
%%%%% /users/gavinl/Teaching/CP/Scala/Trapezium.AdaptiveTest.scala

Suggest ways in which your program can be made more efficient; optional:
implement them.
\end{question}

%%%%%

\begin{answer}
%% We will build a system as illustrated below.
%% %
%% \begin{center}
%% \begin{tikzpicture}
%% \draw(0,0) node[draw] (bag) {\scalashape bag};
%% \draw (3,0) node[draw] (worker) {\scalashape workers};
%% \draw 
%% %% \draw (worker.north east)++(0.2,0.2) node (control1) {};
%% %% \draw (worker.south east)++(0.2,0.2) node (control2) {};
%% %% \draw (worker.north west)++(0.2,0) -- ++ (0,0.2) -- (control1) --
%% %% (worker.south east)++(0.2,0.2); %(control2);
%% \end{tikzpicture}
%% \end{center}
My code is below.  To support termination, the bag keeps track of the
number of workers currently working on a task; a worker informs the
bag that it has completed a task via the channel |done|.  When the bag
is empty and there are no busy workers, the bag closes the channels to
terminate the system.
%
\begin{scala}
/** Calculating integral, using trapezium rule, adaptive quadrature, and bag
  * of tasks pattern. */
class Adaptive(f: Double => Double, a: Double, b: Double, Epsilon: Double, nWorkers: Int){
  require(a <= b)

  /** A task is an interval on which to work. */
  private type Task = (Double, Double)

  /** Channel from the controller to the workers, to distribute tasks.  We
    * create the channel freshly for each run of the system. */
  private val toWorkers = OneMany[Task] 

  /** Channel from the workers to the bag, to return subtasks. */
  private val toBag = ManyOne[(Task,Task)]

  /** Channel from the workers to the adder process, to add up subresults. */
  private val toAdder = ManyOne[Double]

  /** Channel to indicate to the bag that a worker has completed a task. */
  private val done = ManyOne[Unit]

  /** A client, who receives arguments from the server, either estimates the
    * integral directly or returns new tasks to the bag. */
  private def worker = proc("worker"){
    repeat{
      val (a,b) = toWorkers?()
      val mid = (a+b)/2.0
      val fa = f(a); val fb = f(b); val fmid = f(mid)
      val lArea = (fa+fmid)*(mid-a)/2; val rArea = (fmid+fb)*(b-mid)/2
      val area = (fa+fb)*(b-a)/2
      if (Math.abs(lArea+rArea-area) < Epsilon) toAdder!area
      else toBag!((a,mid), (mid,b)) 
      done!(())
    }
  }

  /** The bag, that keeps track of jobs pending. */
  private def bag = proc("bag"){
    val stack = new scala.collection.mutable.Stack[Task]
    stack.push((a,b))
    var busyWorkers = 0 // # workers with tasks

    serve(
      (stack.nonEmpty && toWorkers) =!=> { busyWorkers += 1; stack.pop }
      | (busyWorkers > 0 && toBag) =?=> { case (t1,t2) =>
          stack.push(t1); stack.push(t2) }
      | (busyWorkers > 0 && done) =?=> { _ => busyWorkers -= 1 }
    )
    // busyWorkers == 0 && stack.isEmpty
    toWorkers.close; toAdder.close
  }

  private var result = 0.0

  // Process to receive results from workers and add up the results
  private def adder = proc("adder"){ repeat{ result += (toAdder?()) } }

  def apply(): Double = {
    val workers = || (for (i <- 0 until nWorkers) yield worker)
    run(workers || bag || adder)
    result
  }
}
\end{scala}

%% Testing can be done in a very similar way to as in lectures, by testing
%% against a sequential implementation.  The main part of my code is below.  The
%% function |pickParams| picks parameters.  The function |estimate| is as in the
%% question.
%% %
%% \begin{scala}
%%   /** Do a single test. */
%%   def doTest = {
%%     val (f, p, a, b, nWorkers) = pickParams
%%     val seqResult = estimate(f, a, b)
%%     val concResult = new Adaptive(f, a, b, Epsilon, nWorkers)()
%%     assert(
%%       seqResult != 0.0 && Math.abs((seqResult-concResult)/seqResult) < 1E-7 ||
%%         Math.abs(seqResult-concResult) < 1E-10, ...)
%%   }
%% \end{scala}

Here are four ways this could be made more efficient.  Each aims to reduce
the number of communications (since communications are expensive).
%
\begin{itemize}
\item
Above the worker returns two tasks to the bag, and then (on the next
iteration) gets one back, possibly one of the tasks it just put there.  It
would be more efficient to just return one and to continue calculating with
the other.  In particular, the bag is likely to be a bottleneck, and this
reduces the load on the bag.

\item
The trapezium rule is applied to most intervals twice: once before that
interval is placed in the bag, and once when it is retrieved.  It would be
more efficient to put the calculated estimate into the bag along with the
interval, so it doesn't need to be recalculated.

\item
We could arrange for a |done| message to be sent only when the worker
does not return subtasks to the bag: the bag can infer that the worker
is done in the other case.

\item 
Each worker could accumulate sub-results in a thread-local variable, and send
to the adder only at the end. 

% \item
% We could merge the \SCALA{Adder} with the \SCALA{Bag}, and do away with the
% \SCALA{done} events.  All the workers will be idle if the number of tasks
% distributed equals the number of results received, plus half the number of
% tasks returned to the bag.  (If the change in the first bullet point is
% implemented, then that condition becomes that the number of tasks
% distributed equals the number of results received.)
\end{itemize}
\end{answer}

