\begin{question}
Suppose you are designing an interactive networked application.  You decide to
have one process handling actions by the local user, and another process
handling messages that are received over the network.  What could go wrong
with this design?  Describe how you could avoid these problems.
\end{question}

\begin{answer}
There's a great danger of race conditions.  For example, suppose the
local-action-handling process executes the code \SCALA{x = x+1}, and the
network-message-handling process executes the code \SCALA{x = x-1}; as we saw
in lectures, the read and write actions of these assignments could be
interleaved in many ways, giving unintended results.

One way to deal with this problem would be to code the rest of the program as
a third process.  When either of the other two processes receives a user
action or a network message, it sends a message to the main process.  

\begin{verbatim}
 ------------------------  laChan    -----------
 | local action handler | ---------> |         |
 ------------------------            | main    |
                                     | process |
 ------------------------  nmChan    |         |
 | network msg handler  | ---------> |         |
 ------------------------            -----------
\end{verbatim}

The main process is a sequential process, which repeatedly receives a message
from one of the other threads, and acts appropriately.  Something like:
%
\begin{scala}
serve( laChan ==> { ... }
     | nmChan ==> { ... }
)
\end{scala}
%
Hence the effects of the local actions or network messages are treated
atomically.

Alternatively ---and slightly harder--- we could use a Model-View-Controller
architecture with two controllers, namely the local-action-handling and
network-message-handling processes.  The Model and View are now a separate
process, that presents various atomic operations to the controllers.  This
process again is structured as a sequential \SCALA{serve} loop, as above.
Care needs to be taken that the atomic operations can be interleaved
arbitrarily. (In fact, the Model and View could sensibly be two independent
processes.)
\end{answer}


%%%%%%%%%%%%%%%%%%%%%%%%%%%%%%%%%%%%%%%%%%%%%%%%%%%%%%%

Alternatively, talk about programs branchin off threads to handle different
GUI events.
