\documentclass[12pt,a4paper]{article}

%\parindent=0.0cm
%\usepackage{epsf}
%\usepackage{graphicx}
%\usepackage{csp-cm}

\sloppy

\usepackage{url}
\begin{document}
\begin{center}
\Large\bf
Concurrent Programming, Hilary Term 2023 \\
General advice on practicals
\end{center}


Practicals are held in Weeks 2, 3, 4, 6, 7 and 8.  You should sign up
for practicals.

There are four practicals for this course:
%
\begin{itemize}
\item Practical 1: Sorting Networks; deadline Week 3 practicals.

\item Practical 2: Dining Philosophers; deadline Week 4 practicals.

\item Practical 3: Concurrent Depth-First Search; deadline Week 7
practicals.

\item Practical 4: Monitors and Semaphores; deadline Week~8
practicals.
\end{itemize}
%
Each practical must be marked during the practical sessions, by the deadline
mentioned.

Your report for each practical should be in the form of a well-commented
program, using the SCL library, and including answers to any specific
questions.  You are encouraged to ask the demonstrators for feedback about
each practical report.

All the practicals have parts marked as {\em optional}.  It is
possible to get a score of $S$ without doing these parts; however,
these parts are necessary for you to score $S+$.

%%%%%%%%%%%%%%%%%%%%%%%%%%%%%%%%%%%%%%%%%%%%%%%%%%%%%%%

\subsection*{Practical 0}

This is an un-assessed informal practical, designed to get you up and running
with Scala and SCL\@.  You are advised to complete this before the first
practical session. 
%
\begin{enumerate}
\item If you don't already have it, install Scala (version 2.13) from the web.
  Scala is already installed on Department machines.

\item Download the SCL library from the course webpage.

\item Download an example program from the course webpage,
  for example \texttt{Race.scala}. 

\item Compile the program using \texttt{scalac} or \texttt{fsc}.  You will
  need to include the SCL library (more precisely, the directory that contains
  the \texttt{ox} directory) on your classpath.  For example, if the
  \texttt{ox} directory is in \texttt{/home/alice/CP}, then the following
  should work: 
  \begin{quote}\tt
  fsc -classpath .:/home/alice/CP Race.scala
  \end{quote}

\item Run the program using \texttt{scala}.  You again need to include the SCL
  library in your classpath, for example
  \begin{quote}\tt  
  scala -classpath .:/home/alice/CP Race
  \end{quote}

\item (Optional.)  It might save time in the long term to automatically
  include the SCL files in your classpath.  For example, if you use bash, add
  the line \texttt{export CLASSPATH=/home/alice/CP:\$CLASSPATH} to your
  \texttt{.bashrc} file, then type {\tt source .bashrc} at the shell prompt.
  You should then be able to compile and run programs without the
  ``\texttt{-classpath}'' option.
\end{enumerate}

The course webpage contains links to Scala and SCL documentation.

\begin{flushright}
Gavin Lowe
\end{flushright}


\end{document}
